%% Generated by Sphinx.
\def\sphinxdocclass{report}
\documentclass[letterpaper,10pt,english]{sphinxmanual}
\ifdefined\pdfpxdimen
   \let\sphinxpxdimen\pdfpxdimen\else\newdimen\sphinxpxdimen
\fi \sphinxpxdimen=.75bp\relax
\ifdefined\pdfimageresolution
    \pdfimageresolution= \numexpr \dimexpr1in\relax/\sphinxpxdimen\relax
\fi
%% let collapsible pdf bookmarks panel have high depth per default
\PassOptionsToPackage{bookmarksdepth=5}{hyperref}

\PassOptionsToPackage{warn}{textcomp}
\usepackage[utf8]{inputenc}
\ifdefined\DeclareUnicodeCharacter
% support both utf8 and utf8x syntaxes
  \ifdefined\DeclareUnicodeCharacterAsOptional
    \def\sphinxDUC#1{\DeclareUnicodeCharacter{"#1}}
  \else
    \let\sphinxDUC\DeclareUnicodeCharacter
  \fi
  \sphinxDUC{00A0}{\nobreakspace}
  \sphinxDUC{2500}{\sphinxunichar{2500}}
  \sphinxDUC{2502}{\sphinxunichar{2502}}
  \sphinxDUC{2514}{\sphinxunichar{2514}}
  \sphinxDUC{251C}{\sphinxunichar{251C}}
  \sphinxDUC{2572}{\textbackslash}
\fi
\usepackage{cmap}
\usepackage[T1]{fontenc}
\usepackage{amsmath,amssymb,amstext}
\usepackage{babel}



\usepackage{tgtermes}
\usepackage{tgheros}
\renewcommand{\ttdefault}{txtt}



\usepackage[Bjarne]{fncychap}
\usepackage{sphinx}

\fvset{fontsize=auto}
\usepackage{geometry}


% Include hyperref last.
\usepackage{hyperref}
% Fix anchor placement for figures with captions.
\usepackage{hypcap}% it must be loaded after hyperref.
% Set up styles of URL: it should be placed after hyperref.
\urlstyle{same}

\addto\captionsenglish{\renewcommand{\contentsname}{Contents:}}

\usepackage{sphinxmessages}
\setcounter{tocdepth}{1}



\title{Open FDEM Post\sphinxhyphen{}Processing}
\date{Dec 06, 2022}
\release{1.0}
\author{OpenFDEM 2022}
\newcommand{\sphinxlogo}{\vbox{}}
\renewcommand{\releasename}{Release}
\makeindex
\begin{document}

\ifdefined\shorthandoff
  \ifnum\catcode`\=\string=\active\shorthandoff{=}\fi
  \ifnum\catcode`\"=\active\shorthandoff{"}\fi
\fi

\pagestyle{empty}
\sphinxmaketitle
\pagestyle{plain}
\sphinxtableofcontents
\pagestyle{normal}
\phantomsection\label{\detokenize{index::doc}}


\sphinxstepscope


\chapter{Introduction}
\label{\detokenize{postprocessing_intro:introduction}}\label{\detokenize{postprocessing_intro::doc}}
\sphinxAtStartPar
This Python package performs transformations on hybrid finite\sphinxhyphen{}discrete element method (FDEM) models with an unstructured grid in vtk/vtu/vtp format. It currently supports arrays of simulation files from the FDEM solvers:
\begin{itemize}
\item {} 
\sphinxAtStartPar
\sphinxhref{https://www.geomechanica.com/software/}{Irazu},

\item {} 
\sphinxAtStartPar
\sphinxhref{https://geogroup.utoronto.ca/software/}{Y\sphinxhyphen{}Geo} (and its common derivatives), as well as

\item {} 
\sphinxAtStartPar
\sphinxhref{https://openfdem.com/html/index.html}{OpenFDEM}.

\end{itemize}

\sphinxAtStartPar
The package is heavily dependent on \sphinxcode{\sphinxupquote{pyvista}} and is limited to \sphinxcode{\sphinxupquote{Python \textgreater{}=3.5, \textless{}=3.9}}. The package is maintained by the \sphinxhref{https://geogroup.utoronto.ca/}{Grasselli’s Geomechanics Group} at the University of Toronto, Canada, and is part of a collaborative effort by the open\sphinxhyphen{}source pacakge \sphinxhref{https://openfdem.com/html/index.html}{OpenFDEM}.


\section{Functionality}
\label{\detokenize{postprocessing_intro:functionality}}
\sphinxAtStartPar
The functionality of this script was developed with the objective of extracting common information needed when running simulations. Highlights of the script are:
\begin{itemize}
\item {} 
\sphinxAtStartPar
Extract information within the FDEM Model based on the name of the array (e.g., Stress, Strain, Temperature, etc…) Works in 2D and 3D.

\item {} 
\sphinxAtStartPar
Extract stress\sphinxhyphen{}strain information for UCS and BD Simulations (Works in 2D and 3D). Optional addition of virtual strain gauges (Limited to 2D).

\item {} 
\sphinxAtStartPar
Plotting stress vs strain curve.

\end{itemize}

\noindent{\hspace*{\fill}\sphinxincludegraphics[width=250\sphinxpxdimen]{{stress_strain}.png}\hspace*{\fill}}
\begin{itemize}
\item {} 
\sphinxAtStartPar
Calculate the Elastic Modulus of the dataset. Eavg, Esec and Etan can be evaluated. Works in 2D and 3D.

\end{itemize}

\begin{sphinxVerbatim}[commandchars=\\\{\}]
\PYG{g+gp}{\PYGZsh{} }Variants of E tangent
\PYG{g+go}{Etan at 50\PYGZpc{}: 51683.94MPa}
\PYG{g+go}{Etan at 50\PYGZpc{} with linear best fit disabled: 51639.22MPa}
\PYG{g+go}{Etan at 50\PYGZpc{} using strain gauge data: 50275.03MPa}
\PYG{g+gp}{\PYGZsh{} }Variants of E secant
\PYG{g+go}{Esec at 70\PYGZpc{}: 51681.01MPa}
\PYG{g+go}{Esec at 50\PYGZpc{}: 51817.43MPa}
\PYG{g+gp}{\PYGZsh{} }Variants of E average
\PYG{g+go}{Eavg between 50\PYGZhy{}60\PYGZpc{}: 51594.49MPa}
\PYG{g+go}{Eavg between 20\PYGZhy{}70\PYGZpc{} with linear best fit disabled: 51660.62MPa}
\end{sphinxVerbatim}
\begin{itemize}
\item {} 
\sphinxAtStartPar
Extract information of a particular cell based on a sequence of array names. This can be extended to extracting information along a line. Works in 2D and 3D.

\end{itemize}

\noindent{\hspace*{\fill}\sphinxincludegraphics[width=250\sphinxpxdimen]{{plot_point_over_time}.png}\hspace*{\fill}}
\begin{itemize}
\item {} 
\sphinxAtStartPar
Extract information of a threshold dataset criteria based on a sequence of array names. Works in 2D and 3D.

\end{itemize}

\noindent{\hspace*{\fill}\sphinxincludegraphics[width=250\sphinxpxdimen]{{temp_evolution}.png}\hspace*{\fill}}
\begin{itemize}
\item {} 
\sphinxAtStartPar
Extract mesh information and plot rosette/polar plots. Limited to 2D.

\end{itemize}

\noindent{\hspace*{\fill}\sphinxincludegraphics[width=250\sphinxpxdimen]{{mesh_rose_diagram}.png}\hspace*{\fill}}
\begin{itemize}
\item {} 
\sphinxAtStartPar
Automatic detection/ User\sphinxhyphen{}defined assigment of loading direction when analysing mechanical simulations, namely UCS, BD, and PLT, in both 2D and 3D simulations.

\end{itemize}

\begin{sphinxVerbatim}[commandchars=\\\{\}]
\PYG{g+go}{Script Identifying Platen}
\PYG{g+go}{	Platen Material ID found as [1]}
\PYG{g+go}{	3D Loading direction detected as [1] is Y\PYGZhy{}direction}
\PYG{g+go}{Values used in calculations are}
\PYG{g+go}{	Area	3721.00}
\PYG{g+go}{	Length	122.00}
\PYG{g+go}{Progress: |//////////////////////////////////////////////////| 100.0\PYGZpc{} Complete}
\end{sphinxVerbatim}


\section{Additional Support}
\label{\detokenize{postprocessing_intro:additional-support}}
\sphinxAtStartPar
Please refer to the user manual for detailed information pertaining to the various functions and their usage/arguments. For specific script requests and bug, please report them on our \sphinxhref{https://github.com/OpenFDEM}{github page}.

\sphinxstepscope


\chapter{openfdem}
\label{\detokenize{modules:openfdem}}\label{\detokenize{modules::doc}}
\sphinxstepscope


\section{openfdem package}
\label{\detokenize{openfdem:openfdem-package}}\label{\detokenize{openfdem::doc}}

\subsection{Submodules}
\label{\detokenize{openfdem:submodules}}

\subsection{openfdem.openfdem module}
\label{\detokenize{openfdem:module-openfdem.openfdem}}\label{\detokenize{openfdem:openfdem-openfdem-module}}\index{module@\spxentry{module}!openfdem.openfdem@\spxentry{openfdem.openfdem}}\index{openfdem.openfdem@\spxentry{openfdem.openfdem}!module@\spxentry{module}}\index{Model (class in openfdem.openfdem)@\spxentry{Model}\spxextra{class in openfdem.openfdem}}

\begin{fulllineitems}
\phantomsection\label{\detokenize{openfdem:openfdem.openfdem.Model}}
\pysigstartsignatures
\pysiglinewithargsret{\sphinxbfcode{\sphinxupquote{class\DUrole{w}{  }}}\sphinxcode{\sphinxupquote{openfdem.openfdem.}}\sphinxbfcode{\sphinxupquote{Model}}}{\emph{\DUrole{n}{dir\_path}\DUrole{o}{=}\DUrole{default_value}{None}}, \emph{\DUrole{n}{runfile}\DUrole{o}{=}\DUrole{default_value}{None}}, \emph{\DUrole{n}{fdem\_engine}\DUrole{o}{=}\DUrole{default_value}{None}}}{}
\pysigstopsignatures
\sphinxAtStartPar
Bases: \sphinxcode{\sphinxupquote{object}}

\sphinxAtStartPar
Collects datafiles into one Class. Returns the data array ordered by simulation timestep.
\begin{quote}\begin{description}
\sphinxlineitem{Example}
\begin{sphinxVerbatim}[commandchars=\\\{\}]
\PYG{g+gp}{\PYGZgt{}\PYGZgt{}\PYGZgt{} }\PYG{k+kn}{import} \PYG{n+nn}{openfdem} \PYG{k}{as} \PYG{n+nn}{fdem}
\PYG{g+gp}{\PYGZgt{}\PYGZgt{}\PYGZgt{} }\PYG{n}{model} \PYG{o}{=} \PYG{n}{fdem}\PYG{o}{.}\PYG{n}{Model}\PYG{p}{(}\PYG{l+s+s2}{\PYGZdq{}}\PYG{l+s+s2}{../example\PYGZus{}outputs/Irazu\PYGZus{}UCS}\PYG{l+s+s2}{\PYGZdq{}}\PYG{p}{)}
\end{sphinxVerbatim}

\end{description}\end{quote}
\index{Eavg\_mod() (openfdem.openfdem.Model method)@\spxentry{Eavg\_mod()}\spxextra{openfdem.openfdem.Model method}}

\begin{fulllineitems}
\phantomsection\label{\detokenize{openfdem:openfdem.openfdem.Model.Eavg_mod}}
\pysigstartsignatures
\pysiglinewithargsret{\sphinxbfcode{\sphinxupquote{Eavg\_mod}}}{\emph{\DUrole{n}{ucs\_data}}, \emph{\DUrole{n}{upperrange}}, \emph{\DUrole{n}{lowerrange}}, \emph{\DUrole{n}{linear\_bestfit}\DUrole{o}{=}\DUrole{default_value}{True}}, \emph{\DUrole{n}{loc\_stress}\DUrole{o}{=}\DUrole{default_value}{\textquotesingle{}Platen Stress\textquotesingle{}}}, \emph{\DUrole{n}{loc\_strain}\DUrole{o}{=}\DUrole{default_value}{\textquotesingle{}Platen Strain\textquotesingle{}}}}{}
\pysigstopsignatures
\sphinxAtStartPar
Average Elastic modulus between two ranges
\begin{quote}\begin{description}
\sphinxlineitem{Parameters}\begin{itemize}
\item {} 
\sphinxAtStartPar
\sphinxstyleliteralstrong{\sphinxupquote{ucs\_data}} (\sphinxstyleliteralemphasis{\sphinxupquote{pandas.DataFrame}}) \textendash{} DataFrame containing the stress\sphinxhyphen{}strain data

\item {} 
\sphinxAtStartPar
\sphinxstyleliteralstrong{\sphinxupquote{upperrange}} (\sphinxstyleliteralemphasis{\sphinxupquote{float}}) \textendash{} Upper range to calculate the average

\item {} 
\sphinxAtStartPar
\sphinxstyleliteralstrong{\sphinxupquote{lowerrange}} (\sphinxstyleliteralemphasis{\sphinxupquote{float}}) \textendash{} Lower range to calculate the average

\item {} 
\sphinxAtStartPar
\sphinxstyleliteralstrong{\sphinxupquote{linear\_bestfit}} (\sphinxstyleliteralemphasis{\sphinxupquote{bool}}) \textendash{} Calculate data based on range extents or linear best fit line.

\item {} 
\sphinxAtStartPar
\sphinxstyleliteralstrong{\sphinxupquote{loc\_stress}} (\sphinxstyleliteralemphasis{\sphinxupquote{str}}) \textendash{} Column to obtain stress from. Defaults to Platen Stress

\item {} 
\sphinxAtStartPar
\sphinxstyleliteralstrong{\sphinxupquote{loc\_strain}} (\sphinxstyleliteralemphasis{\sphinxupquote{str}}) \textendash{} Column to obtain strain from. Defaults to Platen Strain

\end{itemize}

\sphinxlineitem{Returns}
\sphinxAtStartPar
Average Elastic modulus

\sphinxlineitem{Return type}
\sphinxAtStartPar
list{[}float{]}

\sphinxlineitem{Raises}
\sphinxAtStartPar
\sphinxstyleliteralstrong{\sphinxupquote{ZeroDivisionError}} \textendash{} The range over which to calculate the Eavg is too small. Consider a larger range.

\sphinxlineitem{Example}
\begin{sphinxVerbatim}[commandchars=\\\{\}]
\PYG{g+gp}{\PYGZgt{}\PYGZgt{}\PYGZgt{} }\PYG{k+kn}{import} \PYG{n+nn}{openfdem} \PYG{k}{as} \PYG{n+nn}{fdem}
\PYG{g+gp}{\PYGZgt{}\PYGZgt{}\PYGZgt{} }\PYG{n}{data} \PYG{o}{=} \PYG{n}{fdem}\PYG{o}{.}\PYG{n}{Model}\PYG{p}{(}\PYG{l+s+s2}{\PYGZdq{}}\PYG{l+s+s2}{../example\PYGZus{}outputs/Irazu\PYGZus{}UCS}\PYG{l+s+s2}{\PYGZdq{}}\PYG{p}{)}
\PYG{g+gp}{\PYGZgt{}\PYGZgt{}\PYGZgt{} }\PYG{n}{df\PYGZus{}1} \PYG{o}{=} \PYG{n}{data}\PYG{o}{.}\PYG{n}{complete\PYGZus{}UCS\PYGZus{}stress\PYGZus{}strain}\PYG{p}{(}\PYG{n}{st\PYGZus{}status}\PYG{o}{=}\PYG{k+kc}{True}\PYG{p}{)}
\PYG{g+gp}{\PYGZgt{}\PYGZgt{}\PYGZgt{} }\PYG{n}{data}\PYG{o}{.}\PYG{n}{Eavg\PYGZus{}mod}\PYG{p}{(}\PYG{n}{df\PYGZus{}1}\PYG{p}{,} \PYG{l+m+mf}{0.5}\PYG{p}{,} \PYG{l+m+mf}{0.6}\PYG{p}{)}\PYG{p}{[}\PYG{l+m+mi}{0}\PYG{p}{]}
\PYG{g+go}{51594.490217007056}
\PYG{g+gp}{\PYGZgt{}\PYGZgt{}\PYGZgt{} }\PYG{n}{data}\PYG{o}{.}\PYG{n}{Eavg\PYGZus{}mod}\PYG{p}{(}\PYG{n}{df\PYGZus{}1}\PYG{p}{,} \PYG{l+m+mf}{0.5}\PYG{p}{,} \PYG{l+m+mf}{0.6}\PYG{p}{,} \PYG{n}{loc\PYGZus{}strain}\PYG{o}{=}\PYG{l+s+s1}{\PYGZsq{}}\PYG{l+s+s1}{Gauge Displacement Y}\PYG{l+s+s1}{\PYGZsq{}}\PYG{p}{)}\PYG{p}{[}\PYG{l+m+mi}{0}\PYG{p}{]}
\PYG{g+go}{51110.06292512191}
\end{sphinxVerbatim}

\end{description}\end{quote}

\end{fulllineitems}

\index{Esec\_mod() (openfdem.openfdem.Model method)@\spxentry{Esec\_mod()}\spxextra{openfdem.openfdem.Model method}}

\begin{fulllineitems}
\phantomsection\label{\detokenize{openfdem:openfdem.openfdem.Model.Esec_mod}}
\pysigstartsignatures
\pysiglinewithargsret{\sphinxbfcode{\sphinxupquote{Esec\_mod}}}{\emph{\DUrole{n}{ucs\_data}}, \emph{\DUrole{n}{upperrange}}, \emph{\DUrole{n}{loc\_stress}\DUrole{o}{=}\DUrole{default_value}{\textquotesingle{}Platen Stress\textquotesingle{}}}, \emph{\DUrole{n}{loc\_strain}\DUrole{o}{=}\DUrole{default_value}{\textquotesingle{}Platen Strain\textquotesingle{}}}}{}
\pysigstopsignatures
\sphinxAtStartPar
Secant Modulus between 0 and upperrange. The upperrange can be a \% or a fraction.
\begin{quote}\begin{description}
\sphinxlineitem{Parameters}\begin{itemize}
\item {} 
\sphinxAtStartPar
\sphinxstyleliteralstrong{\sphinxupquote{ucs\_data}} (\sphinxstyleliteralemphasis{\sphinxupquote{pandas.DataFrame}}) \textendash{} DataFrame containing the stress\sphinxhyphen{}strain data

\item {} 
\sphinxAtStartPar
\sphinxstyleliteralstrong{\sphinxupquote{upperrange}} (\sphinxstyleliteralemphasis{\sphinxupquote{float}}) \textendash{} Range over which to calculate the Secant Modulus

\item {} 
\sphinxAtStartPar
\sphinxstyleliteralstrong{\sphinxupquote{loc\_stress}} (\sphinxstyleliteralemphasis{\sphinxupquote{str}}) \textendash{} Column to obtain stress from. Defaults to Platen Stress

\item {} 
\sphinxAtStartPar
\sphinxstyleliteralstrong{\sphinxupquote{loc\_strain}} (\sphinxstyleliteralemphasis{\sphinxupquote{str}}) \textendash{} Column to obtain strain from. Defaults to Platen Strain

\end{itemize}

\sphinxlineitem{Returns}
\sphinxAtStartPar
Secant Elastic modulus between 0 and upperrange

\sphinxlineitem{Return type}
\sphinxAtStartPar
float

\sphinxlineitem{Example}
\begin{sphinxVerbatim}[commandchars=\\\{\}]
\PYG{g+gp}{\PYGZgt{}\PYGZgt{}\PYGZgt{} }\PYG{k+kn}{import} \PYG{n+nn}{openfdem} \PYG{k}{as} \PYG{n+nn}{fdem}
\PYG{g+gp}{\PYGZgt{}\PYGZgt{}\PYGZgt{} }\PYG{n}{data} \PYG{o}{=} \PYG{n}{fdem}\PYG{o}{.}\PYG{n}{Model}\PYG{p}{(}\PYG{l+s+s2}{\PYGZdq{}}\PYG{l+s+s2}{../example\PYGZus{}outputs/Irazu\PYGZus{}UCS}\PYG{l+s+s2}{\PYGZdq{}}\PYG{p}{)}
\PYG{g+gp}{\PYGZgt{}\PYGZgt{}\PYGZgt{} }\PYG{n}{df\PYGZus{}1} \PYG{o}{=} \PYG{n}{data}\PYG{o}{.}\PYG{n}{complete\PYGZus{}UCS\PYGZus{}stress\PYGZus{}strain}\PYG{p}{(}\PYG{n}{st\PYGZus{}status}\PYG{o}{=}\PYG{k+kc}{True}\PYG{p}{)}
\PYG{g+gp}{\PYGZgt{}\PYGZgt{}\PYGZgt{} }\PYG{n}{data}\PYG{o}{.}\PYG{n}{Esec\PYGZus{}mod}\PYG{p}{(}\PYG{n}{df\PYGZus{}1}\PYG{p}{,} \PYG{l+m+mf}{0.5}\PYG{p}{)}
\PYG{g+go}{51817.43019752671}
\PYG{g+gp}{\PYGZgt{}\PYGZgt{}\PYGZgt{} }\PYG{n}{data}\PYG{o}{.}\PYG{n}{Esec\PYGZus{}mod}\PYG{p}{(}\PYG{n}{df\PYGZus{}1}\PYG{p}{,} \PYG{l+m+mf}{0.5}\PYG{p}{,} \PYG{n}{loc\PYGZus{}strain}\PYG{o}{=}\PYG{l+s+s1}{\PYGZsq{}}\PYG{l+s+s1}{Gauge Displacement Y}\PYG{l+s+s1}{\PYGZsq{}}\PYG{p}{)}
\PYG{g+go}{51355.860814069754}
\end{sphinxVerbatim}

\end{description}\end{quote}

\end{fulllineitems}

\index{Etan50\_mod() (openfdem.openfdem.Model method)@\spxentry{Etan50\_mod()}\spxextra{openfdem.openfdem.Model method}}

\begin{fulllineitems}
\phantomsection\label{\detokenize{openfdem:openfdem.openfdem.Model.Etan50_mod}}
\pysigstartsignatures
\pysiglinewithargsret{\sphinxbfcode{\sphinxupquote{Etan50\_mod}}}{\emph{\DUrole{n}{ucs\_data}}, \emph{\DUrole{n}{linear\_bestfit}\DUrole{o}{=}\DUrole{default_value}{True}}, \emph{\DUrole{n}{loc\_stress}\DUrole{o}{=}\DUrole{default_value}{\textquotesingle{}Platen Stress\textquotesingle{}}}, \emph{\DUrole{n}{loc\_strain}\DUrole{o}{=}\DUrole{default_value}{\textquotesingle{}Platen Strain\textquotesingle{}}}, \emph{\DUrole{n}{plusminus\_range}\DUrole{o}{=}\DUrole{default_value}{1}}}{}
\pysigstopsignatures
\sphinxAtStartPar
Tangent Elastic modulus at 50\%. Calculates +/\sphinxhyphen{} number of datapoint from the 50\% Stress. Defaults to +/\sphinxhyphen{} 1 datapoint.
\begin{quote}\begin{description}
\sphinxlineitem{Parameters}\begin{itemize}
\item {} 
\sphinxAtStartPar
\sphinxstyleliteralstrong{\sphinxupquote{ucs\_data}} (\sphinxstyleliteralemphasis{\sphinxupquote{pandas.DataFrame}}) \textendash{} DataFrame containing the stress\sphinxhyphen{}strain data

\item {} 
\sphinxAtStartPar
\sphinxstyleliteralstrong{\sphinxupquote{linear\_bestfit}} (\sphinxstyleliteralemphasis{\sphinxupquote{bool}}) \textendash{} Calculate data based on range extents or linear best fit line.

\item {} 
\sphinxAtStartPar
\sphinxstyleliteralstrong{\sphinxupquote{loc\_stress}} (\sphinxstyleliteralemphasis{\sphinxupquote{str}}) \textendash{} Column to obtain stress from. Defaults to Platen Stress

\item {} 
\sphinxAtStartPar
\sphinxstyleliteralstrong{\sphinxupquote{loc\_strain}} (\sphinxstyleliteralemphasis{\sphinxupquote{str}}) \textendash{} Column to obtain strain from. Defaults to Platen Strain

\item {} 
\sphinxAtStartPar
\sphinxstyleliteralstrong{\sphinxupquote{plusminus\_range}} (\sphinxstyleliteralemphasis{\sphinxupquote{int}}) \textendash{} Range over which to calculate the Elastic modulus

\end{itemize}

\sphinxlineitem{Returns}
\sphinxAtStartPar
Tangent Elastic modulus at 50\% as a slope and Y\sphinxhyphen{}Intercept. Y\sphinxhyphen{}Intercept = 0 if linear\_bestfit is False

\sphinxlineitem{Return type}
\sphinxAtStartPar
list{[}float{]}

\sphinxlineitem{Example}
\begin{sphinxVerbatim}[commandchars=\\\{\}]
\PYG{g+gp}{\PYGZgt{}\PYGZgt{}\PYGZgt{} }\PYG{k+kn}{import} \PYG{n+nn}{openfdem} \PYG{k}{as} \PYG{n+nn}{fdem}
\PYG{g+gp}{\PYGZgt{}\PYGZgt{}\PYGZgt{} }\PYG{n}{data} \PYG{o}{=} \PYG{n}{fdem}\PYG{o}{.}\PYG{n}{Model}\PYG{p}{(}\PYG{l+s+s2}{\PYGZdq{}}\PYG{l+s+s2}{../example\PYGZus{}outputs/Irazu\PYGZus{}UCS}\PYG{l+s+s2}{\PYGZdq{}}\PYG{p}{)}
\PYG{g+gp}{\PYGZgt{}\PYGZgt{}\PYGZgt{} }\PYG{n}{df\PYGZus{}1} \PYG{o}{=} \PYG{n}{data}\PYG{o}{.}\PYG{n}{complete\PYGZus{}UCS\PYGZus{}stress\PYGZus{}strain}\PYG{p}{(}\PYG{p}{)}
\PYG{g+gp}{\PYGZgt{}\PYGZgt{}\PYGZgt{} }\PYG{n}{data}\PYG{o}{.}\PYG{n}{Etan50\PYGZus{}mod}\PYG{p}{(}\PYG{n}{df\PYGZus{}1}\PYG{p}{)}\PYG{p}{[}\PYG{l+m+mi}{0}\PYG{p}{]}
\PYG{g+go}{51683.94337878284}
\PYG{g+gp}{\PYGZgt{}\PYGZgt{}\PYGZgt{} }\PYG{n}{data}\PYG{o}{.}\PYG{n}{Etan50\PYGZus{}mod}\PYG{p}{(}\PYG{n}{df\PYGZus{}1}\PYG{p}{,} \PYG{n}{linear\PYGZus{}bestfit}\PYG{o}{=}\PYG{k+kc}{False}\PYG{p}{)}\PYG{p}{[}\PYG{l+m+mi}{0}\PYG{p}{]}
\PYG{g+go}{51639.21679789497}
\PYG{g+gp}{\PYGZgt{}\PYGZgt{}\PYGZgt{} }\PYG{n}{df\PYGZus{}1} \PYG{o}{=} \PYG{n}{data}\PYG{o}{.}\PYG{n}{complete\PYGZus{}UCS\PYGZus{}stress\PYGZus{}strain}\PYG{p}{(}\PYG{n}{st\PYGZus{}status}\PYG{o}{=}\PYG{k+kc}{True}\PYG{p}{)}
\PYG{g+gp}{\PYGZgt{}\PYGZgt{}\PYGZgt{} }\PYG{n}{data}\PYG{o}{.}\PYG{n}{Etan50\PYGZus{}mod}\PYG{p}{(}\PYG{n}{df\PYGZus{}1}\PYG{p}{,} \PYG{n}{loc\PYGZus{}strain}\PYG{o}{=}\PYG{l+s+s1}{\PYGZsq{}}\PYG{l+s+s1}{Gauge Displacement Y}\PYG{l+s+s1}{\PYGZsq{}}\PYG{p}{,} \PYG{n}{plusminus\PYGZus{}range}\PYG{o}{=}\PYG{l+m+mi}{1}\PYG{p}{)}\PYG{p}{[}\PYG{l+m+mi}{0}\PYG{p}{]}
\PYG{g+go}{51216.33411269702}
\end{sphinxVerbatim}

\end{description}\end{quote}

\end{fulllineitems}

\index{complete\_BD\_stress\_strain() (openfdem.openfdem.Model method)@\spxentry{complete\_BD\_stress\_strain()}\spxextra{openfdem.openfdem.Model method}}

\begin{fulllineitems}
\phantomsection\label{\detokenize{openfdem:openfdem.openfdem.Model.complete_BD_stress_strain}}
\pysigstartsignatures
\pysiglinewithargsret{\sphinxbfcode{\sphinxupquote{complete\_BD\_stress\_strain}}}{\emph{\DUrole{n}{st\_status}\DUrole{o}{=}\DUrole{default_value}{False}}, \emph{\DUrole{n}{gauge\_width}\DUrole{o}{=}\DUrole{default_value}{0}}, \emph{\DUrole{n}{gauge\_length}\DUrole{o}{=}\DUrole{default_value}{0}}, \emph{\DUrole{n}{c\_center}\DUrole{o}{=}\DUrole{default_value}{None}}, \emph{\DUrole{n}{progress\_bar}\DUrole{o}{=}\DUrole{default_value}{True}}}{}
\pysigstopsignatures
\sphinxAtStartPar
Calculate the full stress\sphinxhyphen{}strain curve for an indirect tensile simulation.
\begin{quote}\begin{description}
\sphinxlineitem{Parameters}\begin{itemize}
\item {} 
\sphinxAtStartPar
\sphinxstyleliteralstrong{\sphinxupquote{st\_status}} (\sphinxstyleliteralemphasis{\sphinxupquote{bool}}) \textendash{} Enable/Disable SG

\item {} 
\sphinxAtStartPar
\sphinxstyleliteralstrong{\sphinxupquote{gauge\_width}} (\sphinxstyleliteralemphasis{\sphinxupquote{float}}) \textendash{} width of the virtual strain gauge

\item {} 
\sphinxAtStartPar
\sphinxstyleliteralstrong{\sphinxupquote{gauge\_length}} (\sphinxstyleliteralemphasis{\sphinxupquote{float}}) \textendash{} length of the virtual strain gauge

\item {} 
\sphinxAtStartPar
\sphinxstyleliteralstrong{\sphinxupquote{c\_center}} (\sphinxstyleliteralemphasis{\sphinxupquote{None}}\sphinxstyleliteralemphasis{\sphinxupquote{ or }}\sphinxstyleliteralemphasis{\sphinxupquote{list}}\sphinxstyleliteralemphasis{\sphinxupquote{{[}}}\sphinxstyleliteralemphasis{\sphinxupquote{float}}\sphinxstyleliteralemphasis{\sphinxupquote{, }}\sphinxstyleliteralemphasis{\sphinxupquote{float}}\sphinxstyleliteralemphasis{\sphinxupquote{, }}\sphinxstyleliteralemphasis{\sphinxupquote{float}}\sphinxstyleliteralemphasis{\sphinxupquote{{]}}}) \textendash{} User\sphinxhyphen{}defined center of the SG

\item {} 
\sphinxAtStartPar
\sphinxstyleliteralstrong{\sphinxupquote{progress\_bar}} (\sphinxstyleliteralemphasis{\sphinxupquote{bool}}) \textendash{} Show/Hide progress bar

\end{itemize}

\sphinxlineitem{Returns}
\sphinxAtStartPar
full stress\sphinxhyphen{}strain information

\sphinxlineitem{Return type}
\sphinxAtStartPar
pandas.DataFrame

\sphinxlineitem{Example}
\begin{sphinxVerbatim}[commandchars=\\\{\}]
\PYG{g+gp}{\PYGZgt{}\PYGZgt{}\PYGZgt{} }\PYG{k+kn}{import} \PYG{n+nn}{openfdem} \PYG{k}{as} \PYG{n+nn}{fdem}
\PYG{g+gp}{\PYGZgt{}\PYGZgt{}\PYGZgt{} }\PYG{n}{data} \PYG{o}{=} \PYG{n}{fdem}\PYG{o}{.}\PYG{n}{Model}\PYG{p}{(}\PYG{l+s+s2}{\PYGZdq{}}\PYG{l+s+s2}{/external/Speed\PYGZus{}Cal\PYGZus{}Using\PYGZus{}Flowstone/BD/BD\PYGZus{}c\PYGZus{}17\PYGZus{}5\PYGZus{}ts\PYGZus{}2\PYGZus{}55\PYGZus{}GII\PYGZus{}90000\PYGZus{}v\PYGZus{}0\PYGZus{}6}\PYG{l+s+s2}{\PYGZdq{}}\PYG{p}{)}
\PYG{g+go}{\PYGZsh{} full stress\PYGZhy{}strain without SG}
\PYG{g+gp}{\PYGZgt{}\PYGZgt{}\PYGZgt{} }\PYG{n}{df\PYGZus{}wo\PYGZus{}SG} \PYG{o}{=} \PYG{n}{data}\PYG{o}{.}\PYG{n}{complete\PYGZus{}BD\PYGZus{}stress\PYGZus{}strain}\PYG{p}{(}\PYG{k+kc}{False}\PYG{p}{)}
\PYG{g+go}{Columns:}
\PYG{g+go}{    Name: Platen Stress, dtype=float64, nullable: False}
\PYG{g+go}{    Name: Platen Strain, dtype=float64, nullable: False}
\PYG{g+go}{Script Identifying Platen}
\PYG{g+go}{    Platen Material ID found as [1]}
\PYG{g+go}{Progress: |//////////////////////////////////////////////////| 100.0\PYGZpc{} Complete}
\PYG{g+go}{\PYGZsh{} full stress\PYGZhy{}strain with SG and default dimensions}
\PYG{g+gp}{\PYGZgt{}\PYGZgt{}\PYGZgt{} }\PYG{n}{df\PYGZus{}Def\PYGZus{}SG} \PYG{o}{=} \PYG{n}{data}\PYG{o}{.}\PYG{n}{complete\PYGZus{}BD\PYGZus{}stress\PYGZus{}strain}\PYG{p}{(}\PYG{k+kc}{True}\PYG{p}{)}
\PYG{g+go}{Columns:}
\PYG{g+go}{    Name: Platen Stress, dtype=float64, nullable: False}
\PYG{g+go}{    Name: Platen Strain, dtype=float64, nullable: False}
\PYG{g+go}{    Name: Gauge Displacement X, dtype=float64, nullable: False}
\PYG{g+go}{    Name: Gauge Displacement Y, dtype=float64, nullable: False}
\PYG{g+go}{    Script Identifying Platen}
\PYG{g+go}{Platen Material ID found as [1]}
\PYG{g+go}{Dimensions of SG are 12.0 x 12.0}
\PYG{g+go}{Vertical Gauges}
\PYG{g+go}{    extends between [[6.0, \PYGZhy{}6.0, 0.0], [\PYGZhy{}6.0, \PYGZhy{}6.0, 0.0], [6.0, 6.0, 0.0], [\PYGZhy{}6.0, 6.0, 0.0]]}
\PYG{g+go}{    cover cells ID [3644, 7481, 4635, 2872]}
\PYG{g+go}{Horizontal Gauges}
\PYG{g+go}{    extends between [[6.0, 6.0, 0.0], [6.0, \PYGZhy{}6.0, 0.0], [\PYGZhy{}6.0, 6.0, 0.0], [\PYGZhy{}6.0, \PYGZhy{}6.0, 0.0]]}
\PYG{g+go}{    cover cells ID [4635, 3644, 2872, 7481]}
\PYG{g+go}{Progress: |//////////////////////////////////////////////////| 100.0\PYGZpc{} Complete}
\PYG{g+go}{\PYGZsh{} full stress\PYGZhy{}strain with SG and user\PYGZhy{}defined dimensions}
\PYG{g+gp}{\PYGZgt{}\PYGZgt{}\PYGZgt{} }\PYG{n}{df\PYGZus{}userdf\PYGZus{}SG} \PYG{o}{=} \PYG{n}{data}\PYG{o}{.}\PYG{n}{complete\PYGZus{}BD\PYGZus{}stress\PYGZus{}strain}\PYG{p}{(}\PYG{k+kc}{True}\PYG{p}{,} \PYG{l+m+mi}{10}\PYG{p}{,} \PYG{l+m+mi}{10}\PYG{p}{)}
\PYG{g+go}{Columns:}
\PYG{g+go}{    Name: Platen Stress, dtype=float64, nullable: False}
\PYG{g+go}{    Name: Platen Strain, dtype=float64, nullable: False}
\PYG{g+go}{    Name: Gauge Displacement X, dtype=float64, nullable: False}
\PYG{g+go}{    Name: Gauge Displacement Y, dtype=float64, nullable: False}
\PYG{g+go}{Script Identifying Platen}
\PYG{g+go}{    Platen Material ID found as [1]}
\PYG{g+go}{    Dimensions of SG are 10 x 10}
\PYG{g+go}{    Vertical Gauges}
\PYG{g+go}{        extends between [[5.0, \PYGZhy{}5.0, 0.0], [\PYGZhy{}5.0, \PYGZhy{}5.0, 0.0], [5.0, 5.0, 0.0], [\PYGZhy{}5.0, 5.0, 0.0]]}
\PYG{g+go}{        cover cells ID [1898, 5999, 5249, 6806]}
\PYG{g+go}{    Horizontal Gauges}
\PYG{g+go}{        extends between [[5.0, 5.0, 0.0], [5.0, \PYGZhy{}5.0, 0.0], [\PYGZhy{}5.0, 5.0, 0.0], [\PYGZhy{}5.0, \PYGZhy{}5.0, 0.0]]}
\PYG{g+go}{        cover cells ID [5249, 1898, 6806, 5999]}
\PYG{g+go}{Progress: |//////////////////////////////////////////////////| 100.0\PYGZpc{} Complete}
\end{sphinxVerbatim}

\end{description}\end{quote}

\end{fulllineitems}

\index{complete\_PLT\_stress\_strain() (openfdem.openfdem.Model method)@\spxentry{complete\_PLT\_stress\_strain()}\spxextra{openfdem.openfdem.Model method}}

\begin{fulllineitems}
\phantomsection\label{\detokenize{openfdem:openfdem.openfdem.Model.complete_PLT_stress_strain}}
\pysigstartsignatures
\pysiglinewithargsret{\sphinxbfcode{\sphinxupquote{complete\_PLT\_stress\_strain}}}{\emph{\DUrole{n}{load\_config}}, \emph{\DUrole{n}{platen\_id}\DUrole{o}{=}\DUrole{default_value}{None}}, \emph{\DUrole{n}{axis\_of\_loading}\DUrole{o}{=}\DUrole{default_value}{None}}, \emph{\DUrole{n}{De\_squared}\DUrole{o}{=}\DUrole{default_value}{None}}, \emph{\DUrole{n}{progress\_bar}\DUrole{o}{=}\DUrole{default_value}{True}}}{}
\pysigstopsignatures
\sphinxAtStartPar
Calculate the full stress\sphinxhyphen{}strain curve for a point load simulation.
\begin{quote}\begin{description}
\sphinxlineitem{Parameters}\begin{itemize}
\item {} 
\sphinxAtStartPar
\sphinxstyleliteralstrong{\sphinxupquote{load\_config}} (\sphinxstyleliteralemphasis{\sphinxupquote{str}}) \textendash{} type of PLT Test. “A” “D” “B”

\item {} 
\sphinxAtStartPar
\sphinxstyleliteralstrong{\sphinxupquote{platen\_id}} (\sphinxstyleliteralemphasis{\sphinxupquote{None}}\sphinxstyleliteralemphasis{\sphinxupquote{ or }}\sphinxstyleliteralemphasis{\sphinxupquote{int}}) \textendash{} Manual override of Platen ID

\item {} 
\sphinxAtStartPar
\sphinxstyleliteralstrong{\sphinxupquote{axis\_of\_loading}} (\sphinxstyleliteralemphasis{\sphinxupquote{None}}\sphinxstyleliteralemphasis{\sphinxupquote{ or }}\sphinxstyleliteralemphasis{\sphinxupquote{int}}) \textendash{} Loading Direction

\item {} 
\sphinxAtStartPar
\sphinxstyleliteralstrong{\sphinxupquote{De\_squared}} (\sphinxstyleliteralemphasis{\sphinxupquote{None}}\sphinxstyleliteralemphasis{\sphinxupquote{ or }}\sphinxstyleliteralemphasis{\sphinxupquote{float}}) \textendash{} equivalent core diameter (i.e., the value of De\_squared)

\item {} 
\sphinxAtStartPar
\sphinxstyleliteralstrong{\sphinxupquote{progress\_bar}} (\sphinxstyleliteralemphasis{\sphinxupquote{bool}}) \textendash{} Show/Hide progress bar

\end{itemize}

\sphinxlineitem{Returns}
\sphinxAtStartPar
full stress\sphinxhyphen{}strain information

\sphinxlineitem{Return type}
\sphinxAtStartPar
pandas.DataFrame

\sphinxlineitem{Example}
\begin{sphinxVerbatim}[commandchars=\\\{\}]
\PYG{g+gp}{\PYGZgt{}\PYGZgt{}\PYGZgt{} }\PYG{k+kn}{import} \PYG{n+nn}{openfdem} \PYG{k}{as} \PYG{n+nn}{fdem}
\PYG{g+gp}{\PYGZgt{}\PYGZgt{}\PYGZgt{} }\PYG{n}{data} \PYG{o}{=} \PYG{n}{fdem}\PYG{o}{.}\PYG{n}{Model}\PYG{p}{(}\PYG{l+s+s2}{\PYGZdq{}}\PYG{l+s+s2}{/external/Yusuf\PYGZus{}PLT/Axial}\PYG{l+s+s2}{\PYGZdq{}}\PYG{p}{)}
\PYG{g+go}{\PYGZsh{} Minimal Arguments}
\PYG{g+gp}{\PYGZgt{}\PYGZgt{}\PYGZgt{} }\PYG{n}{df} \PYG{o}{=} \PYG{n}{data}\PYG{o}{.}\PYG{n}{complete\PYGZus{}PLT\PYGZus{}stress\PYGZus{}strain}\PYG{p}{(}\PYG{n}{load\PYGZus{}config}\PYG{o}{=}\PYG{l+s+s2}{\PYGZdq{}}\PYG{l+s+s2}{A}\PYG{l+s+s2}{\PYGZdq{}}\PYG{p}{,} \PYG{n}{platen\PYGZus{}id}\PYG{o}{=}\PYG{l+m+mi}{1}\PYG{p}{)}
\PYG{g+go}{Columns:}
\PYG{g+go}{    Name: Platen Stress, dtype=float64, nullable: False}
\end{sphinxVerbatim}

\end{description}\end{quote}

\end{fulllineitems}

\index{complete\_UCS\_stress\_strain() (openfdem.openfdem.Model method)@\spxentry{complete\_UCS\_stress\_strain()}\spxextra{openfdem.openfdem.Model method}}

\begin{fulllineitems}
\phantomsection\label{\detokenize{openfdem:openfdem.openfdem.Model.complete_UCS_stress_strain}}
\pysigstartsignatures
\pysiglinewithargsret{\sphinxbfcode{\sphinxupquote{complete\_UCS\_stress\_strain}}}{\emph{\DUrole{n}{platen\_id}\DUrole{o}{=}\DUrole{default_value}{None}}, \emph{\DUrole{n}{st\_status}\DUrole{o}{=}\DUrole{default_value}{False}}, \emph{\DUrole{n}{axis\_of\_loading}\DUrole{o}{=}\DUrole{default_value}{None}}, \emph{\DUrole{n}{gauge\_width}\DUrole{o}{=}\DUrole{default_value}{0}}, \emph{\DUrole{n}{gauge\_length}\DUrole{o}{=}\DUrole{default_value}{0}}, \emph{\DUrole{n}{c\_center}\DUrole{o}{=}\DUrole{default_value}{None}}, \emph{\DUrole{n}{samp\_A}\DUrole{o}{=}\DUrole{default_value}{None}}, \emph{\DUrole{n}{samp\_L}\DUrole{o}{=}\DUrole{default_value}{None}}, \emph{\DUrole{n}{progress\_bar}\DUrole{o}{=}\DUrole{default_value}{True}}}{}
\pysigstopsignatures
\sphinxAtStartPar
Calculate the full stress\sphinxhyphen{}strain curve for a uniaxial compressive strength simulation
\begin{quote}\begin{description}
\sphinxlineitem{Parameters}\begin{itemize}
\item {} 
\sphinxAtStartPar
\sphinxstyleliteralstrong{\sphinxupquote{platen\_id}} (\sphinxstyleliteralemphasis{\sphinxupquote{None}}\sphinxstyleliteralemphasis{\sphinxupquote{ or }}\sphinxstyleliteralemphasis{\sphinxupquote{int}}) \textendash{} Manual override of Platen ID

\item {} 
\sphinxAtStartPar
\sphinxstyleliteralstrong{\sphinxupquote{st\_status}} (\sphinxstyleliteralemphasis{\sphinxupquote{bool}}) \textendash{} Enable/Disable SG

\item {} 
\sphinxAtStartPar
\sphinxstyleliteralstrong{\sphinxupquote{axis\_of\_loading}} (\sphinxstyleliteralemphasis{\sphinxupquote{None}}\sphinxstyleliteralemphasis{\sphinxupquote{ or }}\sphinxstyleliteralemphasis{\sphinxupquote{int}}) \textendash{} Loading Direction

\item {} 
\sphinxAtStartPar
\sphinxstyleliteralstrong{\sphinxupquote{gauge\_width}} (\sphinxstyleliteralemphasis{\sphinxupquote{float}}) \textendash{} width of the virtual strain gauge

\item {} 
\sphinxAtStartPar
\sphinxstyleliteralstrong{\sphinxupquote{gauge\_length}} (\sphinxstyleliteralemphasis{\sphinxupquote{float}}) \textendash{} length of the virtual strain gauge

\item {} 
\sphinxAtStartPar
\sphinxstyleliteralstrong{\sphinxupquote{c\_center}} (\sphinxstyleliteralemphasis{\sphinxupquote{None}}\sphinxstyleliteralemphasis{\sphinxupquote{ or }}\sphinxstyleliteralemphasis{\sphinxupquote{list}}\sphinxstyleliteralemphasis{\sphinxupquote{{[}}}\sphinxstyleliteralemphasis{\sphinxupquote{float}}\sphinxstyleliteralemphasis{\sphinxupquote{, }}\sphinxstyleliteralemphasis{\sphinxupquote{float}}\sphinxstyleliteralemphasis{\sphinxupquote{, }}\sphinxstyleliteralemphasis{\sphinxupquote{float}}\sphinxstyleliteralemphasis{\sphinxupquote{{]}}}) \textendash{} User\sphinxhyphen{}defined center of the SG

\item {} 
\sphinxAtStartPar
\sphinxstyleliteralstrong{\sphinxupquote{samp\_A}} (\sphinxstyleliteralemphasis{\sphinxupquote{None}}\sphinxstyleliteralemphasis{\sphinxupquote{ or }}\sphinxstyleliteralemphasis{\sphinxupquote{float}}) \textendash{} Sample Area

\item {} 
\sphinxAtStartPar
\sphinxstyleliteralstrong{\sphinxupquote{samp\_L}} (\sphinxstyleliteralemphasis{\sphinxupquote{None}}\sphinxstyleliteralemphasis{\sphinxupquote{ or }}\sphinxstyleliteralemphasis{\sphinxupquote{float}}) \textendash{} Sample Length

\item {} 
\sphinxAtStartPar
\sphinxstyleliteralstrong{\sphinxupquote{progress\_bar}} (\sphinxstyleliteralemphasis{\sphinxupquote{bool}}) \textendash{} Show/Hide progress bar

\end{itemize}

\sphinxlineitem{Returns}
\sphinxAtStartPar
full stress\sphinxhyphen{}strain information

\sphinxlineitem{Return type}
\sphinxAtStartPar
pandas.DataFrame

\sphinxlineitem{Example}
\begin{sphinxVerbatim}[commandchars=\\\{\}]
\PYG{g+gp}{\PYGZgt{}\PYGZgt{}\PYGZgt{} }\PYG{k+kn}{import} \PYG{n+nn}{openfdem} \PYG{k}{as} \PYG{n+nn}{fdem}
\PYG{g+gp}{\PYGZgt{}\PYGZgt{}\PYGZgt{} }\PYG{n}{data} \PYG{o}{=} \PYG{n}{fdem}\PYG{o}{.}\PYG{n}{Model}\PYG{p}{(}\PYG{l+s+s2}{\PYGZdq{}}\PYG{l+s+s2}{../example\PYGZus{}outputs/Irazu\PYGZus{}UCS}\PYG{l+s+s2}{\PYGZdq{}}\PYG{p}{)}
\PYG{g+go}{\PYGZsh{} Minimal Arguments}
\PYG{g+gp}{\PYGZgt{}\PYGZgt{}\PYGZgt{} }\PYG{n}{df\PYGZus{}wo\PYGZus{}SG} \PYG{o}{=} \PYG{n}{data}\PYG{o}{.}\PYG{n}{complete\PYGZus{}UCS\PYGZus{}stress\PYGZus{}strain}\PYG{p}{(}\PYG{p}{)}
\PYG{g+go}{Columns:}
\PYG{g+go}{    Name: Platen Stress, dtype=float64, nullable: False}
\PYG{g+go}{    Name: Platen Strain, dtype=float64, nullable: False}
\PYG{g+go}{Script Identifying Platen}
\PYG{g+go}{    Platen Material ID found as [1]}
\PYG{g+go}{    Predefined loading Axis [1] is Y\PYGZhy{}direction}
\PYG{g+go}{Values used in calculations are}
\PYG{g+go}{    Area    52.00}
\PYG{g+go}{    Length  108.00}
\PYG{g+go}{Progress: |//////////////////////////////////////////////////| 100.0\PYGZpc{} Complete}
\PYG{g+go}{\PYGZsh{} full stress\PYGZhy{}strain without SG}
\PYG{g+gp}{\PYGZgt{}\PYGZgt{}\PYGZgt{} }\PYG{n}{df\PYGZus{}Def\PYGZus{}SG} \PYG{o}{=} \PYG{n}{data}\PYG{o}{.}\PYG{n}{complete\PYGZus{}UCS\PYGZus{}stress\PYGZus{}strain}\PYG{p}{(}\PYG{k+kc}{None}\PYG{p}{,} \PYG{k+kc}{True}\PYG{p}{)}
\PYG{g+go}{Columns:}
\PYG{g+go}{    Name: Platen Stress, dtype=float64, nullable: False}
\PYG{g+go}{    Name: Platen Strain, dtype=float64, nullable: False}
\PYG{g+go}{    Name: Gauge Displacement X, dtype=float64, nullable: False}
\PYG{g+go}{    Name: Gauge Displacement Y, dtype=float64, nullable: False}
\PYG{g+go}{Script Identifying Platen}
\PYG{g+go}{    Platen Material ID found as [1]}
\PYG{g+go}{        Predefined loading Axis [1] is Y\PYGZhy{}direction}
\PYG{g+go}{    Values used in calculations are}
\PYG{g+go}{        Area        52.00}
\PYG{g+go}{        Length      108.00}
\PYG{g+go}{        Dimensions of SG are 27.0 x 13.0}
\PYG{g+go}{        Vertical Gauges}
\PYG{g+go}{            extends between [[6.5, \PYGZhy{}13.5, 0.0], [\PYGZhy{}6.5, \PYGZhy{}13.5, 0.0], [6.5, 13.5, 0.0], [\PYGZhy{}6.5, 13.5, 0.0]]}
\PYG{g+go}{            cover cells ID [2744, 1377, 3466, 3789]}
\PYG{g+go}{        Horizontal Gauges}
\PYG{g+go}{            extends between [[13.5, 6.5, 0.0], [13.5, \PYGZhy{}6.5, 0.0], [\PYGZhy{}13.5, 6.5, 0.0], [\PYGZhy{}13.5, \PYGZhy{}6.5, 0.0]]}
\PYG{g+go}{            cover cells ID [2089, 1582, 2210, 1504]}
\PYG{g+go}{    Progress: |//////////////////////////////////////////////////| 100.0\PYGZpc{} Complete}
\PYG{g+go}{\PYGZsh{} full stress\PYGZhy{}strain with SG and user\PYGZhy{}defined dimensions}
\PYG{g+gp}{\PYGZgt{}\PYGZgt{}\PYGZgt{} }\PYG{n}{df\PYGZus{}userdf\PYGZus{}SG} \PYG{o}{=} \PYG{n}{data}\PYG{o}{.}\PYG{n}{complete\PYGZus{}UCS\PYGZus{}stress\PYGZus{}strain}\PYG{p}{(}\PYG{k+kc}{None}\PYG{p}{,} \PYG{k+kc}{True}\PYG{p}{,} \PYG{n}{gauge\PYGZus{}width}\PYG{o}{=}\PYG{l+m+mi}{10}\PYG{p}{,} \PYG{n}{gauge\PYGZus{}length}\PYG{o}{=}\PYG{l+m+mi}{10}\PYG{p}{)}
\PYG{g+go}{Columns:}
\PYG{g+go}{    Name: Platen Stress, dtype=float64, nullable: False}
\PYG{g+go}{    Name: Platen Strain, dtype=float64, nullable: False}
\PYG{g+go}{    Name: Gauge Displacement X, dtype=float64, nullable: False}
\PYG{g+go}{    Name: Gauge Displacement Y, dtype=float64, nullable: False}
\PYG{g+go}{Script Identifying Platen}
\PYG{g+go}{    Platen Material ID found as [1]}
\PYG{g+go}{    Predefined loading Axis [1] is Y\PYGZhy{}direction}
\PYG{g+go}{Values used in calculations are}
\PYG{g+go}{    Area    52.00}
\PYG{g+go}{    Length  108.00}
\PYG{g+go}{    Dimensions of SG are 10 x 10}
\PYG{g+go}{    Vertical Gauges}
\PYG{g+go}{        extends between [[5.0, \PYGZhy{}5.0, 0.0], [\PYGZhy{}5.0, \PYGZhy{}5.0, 0.0], [5.0, 5.0, 0.0], [\PYGZhy{}5.0, 5.0, 0.0]]}
\PYG{g+go}{        cover cells ID [1186, 1397, 1669, 1148]}
\PYG{g+go}{    Horizontal Gauges}
\PYG{g+go}{        extends between [[5.0, 5.0, 0.0], [5.0, \PYGZhy{}5.0, 0.0], [\PYGZhy{}5.0, 5.0, 0.0], [\PYGZhy{}5.0, \PYGZhy{}5.0, 0.0]]}
\PYG{g+go}{        cover cells ID [1669, 1186, 1148, 1397]}
\PYG{g+go}{Progress: |//////////////////////////////////////////////////| 100.0\PYGZpc{} Complete}
\end{sphinxVerbatim}

\end{description}\end{quote}

\end{fulllineitems}

\index{convert\_to\_xyz\_array() (openfdem.openfdem.Model method)@\spxentry{convert\_to\_xyz\_array()}\spxextra{openfdem.openfdem.Model method}}

\begin{fulllineitems}
\phantomsection\label{\detokenize{openfdem:openfdem.openfdem.Model.convert_to_xyz_array}}
\pysigstartsignatures
\pysiglinewithargsret{\sphinxbfcode{\sphinxupquote{convert\_to\_xyz\_array}}}{\emph{\DUrole{n}{node\_df}}}{}
\pysigstopsignatures
\sphinxAtStartPar
Convert extracted node information into summation based on X, Y and Z
\begin{quote}\begin{description}
\sphinxlineitem{Parameters}
\sphinxAtStartPar
\sphinxstyleliteralstrong{\sphinxupquote{node\_df}} (\sphinxstyleliteralemphasis{\sphinxupquote{pandas.DataFrame}}) \textendash{} Extracted node information

\sphinxlineitem{Returns}
\sphinxAtStartPar
A DataFrame with summations along X, Y, Z axis. Column names are {[}“sum\_X”, “sum\_Y”, “sum\_Z”{]}

\sphinxlineitem{Return type}
\sphinxAtStartPar
pandas.DataFrame

\sphinxlineitem{Example}
\begin{sphinxVerbatim}[commandchars=\\\{\}]
\PYG{g+gp}{\PYGZgt{}\PYGZgt{}\PYGZgt{} }\PYG{k+kn}{import} \PYG{n+nn}{openfdem} \PYG{k}{as} \PYG{n+nn}{fdem}
\PYG{g+gp}{\PYGZgt{}\PYGZgt{}\PYGZgt{} }\PYG{n}{data} \PYG{o}{=} \PYG{n}{fdem}\PYG{o}{.}\PYG{n}{Model}\PYG{p}{(}\PYG{l+s+s2}{\PYGZdq{}}\PYG{l+s+s2}{../example\PYGZus{}outputs/Irazu\PYGZus{}3D\PYGZus{}UCS}\PYG{l+s+s2}{\PYGZdq{}}\PYG{p}{)}
\PYG{g+gp}{\PYGZgt{}\PYGZgt{}\PYGZgt{} }\PYG{c+c1}{\PYGZsh{} \PYGZsh{} Extract all cells that meet the criteria and split to nodewise data for each time step.}
\PYG{g+gp}{\PYGZgt{}\PYGZgt{}\PYGZgt{} }\PYG{c+c1}{\PYGZsh{} In this case \PYGZdq{}BOUNDARY CONDITION\PYGZdq{} is set to \PYGZdq{}1\PYGZdq{} for the threshold with the \PYGZdq{}FORCE\PYGZdq{} being extracted at each node.}
\PYG{g+gp}{\PYGZgt{}\PYGZgt{}\PYGZgt{} }\PYG{n}{df} \PYG{o}{=} \PYG{n}{data}\PYG{o}{.}\PYG{n}{extract\PYGZus{}threshold\PYGZus{}info}\PYG{p}{(}\PYG{n}{thres\PYGZus{}id}\PYG{o}{=}\PYG{l+m+mi}{1}\PYG{p}{,} \PYG{n}{thres\PYGZus{}array}\PYG{o}{=}\PYG{l+s+s1}{\PYGZsq{}}\PYG{l+s+s1}{boundary}\PYG{l+s+s1}{\PYGZsq{}}\PYG{p}{,} \PYG{n}{arrays\PYGZus{}needed}\PYG{o}{=}\PYG{p}{[}\PYG{l+s+s1}{\PYGZsq{}}\PYG{l+s+s1}{platen\PYGZus{}force}\PYG{l+s+s1}{\PYGZsq{}}\PYG{p}{]}\PYG{p}{)}
\PYG{g+gp}{\PYGZgt{}\PYGZgt{}\PYGZgt{} }\PYG{c+c1}{\PYGZsh{} Sum the X,Y,Z of all nodes for each time step.}
\PYG{g+gp}{\PYGZgt{}\PYGZgt{}\PYGZgt{} }\PYG{n}{df\PYGZus{}sum} \PYG{o}{=} \PYG{n}{data}\PYG{o}{.}\PYG{n}{convert\PYGZus{}to\PYGZus{}xyz\PYGZus{}array}\PYG{p}{(}\PYG{n}{df}\PYG{p}{)}
\PYG{g+gp}{\PYGZgt{}\PYGZgt{}\PYGZgt{} }\PYG{n+nb}{print}\PYG{p}{(}\PYG{n}{df\PYGZus{}sum}\PYG{p}{)}
\PYG{g+go}{               sum\PYGZus{}X         sum\PYGZus{}Y         sum\PYGZus{}Z}
\PYG{g+go}{    0   0.000000e+00  0.000000e+00  0.000000e+00}
\PYG{g+go}{    1  \PYGZhy{}1.224291e+05 \PYGZhy{}2.348118e+09  4.645789e+04}
\PYG{g+go}{    2  \PYGZhy{}8.768720e+04 \PYGZhy{}4.663436e+09  7.953211e+03}
\PYG{g+go}{    3  \PYGZhy{}5.580583e+04 \PYGZhy{}6.948494e+09 \PYGZhy{}1.039933e+04}
\PYG{g+go}{    4  \PYGZhy{}1.602602e+05 \PYGZhy{}9.240063e+09  1.065935e+04}
\PYG{g+go}{    5  \PYGZhy{}1.588623e+05 \PYGZhy{}1.152608e+10  4.616695e+04}
\PYG{g+go}{    ...}
\end{sphinxVerbatim}

\end{description}\end{quote}

\end{fulllineitems}

\index{crack\_failure\_mode() (openfdem.openfdem.Model method)@\spxentry{crack\_failure\_mode()}\spxextra{openfdem.openfdem.Model method}}

\begin{fulllineitems}
\phantomsection\label{\detokenize{openfdem:openfdem.openfdem.Model.crack_failure_mode}}
\pysigstartsignatures
\pysiglinewithargsret{\sphinxbfcode{\sphinxupquote{crack\_failure\_mode}}}{\emph{\DUrole{n}{remove\_boundary}\DUrole{o}{=}\DUrole{default_value}{True}}, \emph{\DUrole{n}{progress\_bar}\DUrole{o}{=}\DUrole{default_value}{True}}}{}
\pysigstopsignatures
\sphinxAtStartPar
Get all failure modes in every timestep. Boundaries Optional.
This will create a dataframe that has the failure mode for every crack. The time the crack first appears would be the time it inititated.
\begin{quote}\begin{description}
\sphinxlineitem{Parameters}\begin{itemize}
\item {} 
\sphinxAtStartPar
\sphinxstyleliteralstrong{\sphinxupquote{remove\_boundary}} (\sphinxstyleliteralemphasis{\sphinxupquote{bool}}) \textendash{} Optional. Keep or remove boundaries in DataFrame

\item {} 
\sphinxAtStartPar
\sphinxstyleliteralstrong{\sphinxupquote{progress\_bar}} (\sphinxstyleliteralemphasis{\sphinxupquote{bool}}) \textendash{} Show/Hide progress bar

\end{itemize}

\sphinxlineitem{Returns}
\sphinxAtStartPar
A DataFrame containing the failure mode for each crack at every time step.

\sphinxlineitem{Return type}
\sphinxAtStartPar
pd.DataFrame

\sphinxlineitem{Example}
\end{description}\end{quote}

\begin{sphinxVerbatim}[commandchars=\\\{\}]
\PYG{g+gp}{\PYGZgt{}\PYGZgt{}\PYGZgt{} }\PYG{k+kn}{import} \PYG{n+nn}{openfdem} \PYG{k}{as} \PYG{n+nn}{fdem}
\PYG{g+gp}{\PYGZgt{}\PYGZgt{}\PYGZgt{} }\PYG{n}{data} \PYG{o}{=} \PYG{n}{fdem}\PYG{o}{.}\PYG{n}{Model}\PYG{p}{(}\PYG{l+s+s2}{\PYGZdq{}}\PYG{l+s+s2}{../example\PYGZus{}outputs/Irazu\PYGZus{}UCS}\PYG{l+s+s2}{\PYGZdq{}}\PYG{p}{)}
\PYG{g+gp}{\PYGZgt{}\PYGZgt{}\PYGZgt{} }\PYG{n}{df\PYGZus{}cracks} \PYG{o}{=} \PYG{n}{data}\PYG{o}{.}\PYG{n}{crack\PYGZus{}failure\PYGZus{}mode}\PYG{p}{(}\PYG{p}{)}
\end{sphinxVerbatim}

\end{fulllineitems}

\index{crack\_failure\_mode\_clustering() (openfdem.openfdem.Model method)@\spxentry{crack\_failure\_mode\_clustering()}\spxextra{openfdem.openfdem.Model method}}

\begin{fulllineitems}
\phantomsection\label{\detokenize{openfdem:openfdem.openfdem.Model.crack_failure_mode_clustering}}
\pysigstartsignatures
\pysiglinewithargsret{\sphinxbfcode{\sphinxupquote{crack\_failure\_mode\_clustering}}}{\emph{\DUrole{n}{crack\_LUT}\DUrole{o}{=}\DUrole{default_value}{None}}, \emph{\DUrole{n}{crack\_LUT\_name}\DUrole{o}{=}\DUrole{default_value}{None}}, \emph{\DUrole{n}{remove\_boundary}\DUrole{o}{=}\DUrole{default_value}{True}}, \emph{\DUrole{n}{progress\_bar}\DUrole{o}{=}\DUrole{default_value}{True}}}{}
\pysigstopsignatures
\sphinxAtStartPar
Cluster the crack modes based on their failure moed values. By default, it uses the Irazu convention which is 1=pure tensile; 1\sphinxhyphen{}1.5= tensile dominant; 1.5\sphinxhyphen{}2= shear dominant, 2= pure shear; 3= mixed mode. The crack LUT can also be user defined to get a specific range within the dataset.
\begin{quote}\begin{description}
\sphinxlineitem{Parameters}\begin{itemize}
\item {} 
\sphinxAtStartPar
\sphinxstyleliteralstrong{\sphinxupquote{crack\_LUT}} (\sphinxstyleliteralemphasis{\sphinxupquote{list}}\sphinxstyleliteralemphasis{\sphinxupquote{{[}}}\sphinxstyleliteralemphasis{\sphinxupquote{float}}\sphinxstyleliteralemphasis{\sphinxupquote{{]}}}) \textendash{} LUT range for the crack failure (float)

\item {} 
\sphinxAtStartPar
\sphinxstyleliteralstrong{\sphinxupquote{crack\_LUT\_name}} (\sphinxstyleliteralemphasis{\sphinxupquote{list}}\sphinxstyleliteralemphasis{\sphinxupquote{{[}}}\sphinxstyleliteralemphasis{\sphinxupquote{str}}\sphinxstyleliteralemphasis{\sphinxupquote{{]}}}) \textendash{} LUT range for the crack failure (name)

\item {} 
\sphinxAtStartPar
\sphinxstyleliteralstrong{\sphinxupquote{remove\_boundary}} (\sphinxstyleliteralemphasis{\sphinxupquote{bool}}) \textendash{} Optional. Keep or remove boundaries in DataFrame

\item {} 
\sphinxAtStartPar
\sphinxstyleliteralstrong{\sphinxupquote{progress\_bar}} (\sphinxstyleliteralemphasis{\sphinxupquote{bool}}) \textendash{} Show/Hide progress bar

\end{itemize}

\sphinxlineitem{Returns}
\sphinxAtStartPar
A DataFrame containing the total number of cracks in every time step.

\sphinxlineitem{Return type}
\sphinxAtStartPar
pd.DataFrame

\sphinxlineitem{Example}
\end{description}\end{quote}

\begin{sphinxVerbatim}[commandchars=\\\{\}]
\PYG{g+gp}{\PYGZgt{}\PYGZgt{}\PYGZgt{} }\PYG{k+kn}{import} \PYG{n+nn}{openfdem} \PYG{k}{as} \PYG{n+nn}{fdem}
\PYG{g+gp}{\PYGZgt{}\PYGZgt{}\PYGZgt{} }\PYG{n}{data} \PYG{o}{=} \PYG{n}{fdem}\PYG{o}{.}\PYG{n}{Model}\PYG{p}{(}\PYG{l+s+s2}{\PYGZdq{}}\PYG{l+s+s2}{../example\PYGZus{}outputs/Irazu\PYGZus{}UCS}\PYG{l+s+s2}{\PYGZdq{}}\PYG{p}{)}
\PYG{g+gp}{\PYGZgt{}\PYGZgt{}\PYGZgt{} }\PYG{n}{df\PYGZus{}crack\PYGZus{}default\PYGZus{}clustering} \PYG{o}{=} \PYG{n}{data}\PYG{o}{.}\PYG{n}{crack\PYGZus{}failure\PYGZus{}mode\PYGZus{}clustering}\PYG{p}{(}\PYG{p}{)}
\PYG{g+gp}{\PYGZgt{}\PYGZgt{}\PYGZgt{} }\PYG{n}{df\PYGZus{}crack\PYGZus{}userdefined\PYGZus{}clustering} \PYG{o}{=} \PYG{n}{data}\PYG{o}{.}\PYG{n}{crack\PYGZus{}failure\PYGZus{}mode\PYGZus{}clustering}\PYG{p}{(}\PYG{n}{crack\PYGZus{}LUT}\PYG{o}{=}\PYG{p}{[}\PYG{l+m+mi}{1}\PYG{p}{,}\PYG{l+m+mi}{2}\PYG{p}{]}\PYG{p}{,} \PYG{n}{crack\PYGZus{}LUT\PYGZus{}name}\PYG{o}{=}\PYG{p}{[}\PYG{l+s+s1}{\PYGZsq{}}\PYG{l+s+s1}{tensile\PYGZus{}and\PYGZus{}shear}\PYG{l+s+s1}{\PYGZsq{}}\PYG{p}{]}\PYG{p}{)}
\end{sphinxVerbatim}

\end{fulllineitems}

\index{crack\_number() (openfdem.openfdem.Model method)@\spxentry{crack\_number()}\spxextra{openfdem.openfdem.Model method}}

\begin{fulllineitems}
\phantomsection\label{\detokenize{openfdem:openfdem.openfdem.Model.crack_number}}
\pysigstartsignatures
\pysiglinewithargsret{\sphinxbfcode{\sphinxupquote{crack\_number}}}{\emph{\DUrole{n}{progress\_bar}\DUrole{o}{=}\DUrole{default_value}{True}}}{}
\pysigstopsignatures
\sphinxAtStartPar
Get the total number of cracks at every timestep. Excluding boundaries.
\begin{quote}\begin{description}
\sphinxlineitem{Parameters}
\sphinxAtStartPar
\sphinxstyleliteralstrong{\sphinxupquote{progress\_bar}} (\sphinxstyleliteralemphasis{\sphinxupquote{bool}}) \textendash{} Show/Hide progress bar

\sphinxlineitem{Returns}
\sphinxAtStartPar
A DataFrame containing the total number of cracks at every time step.

\sphinxlineitem{Return type}
\sphinxAtStartPar
pd.DataFrame

\sphinxlineitem{Example}
\end{description}\end{quote}

\begin{sphinxVerbatim}[commandchars=\\\{\}]
\PYG{g+gp}{\PYGZgt{}\PYGZgt{}\PYGZgt{} }\PYG{k+kn}{import} \PYG{n+nn}{openfdem} \PYG{k}{as} \PYG{n+nn}{fdem}
\PYG{g+gp}{\PYGZgt{}\PYGZgt{}\PYGZgt{} }\PYG{n}{data} \PYG{o}{=} \PYG{n}{fdem}\PYG{o}{.}\PYG{n}{Model}\PYG{p}{(}\PYG{l+s+s2}{\PYGZdq{}}\PYG{l+s+s2}{../example\PYGZus{}outputs/Irazu\PYGZus{}UCS}\PYG{l+s+s2}{\PYGZdq{}}\PYG{p}{)}
\PYG{g+gp}{\PYGZgt{}\PYGZgt{}\PYGZgt{} }\PYG{n}{crack\PYGZus{}number\PYGZus{}total} \PYG{o}{=} \PYG{n}{data}\PYG{o}{.}\PYG{n}{crack\PYGZus{}number}\PYG{p}{(}\PYG{p}{)}
\PYG{g+gp}{\PYGZgt{}\PYGZgt{}\PYGZgt{} }\PYG{n+nb}{max}\PYG{p}{(}\PYG{n}{crack\PYGZus{}number\PYGZus{}total}\PYG{p}{[}\PYG{l+s+s1}{\PYGZsq{}}\PYG{l+s+s1}{crack number}\PYG{l+s+s1}{\PYGZsq{}}\PYG{p}{]}\PYG{p}{)}
\PYG{g+go}{222.0}
\PYG{g+gp}{\PYGZgt{}\PYGZgt{}\PYGZgt{} }\PYG{n}{crack\PYGZus{}number\PYGZus{}total}
\PYG{g+go}{            crack number}
\PYG{g+go}{0            0.0}
\PYG{g+go}{1            0.0}
\PYG{g+go}{2            0.0}
\PYG{g+go}{3            0.0}
\PYG{g+go}{4            0.0}
\PYG{g+go}{5            0.0}
\PYG{g+go}{6            0.0}
\PYG{g+go}{7            0.0}
\PYG{g+go}{8            0.0}
\PYG{g+go}{9          117.0}
\PYG{g+go}{10         212.0}
\PYG{g+go}{11         219.0}
\PYG{g+go}{12         221.0}
\PYG{g+go}{13         221.0}
\PYG{g+go}{14         221.0}
\PYG{g+go}{15         221.0}
\PYG{g+go}{16         221.0}
\PYG{g+go}{17         221.0}
\PYG{g+go}{18         222.0}
\PYG{g+go}{19         222.0}
\PYG{g+go}{20         222.0}
\PYG{g+go}{21         222.0}
\PYG{g+go}{22         222.0}
\PYG{g+go}{23         222.0}
\PYG{g+go}{24         222.0}
\PYG{g+go}{25         222.0}
\PYG{g+go}{26         222.0}
\PYG{g+go}{27         222.0}
\PYG{g+go}{28         222.0}
\PYG{g+go}{29         222.0}
\PYG{g+go}{30         222.0}
\end{sphinxVerbatim}

\end{fulllineitems}

\index{direct\_shear\_calculation() (openfdem.openfdem.Model method)@\spxentry{direct\_shear\_calculation()}\spxextra{openfdem.openfdem.Model method}}

\begin{fulllineitems}
\phantomsection\label{\detokenize{openfdem:openfdem.openfdem.Model.direct_shear_calculation}}
\pysigstartsignatures
\pysiglinewithargsret{\sphinxbfcode{\sphinxupquote{direct\_shear\_calculation}}}{\emph{\DUrole{n}{platen\_id}}, \emph{\DUrole{n}{array}}, \emph{\DUrole{n}{progress\_bar}\DUrole{o}{=}\DUrole{default_value}{True}}}{}
\pysigstopsignatures
\sphinxAtStartPar
Analyse direcr shear simulation built with a rigid platen on the outside.
\begin{quote}\begin{description}
\sphinxlineitem{Parameters}\begin{itemize}
\item {} 
\sphinxAtStartPar
\sphinxstyleliteralstrong{\sphinxupquote{platen\_id}} (\sphinxstyleliteralemphasis{\sphinxupquote{int}}) \textendash{} Material id of the platen

\item {} 
\sphinxAtStartPar
\sphinxstyleliteralstrong{\sphinxupquote{array}} (\sphinxstyleliteralemphasis{\sphinxupquote{str}}) \textendash{} the name of the array to be extracted

\item {} 
\sphinxAtStartPar
\sphinxstyleliteralstrong{\sphinxupquote{progress\_bar}} (\sphinxstyleliteralemphasis{\sphinxupquote{bool}}) \textendash{} Show/Hide progress bar

\end{itemize}

\sphinxlineitem{Returns}
\sphinxAtStartPar
DataFrame containing the absolute value of the array for each identified corner. Absolute sum of the extracted array split in Top/Bottom ane Left/Rigth sub\sphinxhyphen{}set into Top/Bottom.

\sphinxlineitem{Return type}
\sphinxAtStartPar
pandas.DataFrame

\sphinxlineitem{Example}
\end{description}\end{quote}

\begin{sphinxVerbatim}[commandchars=\\\{\}]
\PYG{g+gp}{\PYGZgt{}\PYGZgt{}\PYGZgt{} }\PYG{k+kn}{import} \PYG{n+nn}{openfdem} \PYG{k}{as} \PYG{n+nn}{fdem}
\PYG{g+gp}{\PYGZgt{}\PYGZgt{}\PYGZgt{} }\PYG{n}{data} \PYG{o}{=} \PYG{n}{fdem}\PYG{o}{.}\PYG{n}{Model}\PYG{p}{(}\PYG{l+s+s2}{\PYGZdq{}}\PYG{l+s+s2}{/external/2D\PYGZus{}shear\PYGZus{}4mm\PYGZus{}profile\PYGZus{}normal\PYGZus{}load\PYGZus{}test}\PYG{l+s+s2}{\PYGZdq{}}\PYG{p}{)}
\PYG{g+gp}{\PYGZgt{}\PYGZgt{}\PYGZgt{} }\PYG{n}{df} \PYG{o}{=} \PYG{n}{data}\PYG{o}{.}\PYG{n}{direct\PYGZus{}shear\PYGZus{}calculation}\PYG{p}{(}\PYG{n}{platen\PYGZus{}id}\PYG{o}{=}\PYG{l+m+mi}{1}\PYG{p}{,} \PYG{n}{array}\PYG{o}{=}\PYG{l+s+s1}{\PYGZsq{}}\PYG{l+s+s1}{platen\PYGZus{}force}\PYG{l+s+s1}{\PYGZsq{}}\PYG{p}{,} \PYG{n}{progress\PYGZus{}bar}\PYG{o}{=}\PYG{k+kc}{True}\PYG{p}{)}
\PYG{g+go}{User Defined Platen ID}
\PYG{g+go}{    Platen Material ID found as 1}
\PYG{g+go}{No. of points}
\PYG{g+go}{    Left        158}
\PYG{g+go}{    Left\PYGZus{}Top    78}
\PYG{g+go}{    Left\PYGZus{}Bottom 80}
\PYG{g+go}{    Right       158}
\PYG{g+go}{    Right\PYGZus{}Top   76}
\PYG{g+go}{    Right\PYGZus{}Bottom        82}
\PYG{g+go}{    Top 35}
\PYG{g+go}{    Bottom      38}
\PYG{g+gp}{\PYGZgt{}\PYGZgt{}\PYGZgt{} }\PYG{k+kn}{import} \PYG{n+nn}{matplotlib}\PYG{n+nn}{.}\PYG{n+nn}{pyplot} \PYG{k}{as} \PYG{n+nn}{plt}
\PYG{g+gp}{\PYGZgt{}\PYGZgt{}\PYGZgt{} }\PYG{n}{plt}\PYG{o}{.}\PYG{n}{plot}\PYG{p}{(}\PYG{n}{df}\PYG{p}{[}\PYG{l+s+s1}{\PYGZsq{}}\PYG{l+s+s1}{Left\PYGZus{}Top}\PYG{l+s+s1}{\PYGZsq{}}\PYG{p}{]}\PYG{p}{,} \PYG{n}{label}\PYG{o}{=}\PYG{l+s+s1}{\PYGZsq{}}\PYG{l+s+s1}{Left Top}\PYG{l+s+s1}{\PYGZsq{}}\PYG{p}{)}
\PYG{g+go}{[\PYGZlt{}matplotlib.lines.Line2D object at 0x7fe71f187320\PYGZgt{}]}
\PYG{g+gp}{\PYGZgt{}\PYGZgt{}\PYGZgt{} }\PYG{n}{plt}\PYG{o}{.}\PYG{n}{plot}\PYG{p}{(}\PYG{n}{df}\PYG{p}{[}\PYG{l+s+s1}{\PYGZsq{}}\PYG{l+s+s1}{Left\PYGZus{}Bottom}\PYG{l+s+s1}{\PYGZsq{}}\PYG{p}{]}\PYG{p}{,} \PYG{n}{label}\PYG{o}{=}\PYG{l+s+s1}{\PYGZsq{}}\PYG{l+s+s1}{Left Bottom}\PYG{l+s+s1}{\PYGZsq{}}\PYG{p}{)}
\PYG{g+go}{[\PYGZlt{}matplotlib.lines.Line2D object at 0x7f65cf8975f8\PYGZgt{}]}
\PYG{g+gp}{\PYGZgt{}\PYGZgt{}\PYGZgt{} }\PYG{n}{plt}\PYG{o}{.}\PYG{n}{plot}\PYG{p}{(}\PYG{n}{df}\PYG{p}{[}\PYG{l+s+s1}{\PYGZsq{}}\PYG{l+s+s1}{Left}\PYG{l+s+s1}{\PYGZsq{}}\PYG{p}{]}\PYG{p}{,} \PYG{n}{label}\PYG{o}{=}\PYG{l+s+s1}{\PYGZsq{}}\PYG{l+s+s1}{Left}\PYG{l+s+s1}{\PYGZsq{}}\PYG{p}{)}
\PYG{g+go}{[\PYGZlt{}matplotlib.lines.Line2D object at 0x7fe71f187390\PYGZgt{}]}
\PYG{g+gp}{\PYGZgt{}\PYGZgt{}\PYGZgt{} }\PYG{n}{plt}\PYG{o}{.}\PYG{n}{legend}\PYG{p}{(}\PYG{p}{)}
\PYG{g+go}{\PYGZlt{}matplotlib.legend.Legend object at 0x7fe71f187668\PYGZgt{}}
\PYG{g+gp}{\PYGZgt{}\PYGZgt{}\PYGZgt{} }\PYG{n}{plt}\PYG{o}{.}\PYG{n}{show}\PYG{p}{(}\PYG{p}{)}
\end{sphinxVerbatim}

\end{fulllineitems}

\index{draw\_rose\_diagram() (openfdem.openfdem.Model method)@\spxentry{draw\_rose\_diagram()}\spxextra{openfdem.openfdem.Model method}}

\begin{fulllineitems}
\phantomsection\label{\detokenize{openfdem:openfdem.openfdem.Model.draw_rose_diagram}}
\pysigstartsignatures
\pysiglinewithargsret{\sphinxbfcode{\sphinxupquote{draw\_rose\_diagram}}}{\emph{\DUrole{n}{t\_step}}, \emph{\DUrole{n}{rose\_data}\DUrole{o}{=}\DUrole{default_value}{None}}, \emph{\DUrole{n}{thres\_id}\DUrole{o}{=}\DUrole{default_value}{None}}, \emph{\DUrole{n}{thres\_array}\DUrole{o}{=}\DUrole{default_value}{\textquotesingle{}mineral\_type\textquotesingle{}}}, \emph{\DUrole{n}{rose\_range}\DUrole{o}{=}\DUrole{default_value}{\textquotesingle{}Length\textquotesingle{}}}}{}
\pysigstopsignatures
\sphinxAtStartPar
Draw a wind rose diagram based on the information passed.
\begin{quote}\begin{description}
\sphinxlineitem{Parameters}\begin{itemize}
\item {} 
\sphinxAtStartPar
\sphinxstyleliteralstrong{\sphinxupquote{t\_step}} (\sphinxstyleliteralemphasis{\sphinxupquote{int}}) \textendash{} Time step in model. Default 0

\item {} 
\sphinxAtStartPar
\sphinxstyleliteralstrong{\sphinxupquote{rose\_data}} (\sphinxstyleliteralemphasis{\sphinxupquote{DataFrame}}) \textendash{} User can bypass requirement and pass a DataFrame with the data. Should be 2 columns with the Angle being the 2nd. Default None

\item {} 
\sphinxAtStartPar
\sphinxstyleliteralstrong{\sphinxupquote{thres\_id}} (\sphinxstyleliteralemphasis{\sphinxupquote{int}}) \textendash{} ID of item to threshold. Default None.

\item {} 
\sphinxAtStartPar
\sphinxstyleliteralstrong{\sphinxupquote{thres\_array}} (\sphinxstyleliteralemphasis{\sphinxupquote{str}}) \textendash{} Array name of item to threshold. Default “mineral\_type”.

\item {} 
\sphinxAtStartPar
\sphinxstyleliteralstrong{\sphinxupquote{rose\_range}} (\sphinxstyleliteralemphasis{\sphinxupquote{str}}) \textendash{} Range to calculate the windrose bins. Default “length”

\end{itemize}

\sphinxlineitem{Returns}
\sphinxAtStartPar
windrose figure

\sphinxlineitem{Return type}
\sphinxAtStartPar
matplotlib.pyplot

\sphinxlineitem{Example}
\end{description}\end{quote}

\begin{sphinxVerbatim}[commandchars=\\\{\}]
\PYG{g+gp}{\PYGZgt{}\PYGZgt{}\PYGZgt{} }\PYG{k+kn}{import} \PYG{n+nn}{openfdem} \PYG{k}{as} \PYG{n+nn}{fdem}
\PYG{g+gp}{\PYGZgt{}\PYGZgt{}\PYGZgt{} }\PYG{n}{data} \PYG{o}{=} \PYG{n}{fdem}\PYG{o}{.}\PYG{n}{Model}\PYG{p}{(}\PYG{l+s+s2}{\PYGZdq{}}\PYG{l+s+s2}{../example\PYGZus{}outputs/Irazu\PYGZus{}UCS}\PYG{l+s+s2}{\PYGZdq{}}\PYG{p}{)}
\PYG{g+gp}{\PYGZgt{}\PYGZgt{}\PYGZgt{} }\PYG{n}{data}\PYG{o}{.}\PYG{n}{draw\PYGZus{}rose\PYGZus{}diagram}\PYG{p}{(}\PYG{n}{t\PYGZus{}step}\PYG{o}{=}\PYG{l+m+mi}{0}\PYG{p}{)}
\PYG{g+go}{\PYGZlt{}module \PYGZsq{}matplotlib.pyplot\PYGZsq{} from \PYGZsq{}/usr/local/lib/python3.8/dist\PYGZhy{}packages/matplotlib/pyplot.py\PYGZsq{}\PYGZgt{}}
\PYG{g+gp}{\PYGZgt{}\PYGZgt{}\PYGZgt{} }\PYG{n}{data}\PYG{o}{.}\PYG{n}{draw\PYGZus{}rose\PYGZus{}diagram}\PYG{p}{(}\PYG{n}{t\PYGZus{}step}\PYG{o}{=}\PYG{l+m+mi}{0}\PYG{p}{,} \PYG{n}{rose\PYGZus{}range}\PYG{o}{=}\PYG{l+s+s1}{\PYGZsq{}}\PYG{l+s+s1}{Length}\PYG{l+s+s1}{\PYGZsq{}}\PYG{p}{,} \PYG{n}{thres\PYGZus{}id}\PYG{o}{=}\PYG{l+m+mi}{0}\PYG{p}{,} \PYG{n}{thres\PYGZus{}array}\PYG{o}{=}\PYG{l+s+s1}{\PYGZsq{}}\PYG{l+s+s1}{boundary}\PYG{l+s+s1}{\PYGZsq{}}\PYG{p}{)}
\PYG{g+go}{\PYGZlt{}module \PYGZsq{}matplotlib.pyplot\PYGZsq{} from \PYGZsq{}/usr/local/lib/python3.8/dist\PYGZhy{}packages/matplotlib/pyplot.py\PYGZsq{}\PYGZgt{}}
\PYG{g+gp}{\PYGZgt{}\PYGZgt{}\PYGZgt{} }\PYG{c+c1}{\PYGZsh{} If you want to save the figure to a pyplot format.}
\PYG{g+gp}{\PYGZgt{}\PYGZgt{}\PYGZgt{} }\PYG{n}{figure\PYGZus{}name} \PYG{o}{=} \PYG{n}{data}\PYG{o}{.}\PYG{n}{draw\PYGZus{}rose\PYGZus{}diagram}\PYG{p}{(}\PYG{n}{t\PYGZus{}step}\PYG{o}{=}\PYG{l+m+mi}{0}\PYG{p}{,} \PYG{n}{rose\PYGZus{}range}\PYG{o}{=}\PYG{l+s+s1}{\PYGZsq{}}\PYG{l+s+s1}{Length}\PYG{l+s+s1}{\PYGZsq{}}\PYG{p}{,} \PYG{n}{thres\PYGZus{}id}\PYG{o}{=}\PYG{l+m+mi}{0}\PYG{p}{,} \PYG{n}{thres\PYGZus{}array}\PYG{o}{=}\PYG{l+s+s1}{\PYGZsq{}}\PYG{l+s+s1}{boundary}\PYG{l+s+s1}{\PYGZsq{}}\PYG{p}{)}
\end{sphinxVerbatim}

\end{fulllineitems}

\index{extract\_based\_coord() (openfdem.openfdem.Model method)@\spxentry{extract\_based\_coord()}\spxextra{openfdem.openfdem.Model method}}

\begin{fulllineitems}
\phantomsection\label{\detokenize{openfdem:openfdem.openfdem.Model.extract_based_coord}}
\pysigstartsignatures
\pysiglinewithargsret{\sphinxbfcode{\sphinxupquote{extract\_based\_coord}}}{\emph{\DUrole{n}{thres\_model}}, \emph{\DUrole{n}{coord\_xyz}}, \emph{\DUrole{n}{location}}, \emph{\DUrole{n}{include\_cells}\DUrole{o}{=}\DUrole{default_value}{False}}, \emph{\DUrole{n}{adjacent\_cells}\DUrole{o}{=}\DUrole{default_value}{False}}}{}
\pysigstopsignatures
\sphinxAtStartPar
Extract the vtkdata set based on the defined coord location in the x=0 y=1 z=2 location.
\begin{quote}\begin{description}
\sphinxlineitem{Parameters}\begin{itemize}
\item {} 
\sphinxAtStartPar
\sphinxstyleliteralstrong{\sphinxupquote{thres\_model}} (\sphinxstyleliteralemphasis{\sphinxupquote{pyvista.core.pointset.UnstructuredGrid}}) \textendash{} threshold dataset of the material id of the rock

\item {} 
\sphinxAtStartPar
\sphinxstyleliteralstrong{\sphinxupquote{coord\_xyz}} (\sphinxstyleliteralemphasis{\sphinxupquote{int}}) \textendash{} x=0 y=1 z=2

\item {} 
\sphinxAtStartPar
\sphinxstyleliteralstrong{\sphinxupquote{location}} (\sphinxstyleliteralemphasis{\sphinxupquote{float}}) \textendash{} Xmin/Xmax/Ymin/Ymax/Zmin/Zmax

\item {} 
\sphinxAtStartPar
\sphinxstyleliteralstrong{\sphinxupquote{include\_cells}} (\sphinxstyleliteralemphasis{\sphinxupquote{bool}}) \textendash{} If True, extract the cells that contain at least one of the extracted points. If False, extract the cells that contain exclusively points from the extracted points list.

\item {} 
\sphinxAtStartPar
\sphinxstyleliteralstrong{\sphinxupquote{adjacent\_cells}} (\sphinxstyleliteralemphasis{\sphinxupquote{bool}}) \textendash{} Specifies if the cells shall be returned or not

\end{itemize}

\sphinxlineitem{Returns}
\sphinxAtStartPar
Pointset of the data being filtered

\sphinxlineitem{Return type}
\sphinxAtStartPar
pyvista.core.pointset.UnstructuredGrid

\sphinxlineitem{Example}
\begin{sphinxVerbatim}[commandchars=\\\{\}]
\PYG{g+gp}{\PYGZgt{}\PYGZgt{}\PYGZgt{} }\PYG{k+kn}{import} \PYG{n+nn}{openfdem} \PYG{k}{as} \PYG{n+nn}{fdem}
\PYG{g+gp}{\PYGZgt{}\PYGZgt{}\PYGZgt{} }\PYG{n}{data} \PYG{o}{=} \PYG{n}{fdem}\PYG{o}{.}\PYG{n}{Model}\PYG{p}{(}\PYG{l+s+s2}{\PYGZdq{}}\PYG{l+s+s2}{/external/2D\PYGZus{}shear\PYGZus{}4mm\PYGZus{}profile\PYGZus{}normal\PYGZus{}load\PYGZus{}test}\PYG{l+s+s2}{\PYGZdq{}}\PYG{p}{)}
\PYG{g+gp}{\PYGZgt{}\PYGZgt{}\PYGZgt{} }\PYG{n}{data}\PYG{o}{.}\PYG{n}{rock\PYGZus{}sample\PYGZus{}dimensions}\PYG{p}{(}\PYG{p}{)}
\PYG{g+go}{Script Identifying Platen}
\PYG{g+go}{    Platen Material ID found as [1]}
\PYG{g+go}{(31.713071, 30.111493, 0.0, [\PYGZhy{}0.2, 31.513071, \PYGZhy{}14.92642, 15.185073, 0.0, 0.0])}
\PYG{g+gp}{\PYGZgt{}\PYGZgt{}\PYGZgt{} }\PYG{n}{extracted\PYGZus{}left} \PYG{o}{=} \PYG{n}{data}\PYG{o}{.}\PYG{n}{extract\PYGZus{}based\PYGZus{}coord}\PYG{p}{(}\PYG{n}{data}\PYG{o}{.}\PYG{n}{rock\PYGZus{}model}\PYG{p}{,} \PYG{l+m+mi}{0}\PYG{p}{,} \PYG{n}{data}\PYG{o}{.}\PYG{n}{rock\PYGZus{}model}\PYG{o}{.}\PYG{n}{bounds}\PYG{p}{[}\PYG{l+m+mi}{0}\PYG{p}{]}\PYG{p}{)}
\end{sphinxVerbatim}

\end{description}\end{quote}

\end{fulllineitems}

\index{extract\_cell\_info() (openfdem.openfdem.Model method)@\spxentry{extract\_cell\_info()}\spxextra{openfdem.openfdem.Model method}}

\begin{fulllineitems}
\phantomsection\label{\detokenize{openfdem:openfdem.openfdem.Model.extract_cell_info}}
\pysigstartsignatures
\pysiglinewithargsret{\sphinxbfcode{\sphinxupquote{extract\_cell\_info}}}{\emph{\DUrole{n}{cell\_id}}, \emph{\DUrole{n}{arrays\_needed}}, \emph{\DUrole{n}{progress\_bar}\DUrole{o}{=}\DUrole{default_value}{True}}}{}
\pysigstopsignatures
\sphinxAtStartPar
Returns the information of the cell based on the array requested.
If the array is a point data, the array is suffixed with \_Nx where x is the node on that cell.
Also shows a quick example on how to plot the information extracted.
\begin{quote}\begin{description}
\sphinxlineitem{Parameters}\begin{itemize}
\item {} 
\sphinxAtStartPar
\sphinxstyleliteralstrong{\sphinxupquote{cell\_id}} (\sphinxstyleliteralemphasis{\sphinxupquote{int}}) \textendash{} Cell ID to extract

\item {} 
\sphinxAtStartPar
\sphinxstyleliteralstrong{\sphinxupquote{arrays\_needed}} (\sphinxstyleliteralemphasis{\sphinxupquote{list}}\sphinxstyleliteralemphasis{\sphinxupquote{{[}}}\sphinxstyleliteralemphasis{\sphinxupquote{str}}\sphinxstyleliteralemphasis{\sphinxupquote{{]}}}) \textendash{} list of array names to extract

\item {} 
\sphinxAtStartPar
\sphinxstyleliteralstrong{\sphinxupquote{progress\_bar}} (\sphinxstyleliteralemphasis{\sphinxupquote{bool}}) \textendash{} Show/Hide progress bar

\end{itemize}

\sphinxlineitem{Returns}
\sphinxAtStartPar
unpacked DataFrame

\sphinxlineitem{Return type}
\sphinxAtStartPar
pandas.DataFrame

\sphinxlineitem{Example}
\begin{sphinxVerbatim}[commandchars=\\\{\}]
\PYG{g+gp}{\PYGZgt{}\PYGZgt{}\PYGZgt{} }\PYG{k+kn}{import} \PYG{n+nn}{openfdem} \PYG{k}{as} \PYG{n+nn}{fdem}
\PYG{g+gp}{\PYGZgt{}\PYGZgt{}\PYGZgt{} }\PYG{k+kn}{import} \PYG{n+nn}{matplotlib}\PYG{n+nn}{.}\PYG{n+nn}{pyplot} \PYG{k}{as} \PYG{n+nn}{plt}
\PYG{g+gp}{\PYGZgt{}\PYGZgt{}\PYGZgt{} }\PYG{n}{data} \PYG{o}{=} \PYG{n}{fdem}\PYG{o}{.}\PYG{n}{Model}\PYG{p}{(}\PYG{l+s+s2}{\PYGZdq{}}\PYG{l+s+s2}{../example\PYGZus{}outputs/Irazu\PYGZus{}UCS}\PYG{l+s+s2}{\PYGZdq{}}\PYG{p}{)}
\PYG{g+gp}{\PYGZgt{}\PYGZgt{}\PYGZgt{} }\PYG{c+c1}{\PYGZsh{} Extract data platen\PYGZus{}force\PYGZsq{}, \PYGZsq{}mineral\PYGZus{}type\PYGZsq{} from Cell ID 1683}
\PYG{g+gp}{\PYGZgt{}\PYGZgt{}\PYGZgt{} }\PYG{n}{extraction\PYGZus{}of\PYGZus{}cellinfo} \PYG{o}{=} \PYG{n}{data}\PYG{o}{.}\PYG{n}{extract\PYGZus{}cell\PYGZus{}info}\PYG{p}{(}\PYG{l+m+mi}{1683}\PYG{p}{,} \PYG{p}{[}\PYG{l+s+s1}{\PYGZsq{}}\PYG{l+s+s1}{platen\PYGZus{}force}\PYG{l+s+s1}{\PYGZsq{}}\PYG{p}{,} \PYG{l+s+s1}{\PYGZsq{}}\PYG{l+s+s1}{mineral\PYGZus{}type}\PYG{l+s+s1}{\PYGZsq{}}\PYG{p}{]}\PYG{p}{)}
\PYG{g+go}{Columns:}
\PYG{g+go}{    Name: platen\PYGZus{}force\PYGZus{}N1, dtype=object, nullable: False}
\PYG{g+go}{    Name: platen\PYGZus{}force\PYGZus{}N2, dtype=object, nullable: False}
\PYG{g+go}{    Name: platen\PYGZus{}force\PYGZus{}N3, dtype=object, nullable: False}
\PYG{g+go}{    Name: mineral\PYGZus{}type, dtype=object, nullable: False}
\PYG{g+gp}{\PYGZgt{}\PYGZgt{}\PYGZgt{} }\PYG{c+c1}{\PYGZsh{} For noded information =\PYGZgt{} PLOTTING METHOD ONE}
\PYG{g+gp}{\PYGZgt{}\PYGZgt{}\PYGZgt{} }\PYG{n}{x}\PYG{p}{,} \PYG{n}{y} \PYG{o}{=} \PYG{p}{[}\PYG{p}{]}\PYG{p}{,} \PYG{p}{[}\PYG{p}{]}
\PYG{g+gp}{\PYGZgt{}\PYGZgt{}\PYGZgt{} }\PYG{k}{for} \PYG{n}{i}\PYG{p}{,} \PYG{n}{row} \PYG{o+ow}{in} \PYG{n}{extraction\PYGZus{}of\PYGZus{}cellinfo}\PYG{o}{.}\PYG{n}{iterrows}\PYG{p}{(}\PYG{p}{)}\PYG{p}{:}
\PYG{g+gp}{\PYGZgt{}\PYGZgt{}\PYGZgt{} }    \PYG{n}{x}\PYG{o}{.}\PYG{n}{append}\PYG{p}{(}\PYG{n}{i}\PYG{p}{)}
\PYG{g+gp}{\PYGZgt{}\PYGZgt{}\PYGZgt{} }    \PYG{n}{y}\PYG{o}{.}\PYG{n}{append}\PYG{p}{(}\PYG{n}{row}\PYG{p}{[}\PYG{l+s+s1}{\PYGZsq{}}\PYG{l+s+s1}{platen\PYGZus{}force\PYGZus{}N2}\PYG{l+s+s1}{\PYGZsq{}}\PYG{p}{]}\PYG{p}{[}\PYG{l+m+mi}{0}\PYG{p}{]}\PYG{p}{)}
\PYG{g+gp}{\PYGZgt{}\PYGZgt{}\PYGZgt{} }\PYG{n}{plt}\PYG{o}{.}\PYG{n}{plot}\PYG{p}{(}\PYG{n}{x}\PYG{p}{,} \PYG{n}{y}\PYG{p}{,} \PYG{n}{c}\PYG{o}{=}\PYG{l+s+s1}{\PYGZsq{}}\PYG{l+s+s1}{red}\PYG{l+s+s1}{\PYGZsq{}}\PYG{p}{,} \PYG{n}{label}\PYG{o}{=}\PYG{l+s+s1}{\PYGZsq{}}\PYG{l+s+s1}{platen\PYGZus{}force\PYGZus{}N2\PYGZus{}x}\PYG{l+s+s1}{\PYGZsq{}}\PYG{p}{)}
\PYG{g+go}{[\PYGZlt{}matplotlib.lines.Line2D object at 0x7f08fe98a310\PYGZgt{}]}
\PYG{g+gp}{\PYGZgt{}\PYGZgt{}\PYGZgt{} }\PYG{n}{plt}\PYG{o}{.}\PYG{n}{legend}\PYG{p}{(}\PYG{p}{)}
\PYG{g+go}{ \PYGZlt{}matplotlib.legend.Legend object at 0x7f08fe9854c0\PYGZgt{}}
\PYG{g+gp}{\PYGZgt{}\PYGZgt{}\PYGZgt{} }\PYG{n}{plt}\PYG{o}{.}\PYG{n}{show}\PYG{p}{(}\PYG{p}{)}
\PYG{g+go}{\PYGZsh{} For noded information =\PYGZgt{} PLOTTING METHOD TWO}
\PYG{g+gp}{\PYGZgt{}\PYGZgt{}\PYGZgt{} }\PYG{n}{lx} \PYG{o}{=} \PYG{n}{extraction\PYGZus{}of\PYGZus{}cellinfo}\PYG{p}{[}\PYG{l+s+s1}{\PYGZsq{}}\PYG{l+s+s1}{platen\PYGZus{}force\PYGZus{}N2}\PYG{l+s+s1}{\PYGZsq{}}\PYG{p}{]}\PYG{o}{.}\PYG{n}{to\PYGZus{}list}\PYG{p}{(}\PYG{p}{)}
\PYG{g+gp}{\PYGZgt{}\PYGZgt{}\PYGZgt{} }\PYG{n}{lx1} \PYG{o}{=} \PYG{n+nb}{list}\PYG{p}{(}\PYG{n+nb}{zip}\PYG{p}{(}\PYG{o}{*}\PYG{n}{lx}\PYG{p}{)}\PYG{p}{)}
\PYG{g+gp}{\PYGZgt{}\PYGZgt{}\PYGZgt{} }\PYG{n}{plt}\PYG{o}{.}\PYG{n}{plot}\PYG{p}{(}\PYG{n}{lx1}\PYG{p}{[}\PYG{l+m+mi}{0}\PYG{p}{]}\PYG{p}{,} \PYG{n}{label}\PYG{o}{=}\PYG{l+s+s1}{\PYGZsq{}}\PYG{l+s+s1}{platen\PYGZus{}force\PYGZus{}N2\PYGZus{}x}\PYG{l+s+s1}{\PYGZsq{}}\PYG{p}{)}
\PYG{g+go}{[\PYGZlt{}matplotlib.lines.Line2D object at 0x7f08fe859b20\PYGZgt{}]}
\PYG{g+gp}{\PYGZgt{}\PYGZgt{}\PYGZgt{} }\PYG{n}{plt}\PYG{o}{.}\PYG{n}{plot}\PYG{p}{(}\PYG{n}{lx1}\PYG{p}{[}\PYG{l+m+mi}{1}\PYG{p}{]}\PYG{p}{,} \PYG{n}{label}\PYG{o}{=}\PYG{l+s+s1}{\PYGZsq{}}\PYG{l+s+s1}{platen\PYGZus{}force\PYGZus{}N2\PYGZus{}y}\PYG{l+s+s1}{\PYGZsq{}}\PYG{p}{)}
\PYG{g+go}{[\PYGZlt{}matplotlib.lines.Line2D object at 0x7f08fe859e50\PYGZgt{}]}
\PYG{g+gp}{\PYGZgt{}\PYGZgt{}\PYGZgt{} }\PYG{n}{plt}\PYG{o}{.}\PYG{n}{plot}\PYG{p}{(}\PYG{n}{lx1}\PYG{p}{[}\PYG{l+m+mi}{2}\PYG{p}{]}\PYG{p}{,} \PYG{n}{label}\PYG{o}{=}\PYG{l+s+s1}{\PYGZsq{}}\PYG{l+s+s1}{platen\PYGZus{}force\PYGZus{}N2\PYGZus{}z}\PYG{l+s+s1}{\PYGZsq{}}\PYG{p}{)}
\PYG{g+go}{[\PYGZlt{}matplotlib.lines.Line2D object at 0x7f08fe86a160\PYGZgt{}]}
\PYG{g+gp}{\PYGZgt{}\PYGZgt{}\PYGZgt{} }\PYG{n}{plt}\PYG{o}{.}\PYG{n}{legend}\PYG{p}{(}\PYG{p}{)}
\PYG{g+go}{\PYGZlt{}matplotlib.legend.Legend object at 0x7f08fe86a340\PYGZgt{}}
\PYG{g+gp}{\PYGZgt{}\PYGZgt{}\PYGZgt{} }\PYG{n}{plt}\PYG{o}{.}\PYG{n}{show}\PYG{p}{(}\PYG{p}{)}
\PYG{g+go}{\PYGZsh{} For non\PYGZhy{}nonded information}
\PYG{g+gp}{\PYGZgt{}\PYGZgt{}\PYGZgt{} }\PYG{n}{plt}\PYG{o}{.}\PYG{n}{plot}\PYG{p}{(}\PYG{n}{lx1}\PYG{p}{[}\PYG{l+m+mi}{0}\PYG{p}{]}\PYG{p}{,} \PYG{n}{label}\PYG{o}{=}\PYG{l+s+s1}{\PYGZsq{}}\PYG{l+s+s1}{mineral\PYGZus{}type}\PYG{l+s+s1}{\PYGZsq{}}\PYG{p}{)}
\PYG{g+go}{[\PYGZlt{}matplotlib.lines.Line2D object at 0x7f08fe7e39a0\PYGZgt{}]}
\PYG{g+gp}{\PYGZgt{}\PYGZgt{}\PYGZgt{} }\PYG{n}{plt}\PYG{o}{.}\PYG{n}{legend}\PYG{p}{(}\PYG{p}{)}
\PYG{g+go}{\PYGZlt{}matplotlib.legend.Legend object at 0x7f08fe7e39d0\PYGZgt{}}
\PYG{g+gp}{\PYGZgt{}\PYGZgt{}\PYGZgt{} }\PYG{n}{plt}\PYG{o}{.}\PYG{n}{show}\PYG{p}{(}\PYG{p}{)}
\end{sphinxVerbatim}

\end{description}\end{quote}

\end{fulllineitems}

\index{extract\_crack\_info() (openfdem.openfdem.Model method)@\spxentry{extract\_crack\_info()}\spxextra{openfdem.openfdem.Model method}}

\begin{fulllineitems}
\phantomsection\label{\detokenize{openfdem:openfdem.openfdem.Model.extract_crack_info}}
\pysigstartsignatures
\pysiglinewithargsret{\sphinxbfcode{\sphinxupquote{extract\_crack\_info}}}{\emph{\DUrole{n}{arrays\_needed}}, \emph{\DUrole{n}{progress\_bar}\DUrole{o}{=}\DUrole{default_value}{True}}}{}
\pysigstopsignatures
\sphinxAtStartPar
Returns the information based on the array requested.
\begin{quote}\begin{description}
\sphinxlineitem{Parameters}\begin{itemize}
\item {} 
\sphinxAtStartPar
\sphinxstyleliteralstrong{\sphinxupquote{arrays\_needed}} (\sphinxstyleliteralemphasis{\sphinxupquote{list}}\sphinxstyleliteralemphasis{\sphinxupquote{{[}}}\sphinxstyleliteralemphasis{\sphinxupquote{str}}\sphinxstyleliteralemphasis{\sphinxupquote{{]}}}) \textendash{} list of array names to extract

\item {} 
\sphinxAtStartPar
\sphinxstyleliteralstrong{\sphinxupquote{progress\_bar}} (\sphinxstyleliteralemphasis{\sphinxupquote{bool}}) \textendash{} Show/Hide progress bar

\end{itemize}

\sphinxlineitem{Returns}
\sphinxAtStartPar
unpacked DataFrame

\sphinxlineitem{Return type}
\sphinxAtStartPar
pandas.DataFrame

\sphinxlineitem{Example}
\begin{sphinxVerbatim}[commandchars=\\\{\}]
\PYG{g+gp}{\PYGZgt{}\PYGZgt{}\PYGZgt{} }\PYG{k+kn}{import} \PYG{n+nn}{openfdem} \PYG{k}{as} \PYG{n+nn}{fdem}
\PYG{g+gp}{\PYGZgt{}\PYGZgt{}\PYGZgt{} }\PYG{n}{data} \PYG{o}{=} \PYG{n}{fdem}\PYG{o}{.}\PYG{n}{Model}\PYG{p}{(}\PYG{l+s+s2}{\PYGZdq{}}\PYG{l+s+s2}{../example\PYGZus{}outputs/Irazu\PYGZus{}UCS}\PYG{l+s+s2}{\PYGZdq{}}\PYG{p}{)}
\PYG{g+gp}{\PYGZgt{}\PYGZgt{}\PYGZgt{} }\PYG{n}{extraction\PYGZus{}of\PYGZus{}cracks} \PYG{o}{=} \PYG{n}{data}\PYG{o}{.}\PYG{n}{extract\PYGZus{}crack\PYGZus{}info}\PYG{p}{(}\PYG{n}{arrays\PYGZus{}needed}\PYG{o}{=}\PYG{p}{[}\PYG{l+s+s1}{\PYGZsq{}}\PYG{l+s+s1}{area}\PYG{l+s+s1}{\PYGZsq{}}\PYG{p}{,} \PYG{l+s+s1}{\PYGZsq{}}\PYG{l+s+s1}{length}\PYG{l+s+s1}{\PYGZsq{}}\PYG{p}{]}\PYG{p}{)}
\PYG{g+go}{Progress: |//////////////////////////////////////////////////| 100.0\PYGZpc{} Complete}
\end{sphinxVerbatim}

\end{description}\end{quote}

\end{fulllineitems}

\index{extract\_threshold\_info() (openfdem.openfdem.Model method)@\spxentry{extract\_threshold\_info()}\spxextra{openfdem.openfdem.Model method}}

\begin{fulllineitems}
\phantomsection\label{\detokenize{openfdem:openfdem.openfdem.Model.extract_threshold_info}}
\pysigstartsignatures
\pysiglinewithargsret{\sphinxbfcode{\sphinxupquote{extract\_threshold\_info}}}{\emph{\DUrole{n}{thres\_id}}, \emph{\DUrole{n}{thres\_array}}, \emph{\DUrole{n}{arrays\_needed}}, \emph{\DUrole{n}{dataset\_to\_load}\DUrole{o}{=}\DUrole{default_value}{\textquotesingle{}basic\textquotesingle{}}}, \emph{\DUrole{n}{progress\_bar}\DUrole{o}{=}\DUrole{default_value}{True}}}{}
\pysigstopsignatures
\sphinxAtStartPar
Returns the information of the cell based on the array requested.
If the array is a point data, the array is suffixed with \_Nx where x is cell ID.
Also shows a quick example on how to plot the information extracted.
\begin{quote}\begin{description}
\sphinxlineitem{Parameters}\begin{itemize}
\item {} 
\sphinxAtStartPar
\sphinxstyleliteralstrong{\sphinxupquote{thres\_id}} (\sphinxstyleliteralemphasis{\sphinxupquote{int}}) \textendash{} Threshold ID to extract

\item {} 
\sphinxAtStartPar
\sphinxstyleliteralstrong{\sphinxupquote{thres\_array}} (\sphinxstyleliteralemphasis{\sphinxupquote{str}}) \textendash{} Array name of item to threshold.

\item {} 
\sphinxAtStartPar
\sphinxstyleliteralstrong{\sphinxupquote{arrays\_needed}} (\sphinxstyleliteralemphasis{\sphinxupquote{list}}\sphinxstyleliteralemphasis{\sphinxupquote{{[}}}\sphinxstyleliteralemphasis{\sphinxupquote{str}}\sphinxstyleliteralemphasis{\sphinxupquote{{]}}}) \textendash{} list of array names to extract

\item {} 
\sphinxAtStartPar
\sphinxstyleliteralstrong{\sphinxupquote{progress\_bar}} (\sphinxstyleliteralemphasis{\sphinxupquote{bool}}) \textendash{} Show/Hide progress bar

\end{itemize}

\sphinxlineitem{Returns}
\sphinxAtStartPar
A DataFrame or a series of DataFrames nested in a dictionary with the key being the name of the array needed

\sphinxlineitem{Return type}
\sphinxAtStartPar
pandas.DataFrame or dict{[}pandas.DataFrame{]}

\sphinxlineitem{Example}
\begin{sphinxVerbatim}[commandchars=\\\{\}]
\PYG{g+gp}{\PYGZgt{}\PYGZgt{}\PYGZgt{} }\PYG{k+kn}{import} \PYG{n+nn}{openfdem} \PYG{k}{as} \PYG{n+nn}{fdem}
\PYG{g+gp}{\PYGZgt{}\PYGZgt{}\PYGZgt{} }\PYG{k+kn}{import} \PYG{n+nn}{matplotlib}\PYG{n+nn}{.}\PYG{n+nn}{pyplot} \PYG{k}{as} \PYG{n+nn}{plt}
\PYG{g+gp}{\PYGZgt{}\PYGZgt{}\PYGZgt{} }\PYG{n}{data} \PYG{o}{=} \PYG{n}{fdem}\PYG{o}{.}\PYG{n}{Model}\PYG{p}{(}\PYG{l+s+s2}{\PYGZdq{}}\PYG{l+s+s2}{../example\PYGZus{}outputs/Irazu\PYGZus{}3D\PYGZus{}UCS}\PYG{l+s+s2}{\PYGZdq{}}\PYG{p}{)}
\PYG{g+gp}{\PYGZgt{}\PYGZgt{}\PYGZgt{} }\PYG{c+c1}{\PYGZsh{} Extract all cells that meet the criteria and split to nodewise data for each time step.}
\PYG{g+gp}{\PYGZgt{}\PYGZgt{}\PYGZgt{} }\PYG{c+c1}{\PYGZsh{} In this case \PYGZdq{}BOUNDARY CONDITION\PYGZdq{} is set to \PYGZdq{}1\PYGZdq{} for the threshold with the \PYGZdq{}FORCE\PYGZdq{} being extracted at each node.}
\PYG{g+gp}{\PYGZgt{}\PYGZgt{}\PYGZgt{} }\PYG{n}{df} \PYG{o}{=} \PYG{n}{data}\PYG{o}{.}\PYG{n}{extract\PYGZus{}threshold\PYGZus{}info}\PYG{p}{(}\PYG{n}{thres\PYGZus{}id}\PYG{o}{=}\PYG{l+m+mi}{1}\PYG{p}{,} \PYG{n}{thres\PYGZus{}array}\PYG{o}{=}\PYG{l+s+s1}{\PYGZsq{}}\PYG{l+s+s1}{boundary}\PYG{l+s+s1}{\PYGZsq{}}\PYG{p}{,} \PYG{n}{arrays\PYGZus{}needed}\PYG{o}{=}\PYG{p}{[}\PYG{l+s+s1}{\PYGZsq{}}\PYG{l+s+s1}{platen\PYGZus{}force}\PYG{l+s+s1}{\PYGZsq{}}\PYG{p}{]}\PYG{p}{)}
\PYG{g+go}{Progress: |//////////////////////////////////////////////////| 100.0\PYGZpc{} Complete}
\PYG{g+go}{54.74 seconds.}
\PYG{g+gp}{\PYGZgt{}\PYGZgt{}\PYGZgt{} }\PYG{c+c1}{\PYGZsh{} Sum the X,Y,Z of all nodes for each time step.}
\PYG{g+gp}{\PYGZgt{}\PYGZgt{}\PYGZgt{} }\PYG{n}{df\PYGZus{}sum} \PYG{o}{=} \PYG{n}{data}\PYG{o}{.}\PYG{n}{convert\PYGZus{}to\PYGZus{}xyz\PYGZus{}array}\PYG{p}{(}\PYG{n}{df}\PYG{p}{)}
\end{sphinxVerbatim}

\end{description}\end{quote}

\end{fulllineitems}

\index{find\_cell() (openfdem.openfdem.Model method)@\spxentry{find\_cell()}\spxextra{openfdem.openfdem.Model method}}

\begin{fulllineitems}
\phantomsection\label{\detokenize{openfdem:openfdem.openfdem.Model.find_cell}}
\pysigstartsignatures
\pysiglinewithargsret{\sphinxbfcode{\sphinxupquote{find\_cell}}}{\emph{\DUrole{n}{model\_point}}}{}
\pysigstopsignatures
\sphinxAtStartPar
Identify the containing cell in the model to the defined point. Will return an error if the point is not within a cell.
\begin{quote}\begin{description}
\sphinxlineitem{Parameters}
\sphinxAtStartPar
\sphinxstyleliteralstrong{\sphinxupquote{model\_point}} (\sphinxstyleliteralemphasis{\sphinxupquote{list}}\sphinxstyleliteralemphasis{\sphinxupquote{{[}}}\sphinxstyleliteralemphasis{\sphinxupquote{float}}\sphinxstyleliteralemphasis{\sphinxupquote{, }}\sphinxstyleliteralemphasis{\sphinxupquote{float}}\sphinxstyleliteralemphasis{\sphinxupquote{, }}\sphinxstyleliteralemphasis{\sphinxupquote{float}}\sphinxstyleliteralemphasis{\sphinxupquote{{]}}}) \textendash{} x,y,z of a point in the model which

\sphinxlineitem{Returns}
\sphinxAtStartPar
The cell that contains the point.

\sphinxlineitem{Return type}
\sphinxAtStartPar
int

\sphinxlineitem{Raises}
\sphinxAtStartPar
\sphinxstyleliteralstrong{\sphinxupquote{IndexError}} \textendash{} Point outside model domain.

\sphinxlineitem{Example}
\begin{sphinxVerbatim}[commandchars=\\\{\}]
\PYG{g+gp}{\PYGZgt{}\PYGZgt{}\PYGZgt{} }\PYG{k+kn}{import} \PYG{n+nn}{openfdem} \PYG{k}{as} \PYG{n+nn}{fdem}
\PYG{g+gp}{\PYGZgt{}\PYGZgt{}\PYGZgt{} }\PYG{n}{data} \PYG{o}{=} \PYG{n}{fdem}\PYG{o}{.}\PYG{n}{Model}\PYG{p}{(}\PYG{l+s+s2}{\PYGZdq{}}\PYG{l+s+s2}{../example\PYGZus{}outputs/Irazu\PYGZus{}UCS}\PYG{l+s+s2}{\PYGZdq{}}\PYG{p}{)}
\PYG{g+gp}{\PYGZgt{}\PYGZgt{}\PYGZgt{} }\PYG{n}{data}\PYG{o}{.}\PYG{n}{find\PYGZus{}cell}\PYG{p}{(}\PYG{p}{[}\PYG{l+m+mi}{0}\PYG{p}{,} \PYG{l+m+mi}{0}\PYG{p}{,} \PYG{l+m+mi}{0}\PYG{p}{]}\PYG{p}{)}
\PYG{g+go}{2167}
\PYG{g+gp}{\PYGZgt{}\PYGZgt{}\PYGZgt{} }\PYG{n}{data}\PYG{o}{.}\PYG{n}{find\PYGZus{}cell}\PYG{p}{(}\PYG{p}{[}\PYG{l+m+mi}{2000}\PYG{p}{,} \PYG{l+m+mi}{2000}\PYG{p}{,} \PYG{l+m+mi}{0}\PYG{p}{]}\PYG{p}{)}
\PYG{g+go}{IndexError: Point outside model domain.}
\PYG{g+go}{X=56.0, Y=116.0, Z=0.0}
\end{sphinxVerbatim}

\end{description}\end{quote}

\end{fulllineitems}

\index{mesh\_geometry() (openfdem.openfdem.Model method)@\spxentry{mesh\_geometry()}\spxextra{openfdem.openfdem.Model method}}

\begin{fulllineitems}
\phantomsection\label{\detokenize{openfdem:openfdem.openfdem.Model.mesh_geometry}}
\pysigstartsignatures
\pysiglinewithargsret{\sphinxbfcode{\sphinxupquote{mesh\_geometry}}}{\emph{\DUrole{n}{vertices}}}{}
\pysigstopsignatures
\sphinxAtStartPar
Returns a unique set of vertices and calculates their length and orientation.
\begin{quote}\begin{description}
\sphinxlineitem{Parameters}
\sphinxAtStartPar
\sphinxstyleliteralstrong{\sphinxupquote{vertices}} (\sphinxstyleliteralemphasis{\sphinxupquote{list}}\sphinxstyleliteralemphasis{\sphinxupquote{{[}}}\sphinxstyleliteralemphasis{\sphinxupquote{tuples}}\sphinxstyleliteralemphasis{\sphinxupquote{{]}}}) \textendash{} list of vertices in the model at a given time step

\sphinxlineitem{Returns}
\sphinxAtStartPar
DataFrame of the vertices length and orientation

\sphinxlineitem{Return type}
\sphinxAtStartPar
pandas.DataFrame

\sphinxlineitem{Example}
\end{description}\end{quote}

\begin{sphinxVerbatim}[commandchars=\\\{\}]
\PYG{g+gp}{\PYGZgt{}\PYGZgt{}\PYGZgt{} }\PYG{k+kn}{import} \PYG{n+nn}{openfdem} \PYG{k}{as} \PYG{n+nn}{fdem}
\PYG{g+gp}{\PYGZgt{}\PYGZgt{}\PYGZgt{} }\PYG{n}{data} \PYG{o}{=} \PYG{n}{fdem}\PYG{o}{.}\PYG{n}{Model}\PYG{p}{(}\PYG{l+s+s2}{\PYGZdq{}}\PYG{l+s+s2}{../example\PYGZus{}outputs/Irazu\PYGZus{}UCS}\PYG{l+s+s2}{\PYGZdq{}}\PYG{p}{)}
\PYG{g+gp}{\PYGZgt{}\PYGZgt{}\PYGZgt{} }\PYG{n}{vert} \PYG{o}{=} \PYG{n}{data}\PYG{o}{.}\PYG{n}{model\PYGZus{}vertices}\PYG{p}{(}\PYG{n}{t\PYGZus{}step}\PYG{o}{=}\PYG{l+m+mi}{0}\PYG{p}{,} \PYG{n}{thres\PYGZus{}id}\PYG{o}{=}\PYG{l+m+mi}{1}\PYG{p}{,} \PYG{n}{thres\PYGZus{}array}\PYG{o}{=}\PYG{l+s+s1}{\PYGZsq{}}\PYG{l+s+s1}{mineral\PYGZus{}type}\PYG{l+s+s1}{\PYGZsq{}}\PYG{p}{)}
\PYG{g+gp}{\PYGZgt{}\PYGZgt{}\PYGZgt{} }\PYG{n}{data}\PYG{o}{.}\PYG{n}{mesh\PYGZus{}geometry}\PYG{p}{(}\PYG{n}{vert}\PYG{p}{)}
\PYG{g+go}{       Length       Angle}
\PYG{g+go}{0    2.236068   63.434949}
\PYG{g+go}{1    2.000000    0.000000}
\PYG{g+go}{2    2.363608   59.436301}
\PYG{g+go}{3    2.000000    0.000000}
\PYG{g+go}{4    2.244731  117.123188}
\PYG{g+go}{..        ...         ...}
\PYG{g+go}{409  2.116948    0.287685}
\PYG{g+go}{410  2.000000    0.000000}
\PYG{g+go}{411  2.000000    0.000000}
\PYG{g+go}{412  1.802781   45.829911}
\PYG{g+go}{413  2.227619  116.002627}

\PYG{g+go}{[414 rows x 2 columns]}
\end{sphinxVerbatim}

\end{fulllineitems}

\index{model\_dimensions() (openfdem.openfdem.Model method)@\spxentry{model\_dimensions()}\spxextra{openfdem.openfdem.Model method}}

\begin{fulllineitems}
\phantomsection\label{\detokenize{openfdem:openfdem.openfdem.Model.model_dimensions}}
\pysigstartsignatures
\pysiglinewithargsret{\sphinxbfcode{\sphinxupquote{model\_dimensions}}}{\emph{\DUrole{n}{mat\_id}\DUrole{o}{=}\DUrole{default_value}{None}}}{}
\pysigstopsignatures
\sphinxAtStartPar
Function to get the “INITIAL” model bounds and returns the width, height, thickness
\begin{quote}\begin{description}
\sphinxlineitem{Parameters}
\sphinxAtStartPar
\sphinxstyleliteralstrong{\sphinxupquote{mat\_id}} (\sphinxstyleliteralemphasis{\sphinxupquote{int}}) \textendash{} Optional, if a threshold is specific to a material type

\sphinxlineitem{Returns}
\sphinxAtStartPar
model width, model height, model thickness

\sphinxlineitem{Type}
\sphinxAtStartPar
tuple{[}float, float, float{]}

\sphinxlineitem{Example}
\begin{sphinxVerbatim}[commandchars=\\\{\}]
\PYG{g+gp}{\PYGZgt{}\PYGZgt{}\PYGZgt{} }\PYG{k+kn}{import} \PYG{n+nn}{openfdem} \PYG{k}{as} \PYG{n+nn}{fdem}
\PYG{g+gp}{\PYGZgt{}\PYGZgt{}\PYGZgt{} }\PYG{n}{model} \PYG{o}{=} \PYG{n}{fdem}\PYG{o}{.}\PYG{n}{Model}\PYG{p}{(}\PYG{l+s+s2}{\PYGZdq{}}\PYG{l+s+s2}{../example\PYGZus{}outputs/Irazu\PYGZus{}UCS}\PYG{l+s+s2}{\PYGZdq{}}\PYG{p}{)}
\PYG{g+gp}{\PYGZgt{}\PYGZgt{}\PYGZgt{} }\PYG{c+c1}{\PYGZsh{} Returns the overall model dimensions}
\PYG{g+gp}{\PYGZgt{}\PYGZgt{}\PYGZgt{} }\PYG{n}{model}\PYG{o}{.}\PYG{n}{model\PYGZus{}dimensions}\PYG{p}{(}\PYG{p}{)}
\PYG{g+go}{(56.0, 116.0, 0.0)}
\PYG{g+gp}{\PYGZgt{}\PYGZgt{}\PYGZgt{} }\PYG{c+c1}{\PYGZsh{} Returns the model dimensions based on material id 1}
\PYG{g+gp}{\PYGZgt{}\PYGZgt{}\PYGZgt{} }\PYG{n}{model}\PYG{o}{.}\PYG{n}{model\PYGZus{}dimensions}\PYG{p}{(}\PYG{l+m+mi}{1}\PYG{p}{)}
\PYG{g+go}{(56.0, 116.0, 0.0)}
\PYG{g+gp}{\PYGZgt{}\PYGZgt{}\PYGZgt{} }\PYG{c+c1}{\PYGZsh{} Error when material is not found}
\PYG{g+gp}{\PYGZgt{}\PYGZgt{}\PYGZgt{} }\PYG{n}{model}\PYG{o}{.}\PYG{n}{model\PYGZus{}dimensions}\PYG{p}{(}\PYG{l+m+mi}{3}\PYG{p}{)}
\PYG{g+go}{IndexError: Material ID for platen out of range.}
\PYG{g+go}{Material Range 0\PYGZhy{}1}
\end{sphinxVerbatim}

\end{description}\end{quote}

\end{fulllineitems}

\index{model\_domain() (openfdem.openfdem.Model method)@\spxentry{model\_domain()}\spxextra{openfdem.openfdem.Model method}}

\begin{fulllineitems}
\phantomsection\label{\detokenize{openfdem:openfdem.openfdem.Model.model_domain}}
\pysigstartsignatures
\pysiglinewithargsret{\sphinxbfcode{\sphinxupquote{model\_domain}}}{}{}
\pysigstopsignatures\begin{description}
\sphinxlineitem{Identifies the model domain by confirming the simulation cell vertex.}
\sphinxAtStartPar
2D (3 Points \sphinxhyphen{} Triangle)
3D (4 Points \sphinxhyphen{} Tetrahedral)

\end{description}
\begin{quote}\begin{description}
\sphinxlineitem{Returns}
\sphinxAtStartPar
number of nodes to skip in analysis

\sphinxlineitem{Return type}
\sphinxAtStartPar
int

\sphinxlineitem{Raises}
\sphinxAtStartPar
\sphinxstyleliteralstrong{\sphinxupquote{Warning}} \textendash{} Simulation partially supported.

\sphinxlineitem{Example}
\begin{sphinxVerbatim}[commandchars=\\\{\}]
\PYG{g+gp}{\PYGZgt{}\PYGZgt{}\PYGZgt{} }\PYG{k+kn}{import} \PYG{n+nn}{openfdem} \PYG{k}{as} \PYG{n+nn}{fdem}
\PYG{g+gp}{\PYGZgt{}\PYGZgt{}\PYGZgt{} }\PYG{n}{model} \PYG{o}{=} \PYG{n}{fdem}\PYG{o}{.}\PYG{n}{Model}\PYG{p}{(}\PYG{l+s+s2}{\PYGZdq{}}\PYG{l+s+s2}{../example\PYGZus{}outputs/Irazu\PYGZus{}UCS}\PYG{l+s+s2}{\PYGZdq{}}\PYG{p}{)}
\PYG{g+gp}{\PYGZgt{}\PYGZgt{}\PYGZgt{} }\PYG{n}{model}\PYG{o}{.}\PYG{n}{model\PYGZus{}domain}\PYG{p}{(}\PYG{p}{)}
\PYG{g+go}{2D Simulation}
\PYG{g+go}{4}
\end{sphinxVerbatim}

\end{description}\end{quote}

\end{fulllineitems}

\index{model\_vertices() (openfdem.openfdem.Model method)@\spxentry{model\_vertices()}\spxextra{openfdem.openfdem.Model method}}

\begin{fulllineitems}
\phantomsection\label{\detokenize{openfdem:openfdem.openfdem.Model.model_vertices}}
\pysigstartsignatures
\pysiglinewithargsret{\sphinxbfcode{\sphinxupquote{model\_vertices}}}{\emph{\DUrole{n}{t\_step}\DUrole{o}{=}\DUrole{default_value}{0}}, \emph{\DUrole{n}{thres\_id}\DUrole{o}{=}\DUrole{default_value}{None}}, \emph{\DUrole{n}{thres\_array}\DUrole{o}{=}\DUrole{default_value}{\textquotesingle{}mineral\_type\textquotesingle{}}}}{}
\pysigstopsignatures
\sphinxAtStartPar
Returns a list of the vertices in the form of Point1, Point 2
\begin{quote}\begin{description}
\sphinxlineitem{Parameters}\begin{itemize}
\item {} 
\sphinxAtStartPar
\sphinxstyleliteralstrong{\sphinxupquote{t\_step}} (\sphinxstyleliteralemphasis{\sphinxupquote{int}}) \textendash{} Time step in model. Default 0

\item {} 
\sphinxAtStartPar
\sphinxstyleliteralstrong{\sphinxupquote{thres\_id}} (\sphinxstyleliteralemphasis{\sphinxupquote{int}}) \textendash{} ID of item to threshold. Default None.

\item {} 
\sphinxAtStartPar
\sphinxstyleliteralstrong{\sphinxupquote{thres\_array}} (\sphinxstyleliteralemphasis{\sphinxupquote{str}}) \textendash{} Array name of item to threshold. Default “mineral\_type”.

\end{itemize}

\sphinxlineitem{Returns}
\sphinxAtStartPar
list of the verticies in the model and/or the threshold of it.

\sphinxlineitem{Return type}
\sphinxAtStartPar
list{[}tuples{]}

\sphinxlineitem{Example}
\end{description}\end{quote}

\begin{sphinxVerbatim}[commandchars=\\\{\}]
\PYG{g+gp}{\PYGZgt{}\PYGZgt{}\PYGZgt{} }\PYG{k+kn}{import} \PYG{n+nn}{openfdem} \PYG{k}{as} \PYG{n+nn}{fdem}
\PYG{g+gp}{\PYGZgt{}\PYGZgt{}\PYGZgt{} }\PYG{n}{data} \PYG{o}{=} \PYG{n}{fdem}\PYG{o}{.}\PYG{n}{Model}\PYG{p}{(}\PYG{l+s+s2}{\PYGZdq{}}\PYG{l+s+s2}{../example\PYGZus{}outputs/Irazu\PYGZus{}UCS}\PYG{l+s+s2}{\PYGZdq{}}\PYG{p}{)}
\PYG{g+gp}{\PYGZgt{}\PYGZgt{}\PYGZgt{} }\PYG{n+nb}{len}\PYG{p}{(}\PYG{n}{data}\PYG{o}{.}\PYG{n}{model\PYGZus{}vertices}\PYG{p}{(}\PYG{n}{t\PYGZus{}step}\PYG{o}{=}\PYG{l+m+mi}{0}\PYG{p}{,} \PYG{n}{thres\PYGZus{}id}\PYG{o}{=}\PYG{l+m+mi}{0}\PYG{p}{,} \PYG{n}{thres\PYGZus{}array}\PYG{o}{=}\PYG{l+s+s1}{\PYGZsq{}}\PYG{l+s+s1}{mineral\PYGZus{}type}\PYG{l+s+s1}{\PYGZsq{}}\PYG{p}{)}\PYG{p}{)}
\PYG{g+go}{11196}
\PYG{g+gp}{\PYGZgt{}\PYGZgt{}\PYGZgt{} }\PYG{n+nb}{len}\PYG{p}{(}\PYG{n}{data}\PYG{o}{.}\PYG{n}{model\PYGZus{}vertices}\PYG{p}{(}\PYG{n}{t\PYGZus{}step}\PYG{o}{=}\PYG{l+m+mi}{0}\PYG{p}{,} \PYG{n}{thres\PYGZus{}id}\PYG{o}{=}\PYG{l+m+mi}{1}\PYG{p}{,} \PYG{n}{thres\PYGZus{}array}\PYG{o}{=}\PYG{l+s+s1}{\PYGZsq{}}\PYG{l+s+s1}{boundary}\PYG{l+s+s1}{\PYGZsq{}}\PYG{p}{)}\PYG{p}{)}
\PYG{g+go}{354}
\end{sphinxVerbatim}

\end{fulllineitems}

\index{openfdem\_att\_check() (openfdem.openfdem.Model method)@\spxentry{openfdem\_att\_check()}\spxextra{openfdem.openfdem.Model method}}

\begin{fulllineitems}
\phantomsection\label{\detokenize{openfdem:openfdem.openfdem.Model.openfdem_att_check}}
\pysigstartsignatures
\pysiglinewithargsret{\sphinxbfcode{\sphinxupquote{openfdem\_att\_check}}}{\emph{\DUrole{n}{att}}, \emph{\DUrole{n}{dict\_to\_check}\DUrole{o}{=}\DUrole{default_value}{None}}}{}
\pysigstopsignatures
\sphinxAtStartPar
Checks that the attribute is a valid choice.
\begin{quote}\begin{description}
\sphinxlineitem{Param}
\sphinxAtStartPar
Attribute

\sphinxlineitem{Type}
\sphinxAtStartPar
str

\sphinxlineitem{Returns}
\sphinxAtStartPar
Attribute

\sphinxlineitem{Return type}
\sphinxAtStartPar
str

\sphinxlineitem{Raises}
\sphinxAtStartPar
\sphinxstyleliteralstrong{\sphinxupquote{KeyError}} \textendash{} Attribute does not exist.

\sphinxlineitem{Example}
\begin{sphinxVerbatim}[commandchars=\\\{\}]
\PYG{g+gp}{\PYGZgt{}\PYGZgt{}\PYGZgt{} }\PYG{k+kn}{import} \PYG{n+nn}{openfdem} \PYG{k}{as} \PYG{n+nn}{fdem}
\PYG{g+gp}{\PYGZgt{}\PYGZgt{}\PYGZgt{} }\PYG{n}{model} \PYG{o}{=} \PYG{n}{fdem}\PYG{o}{.}\PYG{n}{Model}\PYG{p}{(}\PYG{l+s+s2}{\PYGZdq{}}\PYG{l+s+s2}{../example\PYGZus{}outputs/Irazu\PYGZus{}UCS}\PYG{l+s+s2}{\PYGZdq{}}\PYG{p}{)}
\PYG{g+gp}{\PYGZgt{}\PYGZgt{}\PYGZgt{} }\PYG{n}{model}\PYG{o}{.}\PYG{n}{openfdem\PYGZus{}att\PYGZus{}check}\PYG{p}{(}\PYG{l+s+s1}{\PYGZsq{}}\PYG{l+s+s1}{mineral\PYGZus{}type}\PYG{l+s+s1}{\PYGZsq{}}\PYG{p}{)}
\PYG{g+go}{\PYGZsq{}mineral\PYGZus{}type\PYGZsq{}}
\PYG{g+gp}{\PYGZgt{}\PYGZgt{}\PYGZgt{} }\PYG{n}{model}\PYG{o}{.}\PYG{n}{openfdem\PYGZus{}att\PYGZus{}check}\PYG{p}{(}\PYG{l+s+s1}{\PYGZsq{}}\PYG{l+s+s1}{material\PYGZus{}property}\PYG{l+s+s1}{\PYGZsq{}}\PYG{p}{)}
\PYG{g+go}{KeyError: Attribute does not exist.}
\PYG{g+go}{Available options are mineral\PYGZus{}type, boundary, platen\PYGZus{}force, platen\PYGZus{}displacement, gauge\PYGZus{}displacement\PYGZsq{}}
\end{sphinxVerbatim}

\end{description}\end{quote}

\end{fulllineitems}

\index{platen\_info() (openfdem.openfdem.Model method)@\spxentry{platen\_info()}\spxextra{openfdem.openfdem.Model method}}

\begin{fulllineitems}
\phantomsection\label{\detokenize{openfdem:openfdem.openfdem.Model.platen_info}}
\pysigstartsignatures
\pysiglinewithargsret{\sphinxbfcode{\sphinxupquote{platen\_info}}}{\emph{\DUrole{n}{pv\_cells}}, \emph{\DUrole{n}{platen\_boundary\_id}}, \emph{\DUrole{n}{var\_property}}}{}
\pysigstopsignatures
\sphinxAtStartPar
This function thresholds cells based on boundary condition and sums them based on the defined parameter var\_property
\begin{quote}\begin{description}
\sphinxlineitem{Parameters}\begin{itemize}
\item {} 
\sphinxAtStartPar
\sphinxstyleliteralstrong{\sphinxupquote{pv\_cells}} (\sphinxstyleliteralemphasis{\sphinxupquote{pyvista.core.pointset.UnstructuredGrid}}\sphinxstyleliteralemphasis{\sphinxupquote{ or }}\sphinxstyleliteralemphasis{\sphinxupquote{DataSet}}) \textendash{} 

\item {} 
\sphinxAtStartPar
\sphinxstyleliteralstrong{\sphinxupquote{platen\_boundary\_id}} (\sphinxstyleliteralemphasis{\sphinxupquote{float}}) \textendash{} boundary id that the threshold should be based on

\item {} 
\sphinxAtStartPar
\sphinxstyleliteralstrong{\sphinxupquote{var\_property}} (\sphinxstyleliteralemphasis{\sphinxupquote{str}}) \textendash{} name of the property (array to b returned)

\end{itemize}

\sphinxlineitem{Returns}
\sphinxAtStartPar
array of the property based on the threshold

\sphinxlineitem{Return type}
\sphinxAtStartPar
ndarray

\end{description}\end{quote}

\end{fulllineitems}

\index{plot\_stress\_strain() (openfdem.openfdem.Model method)@\spxentry{plot\_stress\_strain()}\spxextra{openfdem.openfdem.Model method}}

\begin{fulllineitems}
\phantomsection\label{\detokenize{openfdem:openfdem.openfdem.Model.plot_stress_strain}}
\pysigstartsignatures
\pysiglinewithargsret{\sphinxbfcode{\sphinxupquote{plot\_stress\_strain}}}{\emph{\DUrole{n}{strain}}, \emph{\DUrole{n}{stress}}, \emph{\DUrole{n}{ax}\DUrole{o}{=}\DUrole{default_value}{None}}, \emph{\DUrole{o}{**}\DUrole{n}{plt\_kwargs}}}{}
\pysigstopsignatures
\sphinxAtStartPar
Simple plot of the stress\sphinxhyphen{}strain curve of a given dataframe
\begin{quote}\begin{description}
\sphinxlineitem{Parameters}\begin{itemize}
\item {} 
\sphinxAtStartPar
\sphinxstyleliteralstrong{\sphinxupquote{strain}} (\sphinxstyleliteralemphasis{\sphinxupquote{pandas.DataFrame}}) \textendash{} X\sphinxhyphen{}axis data {[}Strain{]}

\item {} 
\sphinxAtStartPar
\sphinxstyleliteralstrong{\sphinxupquote{stress}} (\sphinxstyleliteralemphasis{\sphinxupquote{pandas.DataFrame}}) \textendash{} Y\sphinxhyphen{}axis data {[}Stress{]}

\item {} 
\sphinxAtStartPar
\sphinxstyleliteralstrong{\sphinxupquote{ax}} (\sphinxstyleliteralemphasis{\sphinxupquote{matplotlib}}) \textendash{} Matplotlib Axis

\item {} 
\sphinxAtStartPar
\sphinxstyleliteralstrong{\sphinxupquote{plt\_kwargs}} \textendash{} \sphinxtitleref{\textasciitilde{}matplotlib.Modules} submodules

\end{itemize}

\sphinxlineitem{Returns}
\sphinxAtStartPar
Matplotlib AxesSubplots

\sphinxlineitem{Return type}
\sphinxAtStartPar
Matplotlib Axis

\sphinxlineitem{Example}
\begin{sphinxVerbatim}[commandchars=\\\{\}]
\PYG{g+gp}{\PYGZgt{}\PYGZgt{}\PYGZgt{} }\PYG{k+kn}{import} \PYG{n+nn}{openfdem} \PYG{k}{as} \PYG{n+nn}{fdem}
\PYG{g+gp}{\PYGZgt{}\PYGZgt{}\PYGZgt{} }\PYG{n}{data} \PYG{o}{=} \PYG{n}{fdem}\PYG{o}{.}\PYG{n}{Model}\PYG{p}{(}\PYG{l+s+s2}{\PYGZdq{}}\PYG{l+s+s2}{../example\PYGZus{}outputs/Irazu\PYGZus{}UCS}\PYG{l+s+s2}{\PYGZdq{}}\PYG{p}{)}
\PYG{g+go}{\PYGZsh{} Minimal Arguments}
\PYG{g+gp}{\PYGZgt{}\PYGZgt{}\PYGZgt{} }\PYG{n}{df\PYGZus{}wo\PYGZus{}SG} \PYG{o}{=} \PYG{n}{data}\PYG{o}{.}\PYG{n}{complete\PYGZus{}UCS\PYGZus{}stress\PYGZus{}strain}\PYG{p}{(}\PYG{p}{)}
\PYG{g+go}{Columns:}
\PYG{g+go}{    Name: Platen Stress, dtype=float64, nullable: False}
\PYG{g+go}{    Name: Platen Strain, dtype=float64, nullable: False.}
\PYG{g+go}{    Script Identifying Platen}
\PYG{g+go}{Platen Material ID found as [1]}
\PYG{g+go}{    UCS Simulation}
\PYG{g+go}{        Predefined loading Axis [1] is Y\PYGZhy{}direction}
\PYG{g+go}{    Values used in calculations are}
\PYG{g+go}{        Area        52.00}
\PYG{g+go}{        Length      108.00}
\PYG{g+go}{Progress: |//////////////////////////////////////////////////| 100.0\PYGZpc{} Complete}
\PYG{g+gp}{\PYGZgt{}\PYGZgt{}\PYGZgt{} }\PYG{n}{data}\PYG{o}{.}\PYG{n}{plot\PYGZus{}stress\PYGZus{}strain}\PYG{p}{(}\PYG{n}{df\PYGZus{}wo\PYGZus{}SG}\PYG{p}{[}\PYG{l+s+s1}{\PYGZsq{}}\PYG{l+s+s1}{Platen Strain}\PYG{l+s+s1}{\PYGZsq{}}\PYG{p}{]}\PYG{p}{,} \PYG{n}{df\PYGZus{}wo\PYGZus{}SG}\PYG{p}{[}\PYG{l+s+s1}{\PYGZsq{}}\PYG{l+s+s1}{Platen Stress}\PYG{l+s+s1}{\PYGZsq{}}\PYG{p}{]}\PYG{p}{,} \PYG{n}{label}\PYG{o}{=}\PYG{l+s+s1}{\PYGZsq{}}\PYG{l+s+s1}{stress\PYGZhy{}strain}\PYG{l+s+s1}{\PYGZsq{}}\PYG{p}{,} \PYG{n}{color}\PYG{o}{=}\PYG{l+s+s1}{\PYGZsq{}}\PYG{l+s+s1}{green}\PYG{l+s+s1}{\PYGZsq{}}\PYG{p}{)}
\PYG{g+go}{\PYGZlt{}AxesSubplot:xlabel=\PYGZsq{}Strain (\PYGZhy{})\PYGZsq{}, ylabel=\PYGZsq{}Axial Stress (MPa)\PYGZsq{}\PYGZgt{}}
\end{sphinxVerbatim}

\end{description}\end{quote}

\end{fulllineitems}

\index{rock\_sample\_dimensions() (openfdem.openfdem.Model method)@\spxentry{rock\_sample\_dimensions()}\spxextra{openfdem.openfdem.Model method}}

\begin{fulllineitems}
\phantomsection\label{\detokenize{openfdem:openfdem.openfdem.Model.rock_sample_dimensions}}
\pysigstartsignatures
\pysiglinewithargsret{\sphinxbfcode{\sphinxupquote{rock\_sample\_dimensions}}}{\emph{\DUrole{n}{platen\_id}\DUrole{o}{=}\DUrole{default_value}{None}}}{}
\pysigstopsignatures
\sphinxAtStartPar
Lookup cell element ID on the top center and then trace points Using this information, we obtain the platen prop ID.
Alternatively the user can define the ID of thresholding
\begin{quote}\begin{description}
\sphinxlineitem{Parameters}
\sphinxAtStartPar
\sphinxstyleliteralstrong{\sphinxupquote{platen\_id}} (\sphinxstyleliteralemphasis{\sphinxupquote{None}}\sphinxstyleliteralemphasis{\sphinxupquote{ or }}\sphinxstyleliteralemphasis{\sphinxupquote{int}}) \textendash{} Manual override of Platen ID

\sphinxlineitem{Returns}
\sphinxAtStartPar
sample width, sample height, sample thickness

\sphinxlineitem{Return type}
\sphinxAtStartPar
tuple{[}float, float, float{]}

\sphinxlineitem{Example}
\begin{sphinxVerbatim}[commandchars=\\\{\}]
\PYG{g+gp}{\PYGZgt{}\PYGZgt{}\PYGZgt{} }\PYG{k+kn}{import} \PYG{n+nn}{openfdem} \PYG{k}{as} \PYG{n+nn}{fdem}
\PYG{g+gp}{\PYGZgt{}\PYGZgt{}\PYGZgt{} }\PYG{n}{data} \PYG{o}{=} \PYG{n}{fdem}\PYG{o}{.}\PYG{n}{Model}\PYG{p}{(}\PYG{l+s+s2}{\PYGZdq{}}\PYG{l+s+s2}{../example\PYGZus{}outputs/Irazu\PYGZus{}UCS}\PYG{l+s+s2}{\PYGZdq{}}\PYG{p}{)}
\PYG{g+gp}{\PYGZgt{}\PYGZgt{}\PYGZgt{} }\PYG{c+c1}{\PYGZsh{} Let the script try to identify the platen material ID}
\PYG{g+gp}{\PYGZgt{}\PYGZgt{}\PYGZgt{} }\PYG{n}{data}\PYG{o}{.}\PYG{n}{rock\PYGZus{}sample\PYGZus{}dimensions}\PYG{p}{(}\PYG{p}{)}
\PYG{g+go}{Script Identifying Platen}
\PYG{g+go}{    Platen Material ID found as [1]}
\PYG{g+go}{(52.0, 108.0, 0.0, [\PYGZhy{}26.0, 26.0, \PYGZhy{}54.0, 54.0, 0.0, 0.0])}
\PYG{g+gp}{\PYGZgt{}\PYGZgt{}\PYGZgt{} }\PYG{c+c1}{\PYGZsh{} Explicitly defined the platen material ID}
\PYG{g+gp}{\PYGZgt{}\PYGZgt{}\PYGZgt{} }\PYG{n}{data}\PYG{o}{.}\PYG{n}{rock\PYGZus{}sample\PYGZus{}dimensions}\PYG{p}{(}\PYG{l+m+mi}{0}\PYG{p}{)}
\PYG{g+go}{User Defined Platen ID}
\PYG{g+go}{    Platen Material ID found as 0}
\PYG{g+go}{(56.0, 116.0, 0.0, [\PYGZhy{}28.0, 28.0, \PYGZhy{}58.0, 58.0, 0.0, 0.0])}
\PYG{g+gp}{\PYGZgt{}\PYGZgt{}\PYGZgt{} }\PYG{c+c1}{\PYGZsh{} Explicitly defined the platen material ID is out of range}
\PYG{g+gp}{\PYGZgt{}\PYGZgt{}\PYGZgt{} }\PYG{n}{data}\PYG{o}{.}\PYG{n}{rock\PYGZus{}sample\PYGZus{}dimensions}\PYG{p}{(}\PYG{l+m+mi}{3}\PYG{p}{)}
\PYG{g+go}{IndexError: Threshold ID out of range.}
\PYG{g+go}{boundary Range \PYGZhy{}1\PYGZhy{}1}
\end{sphinxVerbatim}

\end{description}\end{quote}

\end{fulllineitems}

\index{simulation\_type() (openfdem.openfdem.Model method)@\spxentry{simulation\_type()}\spxextra{openfdem.openfdem.Model method}}

\begin{fulllineitems}
\phantomsection\label{\detokenize{openfdem:openfdem.openfdem.Model.simulation_type}}
\pysigstartsignatures
\pysiglinewithargsret{\sphinxbfcode{\sphinxupquote{simulation\_type}}}{}{}
\pysigstopsignatures
\sphinxAtStartPar
Identifies the type of simulation running. BD or UCS.
Checks the top left corner of the model. If it contains material it is assumed as a rectangle.
\begin{quote}\begin{description}
\sphinxlineitem{Returns}
\sphinxAtStartPar
Type of simulation. BD/UCS

\sphinxlineitem{Return type}
\sphinxAtStartPar
str

\sphinxlineitem{Example}
\begin{sphinxVerbatim}[commandchars=\\\{\}]
\PYG{g+gp}{\PYGZgt{}\PYGZgt{}\PYGZgt{} }\PYG{k+kn}{import} \PYG{n+nn}{openfdem} \PYG{k}{as} \PYG{n+nn}{fdem}
\PYG{g+gp}{\PYGZgt{}\PYGZgt{}\PYGZgt{} }\PYG{n}{data} \PYG{o}{=} \PYG{n}{fdem}\PYG{o}{.}\PYG{n}{Model}\PYG{p}{(}\PYG{l+s+s2}{\PYGZdq{}}\PYG{l+s+s2}{../example\PYGZus{}outputs/Irazu\PYGZus{}UCS}\PYG{l+s+s2}{\PYGZdq{}}\PYG{p}{)}
\PYG{g+gp}{\PYGZgt{}\PYGZgt{}\PYGZgt{} }\PYG{n}{data}\PYG{o}{.}\PYG{n}{rock\PYGZus{}sample\PYGZus{}dimensions}\PYG{p}{(}\PYG{p}{)}
\PYG{g+go}{Script Identifying Platen}
\PYG{g+go}{    Platen Material ID found as [1]}
\PYG{g+go}{(52.0, 108.0, 0.0, [\PYGZhy{}26.0, 26.0, \PYGZhy{}54.0, 54.0, 0.0, 0.0])}
\PYG{g+gp}{\PYGZgt{}\PYGZgt{}\PYGZgt{} }\PYG{n}{data}\PYG{o}{.}\PYG{n}{simulation\PYGZus{}type}\PYG{p}{(}\PYG{p}{)}
\PYG{g+go}{\PYGZsq{}UCS Simulation\PYGZsq{}}
\PYG{g+gp}{\PYGZgt{}\PYGZgt{}\PYGZgt{} }\PYG{n}{data\PYGZus{}bd} \PYG{o}{=} \PYG{n}{fdem}\PYG{o}{.}\PYG{n}{Model}\PYG{p}{(}\PYG{l+s+s2}{\PYGZdq{}}\PYG{l+s+s2}{../example\PYGZus{}outputs/openfdem\PYGZus{}BD}\PYG{l+s+s2}{\PYGZdq{}}\PYG{p}{,} \PYG{n}{runfile}\PYG{o}{=}\PYG{l+s+s1}{\PYGZsq{}}\PYG{l+s+s1}{.y}\PYG{l+s+s1}{\PYGZsq{}}\PYG{p}{)}
\PYG{g+gp}{\PYGZgt{}\PYGZgt{}\PYGZgt{} }\PYG{n}{data\PYGZus{}bd}\PYG{o}{.}\PYG{n}{rock\PYGZus{}sample\PYGZus{}dimensions}\PYG{p}{(}\PYG{p}{)}
\PYG{g+go}{Script Identifying Platen}
\PYG{g+go}{    Platen Material ID found as [1]}
\PYG{g+go}{(100.0, 99.94999694824219, 0.0, [\PYGZhy{}50.0, 50.0, \PYGZhy{}49.974998474121094, 49.974998474121094, 0.0, 0.0])}
\PYG{g+gp}{\PYGZgt{}\PYGZgt{}\PYGZgt{} }\PYG{n}{data\PYGZus{}bd}\PYG{o}{.}\PYG{n}{simulation\PYGZus{}type}\PYG{p}{(}\PYG{p}{)}
\PYG{g+go}{\PYGZsq{}BD Simulation\PYGZsq{}}
\end{sphinxVerbatim}

\end{description}\end{quote}

\end{fulllineitems}

\index{threshold\_bound\_check() (openfdem.openfdem.Model method)@\spxentry{threshold\_bound\_check()}\spxextra{openfdem.openfdem.Model method}}

\begin{fulllineitems}
\phantomsection\label{\detokenize{openfdem:openfdem.openfdem.Model.threshold_bound_check}}
\pysigstartsignatures
\pysiglinewithargsret{\sphinxbfcode{\sphinxupquote{threshold\_bound\_check}}}{\emph{\DUrole{n}{thres\_id}}, \emph{\DUrole{n}{thres\_array}\DUrole{o}{=}\DUrole{default_value}{\textquotesingle{}boundary\textquotesingle{}}}}{}
\pysigstopsignatures
\sphinxAtStartPar
Checks the material ID is a valid choice.
\begin{quote}\begin{description}
\sphinxlineitem{Parameters}\begin{itemize}
\item {} 
\sphinxAtStartPar
\sphinxstyleliteralstrong{\sphinxupquote{thres\_id}} (\sphinxstyleliteralemphasis{\sphinxupquote{int}}) \textendash{} ID of the item ot be threshold

\item {} 
\sphinxAtStartPar
\sphinxstyleliteralstrong{\sphinxupquote{thres\_array}} (\sphinxstyleliteralemphasis{\sphinxupquote{str}}) \textendash{} Array name of the item ot be threshold

\end{itemize}

\sphinxlineitem{Returns}
\sphinxAtStartPar
ID of the material

\sphinxlineitem{Return type}
\sphinxAtStartPar
int

\sphinxlineitem{Raises}
\sphinxAtStartPar
\sphinxstyleliteralstrong{\sphinxupquote{IndexError}} \textendash{} ID out of range.

\sphinxlineitem{Example}
\begin{sphinxVerbatim}[commandchars=\\\{\}]
\PYG{g+gp}{\PYGZgt{}\PYGZgt{}\PYGZgt{} }\PYG{k+kn}{import} \PYG{n+nn}{openfdem} \PYG{k}{as} \PYG{n+nn}{fdem}
\PYG{g+gp}{\PYGZgt{}\PYGZgt{}\PYGZgt{} }\PYG{n}{model} \PYG{o}{=} \PYG{n}{fdem}\PYG{o}{.}\PYG{n}{Model}\PYG{p}{(}\PYG{l+s+s2}{\PYGZdq{}}\PYG{l+s+s2}{../example\PYGZus{}outputs/Irazu\PYGZus{}UCS}\PYG{l+s+s2}{\PYGZdq{}}\PYG{p}{)}
\PYG{g+gp}{\PYGZgt{}\PYGZgt{}\PYGZgt{} }\PYG{n}{model}\PYG{o}{.}\PYG{n}{threshold\PYGZus{}bound\PYGZus{}check}\PYG{p}{(}\PYG{l+m+mi}{0}\PYG{p}{)}
\PYG{g+go}{0}
\PYG{g+gp}{\PYGZgt{}\PYGZgt{}\PYGZgt{} }\PYG{n}{model}\PYG{o}{.}\PYG{n}{threshold\PYGZus{}bound\PYGZus{}check}\PYG{p}{(}\PYG{l+m+mi}{5}\PYG{p}{)}
\PYG{g+go}{IndexError: Material ID for platen out of range.}
\PYG{g+go}{Material Range 0\PYGZhy{}1}
\end{sphinxVerbatim}

\end{description}\end{quote}

\end{fulllineitems}

\index{unpack\_DataFrame() (openfdem.openfdem.Model method)@\spxentry{unpack\_DataFrame()}\spxextra{openfdem.openfdem.Model method}}

\begin{fulllineitems}
\phantomsection\label{\detokenize{openfdem:openfdem.openfdem.Model.unpack_DataFrame}}
\pysigstartsignatures
\pysiglinewithargsret{\sphinxbfcode{\sphinxupquote{unpack\_DataFrame}}}{\emph{\DUrole{n}{packed\_cell\_info}}}{}
\pysigstopsignatures
\sphinxAtStartPar
Unpacking of the original array produced by pyvista
If the array is a point data, the array is suffixed with \_Nx where x is the node on that cell.
\begin{quote}\begin{description}
\sphinxlineitem{Parameters}
\sphinxAtStartPar
\sphinxstyleliteralstrong{\sphinxupquote{packed\_cell\_info}} (\sphinxstyleliteralemphasis{\sphinxupquote{pandas.DataFrame}}) \textendash{} 

\sphinxlineitem{Returns}
\sphinxAtStartPar
Unpacked DataFrame

\sphinxlineitem{Return type}
\sphinxAtStartPar
pandas.DataFrame

\end{description}\end{quote}

\end{fulllineitems}


\end{fulllineitems}

\index{Timestep (class in openfdem.openfdem)@\spxentry{Timestep}\spxextra{class in openfdem.openfdem}}

\begin{fulllineitems}
\phantomsection\label{\detokenize{openfdem:openfdem.openfdem.Timestep}}
\pysigstartsignatures
\pysiglinewithargsret{\sphinxbfcode{\sphinxupquote{class\DUrole{w}{  }}}\sphinxcode{\sphinxupquote{openfdem.openfdem.}}\sphinxbfcode{\sphinxupquote{Timestep}}}{\emph{\DUrole{n}{file}}, \emph{\DUrole{n}{runfile}\DUrole{o}{=}\DUrole{default_value}{None}}}{}
\pysigstopsignatures
\sphinxAtStartPar
Bases: \sphinxcode{\sphinxupquote{object}}

\sphinxAtStartPar
A class handling the data of each timestep.

\sphinxAtStartPar
Each data array returns for only the timestep
handles spatial manipulations

\end{fulllineitems}



\subsection{openfdem.aggregate\_storage module}
\label{\detokenize{openfdem:module-openfdem.aggregate_storage}}\label{\detokenize{openfdem:openfdem-aggregate-storage-module}}\index{module@\spxentry{module}!openfdem.aggregate\_storage@\spxentry{openfdem.aggregate\_storage}}\index{openfdem.aggregate\_storage@\spxentry{openfdem.aggregate\_storage}!module@\spxentry{module}}\index{aggregate\_storage (class in openfdem.aggregate\_storage)@\spxentry{aggregate\_storage}\spxextra{class in openfdem.aggregate\_storage}}

\begin{fulllineitems}
\phantomsection\label{\detokenize{openfdem:openfdem.aggregate_storage.aggregate_storage}}
\pysigstartsignatures
\pysiglinewithargsret{\sphinxbfcode{\sphinxupquote{class\DUrole{w}{  }}}\sphinxcode{\sphinxupquote{openfdem.aggregate\_storage.}}\sphinxbfcode{\sphinxupquote{aggregate\_storage}}}{\emph{\DUrole{n}{file\_directory}}, \emph{\DUrole{n}{h5filename}\DUrole{o}{=}\DUrole{default_value}{None}}, \emph{\DUrole{n}{overwrite}\DUrole{o}{=}\DUrole{default_value}{False}}, \emph{\DUrole{n}{compression}\DUrole{o}{=}\DUrole{default_value}{None}}, \emph{\DUrole{n}{verbose}\DUrole{o}{=}\DUrole{default_value}{True}}}{}
\pysigstopsignatures
\sphinxAtStartPar
Bases: \sphinxcode{\sphinxupquote{object}}

\sphinxAtStartPar
Aggregator class to store VTK files in a single h5 file for faster access to data.
\index{file\_group\_key() (openfdem.aggregate\_storage.aggregate\_storage method)@\spxentry{file\_group\_key()}\spxextra{openfdem.aggregate\_storage.aggregate\_storage method}}

\begin{fulllineitems}
\phantomsection\label{\detokenize{openfdem:openfdem.aggregate_storage.aggregate_storage.file_group_key}}
\pysigstartsignatures
\pysiglinewithargsret{\sphinxbfcode{\sphinxupquote{file\_group\_key}}}{\emph{\DUrole{n}{vtkfilename}}}{}
\pysigstopsignatures
\sphinxAtStartPar
Produces a standard group/key based on VTK file name
\begin{quote}\begin{description}
\sphinxlineitem{Parameters}
\sphinxAtStartPar
\sphinxstyleliteralstrong{\sphinxupquote{vtkfilename}} (\sphinxstyleliteralemphasis{\sphinxupquote{str path}}) \textendash{} VTK file name to be stored/read

\sphinxlineitem{Returns}
\sphinxAtStartPar
Key described using timestep and filename

\sphinxlineitem{Return type}
\sphinxAtStartPar
str

\end{description}\end{quote}

\end{fulllineitems}

\index{read\_file() (openfdem.aggregate\_storage.aggregate\_storage method)@\spxentry{read\_file()}\spxextra{openfdem.aggregate\_storage.aggregate\_storage method}}

\begin{fulllineitems}
\phantomsection\label{\detokenize{openfdem:openfdem.aggregate_storage.aggregate_storage.read_file}}
\pysigstartsignatures
\pysiglinewithargsret{\sphinxbfcode{\sphinxupquote{read\_file}}}{\emph{\DUrole{n}{filename}}, \emph{\DUrole{n}{verbose}\DUrole{o}{=}\DUrole{default_value}{False}}}{}
\pysigstopsignatures
\sphinxAtStartPar
Extract VTK file from HDF5 file given original filename

\sphinxAtStartPar
The VTK file is reconstructed from the data arrays stored in the HDF5 file. It will be similar but different from the original.
\begin{quote}\begin{description}
\sphinxlineitem{Parameters}\begin{itemize}
\item {} 
\sphinxAtStartPar
\sphinxstyleliteralstrong{\sphinxupquote{filename}} (\sphinxstyleliteralemphasis{\sphinxupquote{str path}}) \textendash{} File name to be extracted (unaltered since HDF5 file creation)

\item {} 
\sphinxAtStartPar
\sphinxstyleliteralstrong{\sphinxupquote{verbose}} (\sphinxstyleliteralemphasis{\sphinxupquote{bool}}\sphinxstyleliteralemphasis{\sphinxupquote{, }}\sphinxstyleliteralemphasis{\sphinxupquote{optional}}) \textendash{} Print progress statements, defaults to False

\end{itemize}

\sphinxlineitem{Returns}
\sphinxAtStartPar
VTK Unstructured Grid as if read from a {\color{red}\bfseries{}*}.vtp or {\color{red}\bfseries{}*}.vtu file

\sphinxlineitem{Return type}
\sphinxAtStartPar
VTK Unstructured Grid

\end{description}\end{quote}

\end{fulllineitems}

\index{store\_file() (openfdem.aggregate\_storage.aggregate\_storage method)@\spxentry{store\_file()}\spxextra{openfdem.aggregate\_storage.aggregate\_storage method}}

\begin{fulllineitems}
\phantomsection\label{\detokenize{openfdem:openfdem.aggregate_storage.aggregate_storage.store_file}}
\pysigstartsignatures
\pysiglinewithargsret{\sphinxbfcode{\sphinxupquote{store\_file}}}{\emph{\DUrole{n}{vtkfilename}}}{}
\pysigstopsignatures
\sphinxAtStartPar
Stores VTK file into HDF5 file
\begin{quote}\begin{description}
\sphinxlineitem{Parameters}
\sphinxAtStartPar
\sphinxstyleliteralstrong{\sphinxupquote{vtkfilename}} (\sphinxstyleliteralemphasis{\sphinxupquote{str path}}) \textendash{} VTK file name

\end{description}\end{quote}

\end{fulllineitems}


\end{fulllineitems}



\subsection{openfdem.complete\_BD\_thread\_pool\_generators module}
\label{\detokenize{openfdem:module-openfdem.complete_BD_thread_pool_generators}}\label{\detokenize{openfdem:openfdem-complete-bd-thread-pool-generators-module}}\index{module@\spxentry{module}!openfdem.complete\_BD\_thread\_pool\_generators@\spxentry{openfdem.complete\_BD\_thread\_pool\_generators}}\index{openfdem.complete\_BD\_thread\_pool\_generators@\spxentry{openfdem.complete\_BD\_thread\_pool\_generators}!module@\spxentry{module}}\index{history\_strain\_func() (in module openfdem.complete\_BD\_thread\_pool\_generators)@\spxentry{history\_strain\_func()}\spxextra{in module openfdem.complete\_BD\_thread\_pool\_generators}}

\begin{fulllineitems}
\phantomsection\label{\detokenize{openfdem:openfdem.complete_BD_thread_pool_generators.history_strain_func}}
\pysigstartsignatures
\pysiglinewithargsret{\sphinxcode{\sphinxupquote{openfdem.complete\_BD\_thread\_pool\_generators.}}\sphinxbfcode{\sphinxupquote{history\_strain\_func}}}{\emph{\DUrole{n}{f\_name}}, \emph{\DUrole{n}{model}}, \emph{\DUrole{n}{cv}}, \emph{\DUrole{n}{ch}}}{}
\pysigstopsignatures
\sphinxAtStartPar
Calculate the axial stress from platens, axial strain from platens and SG as well as lateral strain from SG
\begin{quote}\begin{description}
\sphinxlineitem{Parameters}\begin{itemize}
\item {} 
\sphinxAtStartPar
\sphinxstyleliteralstrong{\sphinxupquote{f\_name}} (\sphinxstyleliteralemphasis{\sphinxupquote{str}}) \textendash{} name of vtu file being processed

\item {} 
\sphinxAtStartPar
\sphinxstyleliteralstrong{\sphinxupquote{model}} ({\hyperref[\detokenize{openfdem:openfdem.openfdem.Model}]{\sphinxcrossref{\sphinxstyleliteralemphasis{\sphinxupquote{openfdem.openfdem.Model}}}}}) \textendash{} FDEM Model Class

\item {} 
\sphinxAtStartPar
\sphinxstyleliteralstrong{\sphinxupquote{cv}} (\sphinxstyleliteralemphasis{\sphinxupquote{list}}) \textendash{} list of cells at the corner of the vertical strain gauge

\item {} 
\sphinxAtStartPar
\sphinxstyleliteralstrong{\sphinxupquote{ch}} (\sphinxstyleliteralemphasis{\sphinxupquote{list}}) \textendash{} list of cells at the corner of the horizontal strain gauge

\end{itemize}

\sphinxlineitem{Returns}
\sphinxAtStartPar
Stress, Platen Strain, SG Strain, SG Lateral Strain

\sphinxlineitem{Return type}
\sphinxAtStartPar
Generator{[}Tuple{[}list, list, list, list{]}, Any, None{]}

\end{description}\end{quote}

\end{fulllineitems}

\index{main() (in module openfdem.complete\_BD\_thread\_pool\_generators)@\spxentry{main()}\spxextra{in module openfdem.complete\_BD\_thread\_pool\_generators}}

\begin{fulllineitems}
\phantomsection\label{\detokenize{openfdem:openfdem.complete_BD_thread_pool_generators.main}}
\pysigstartsignatures
\pysiglinewithargsret{\sphinxcode{\sphinxupquote{openfdem.complete\_BD\_thread\_pool\_generators.}}\sphinxbfcode{\sphinxupquote{main}}}{\emph{\DUrole{n}{model}}, \emph{\DUrole{n}{st\_status}}, \emph{\DUrole{n}{gauge\_width}}, \emph{\DUrole{n}{gauge\_length}}, \emph{\DUrole{n}{c\_center}}, \emph{\DUrole{n}{progress\_bar}\DUrole{o}{=}\DUrole{default_value}{False}}}{}
\pysigstopsignatures
\sphinxAtStartPar
Main concurrent Thread Pool to calculate the full stress\sphinxhyphen{}strain
\begin{quote}\begin{description}
\sphinxlineitem{Parameters}\begin{itemize}
\item {} 
\sphinxAtStartPar
\sphinxstyleliteralstrong{\sphinxupquote{model}} ({\hyperref[\detokenize{openfdem:openfdem.openfdem.Model}]{\sphinxcrossref{\sphinxstyleliteralemphasis{\sphinxupquote{openfdem.openfdem.Model}}}}}) \textendash{} FDEM Model Class

\item {} 
\sphinxAtStartPar
\sphinxstyleliteralstrong{\sphinxupquote{st\_status}} (\sphinxstyleliteralemphasis{\sphinxupquote{bool}}) \textendash{} Enable/Disable SG Calculations

\item {} 
\sphinxAtStartPar
\sphinxstyleliteralstrong{\sphinxupquote{gauge\_width}} (\sphinxstyleliteralemphasis{\sphinxupquote{float}}) \textendash{} SG width

\item {} 
\sphinxAtStartPar
\sphinxstyleliteralstrong{\sphinxupquote{gauge\_length}} (\sphinxstyleliteralemphasis{\sphinxupquote{float}}) \textendash{} SG length

\item {} 
\sphinxAtStartPar
\sphinxstyleliteralstrong{\sphinxupquote{c\_center}} (\sphinxstyleliteralemphasis{\sphinxupquote{None}}\sphinxstyleliteralemphasis{\sphinxupquote{ or }}\sphinxstyleliteralemphasis{\sphinxupquote{list}}\sphinxstyleliteralemphasis{\sphinxupquote{{[}}}\sphinxstyleliteralemphasis{\sphinxupquote{float}}\sphinxstyleliteralemphasis{\sphinxupquote{, }}\sphinxstyleliteralemphasis{\sphinxupquote{float}}\sphinxstyleliteralemphasis{\sphinxupquote{, }}\sphinxstyleliteralemphasis{\sphinxupquote{float}}\sphinxstyleliteralemphasis{\sphinxupquote{{]}}}) \textendash{} User\sphinxhyphen{}defined center of the SG

\item {} 
\sphinxAtStartPar
\sphinxstyleliteralstrong{\sphinxupquote{progress\_bar}} (\sphinxstyleliteralemphasis{\sphinxupquote{bool}}) \textendash{} Show/Hide progress bar

\end{itemize}

\sphinxlineitem{Returns}
\sphinxAtStartPar
full stress\sphinxhyphen{}strain information

\sphinxlineitem{Return type}
\sphinxAtStartPar
pd.DataFrame

\end{description}\end{quote}

\end{fulllineitems}

\index{set\_strain\_gauge() (in module openfdem.complete\_BD\_thread\_pool\_generators)@\spxentry{set\_strain\_gauge()}\spxextra{in module openfdem.complete\_BD\_thread\_pool\_generators}}

\begin{fulllineitems}
\phantomsection\label{\detokenize{openfdem:openfdem.complete_BD_thread_pool_generators.set_strain_gauge}}
\pysigstartsignatures
\pysiglinewithargsret{\sphinxcode{\sphinxupquote{openfdem.complete\_BD\_thread\_pool\_generators.}}\sphinxbfcode{\sphinxupquote{set\_strain\_gauge}}}{\emph{\DUrole{n}{model}}, \emph{\DUrole{n}{gauge\_length}\DUrole{o}{=}\DUrole{default_value}{None}}, \emph{\DUrole{n}{gauge\_width}\DUrole{o}{=}\DUrole{default_value}{None}}, \emph{\DUrole{n}{c\_center}\DUrole{o}{=}\DUrole{default_value}{None}}}{}
\pysigstopsignatures
\sphinxAtStartPar
Calculate local strain based on the dimensions of a virtual strain gauge placed at the center of teh model with
x/y dimensions. By default set to 0.25 of the length/width.
\begin{quote}\begin{description}
\sphinxlineitem{Parameters}\begin{itemize}
\item {} 
\sphinxAtStartPar
\sphinxstyleliteralstrong{\sphinxupquote{model}} ({\hyperref[\detokenize{openfdem:openfdem.openfdem.Model}]{\sphinxcrossref{\sphinxstyleliteralemphasis{\sphinxupquote{openfdem.openfdem.Model}}}}}) \textendash{} FDEM Model Class

\item {} 
\sphinxAtStartPar
\sphinxstyleliteralstrong{\sphinxupquote{gauge\_length}} (\sphinxstyleliteralemphasis{\sphinxupquote{float}}) \textendash{} length of the virtual strain gauge

\item {} 
\sphinxAtStartPar
\sphinxstyleliteralstrong{\sphinxupquote{gauge\_width}} (\sphinxstyleliteralemphasis{\sphinxupquote{float}}) \textendash{} width of the virtual strain gauge

\item {} 
\sphinxAtStartPar
\sphinxstyleliteralstrong{\sphinxupquote{c\_center}} (\sphinxstyleliteralemphasis{\sphinxupquote{None}}\sphinxstyleliteralemphasis{\sphinxupquote{ or }}\sphinxstyleliteralemphasis{\sphinxupquote{list}}\sphinxstyleliteralemphasis{\sphinxupquote{{[}}}\sphinxstyleliteralemphasis{\sphinxupquote{float}}\sphinxstyleliteralemphasis{\sphinxupquote{, }}\sphinxstyleliteralemphasis{\sphinxupquote{float}}\sphinxstyleliteralemphasis{\sphinxupquote{, }}\sphinxstyleliteralemphasis{\sphinxupquote{float}}\sphinxstyleliteralemphasis{\sphinxupquote{{]}}}) \textendash{} User\sphinxhyphen{}defined center of the SG

\end{itemize}

\sphinxlineitem{Returns}
\sphinxAtStartPar
Cells that cover the horizontal and vertical gauges as well as the gauge width and length

\sphinxlineitem{Return type}
\sphinxAtStartPar
{[}list, list, float, float{]}

\end{description}\end{quote}

\end{fulllineitems}



\subsection{openfdem.complete\_UCS\_thread\_pool\_generators module}
\label{\detokenize{openfdem:module-openfdem.complete_UCS_thread_pool_generators}}\label{\detokenize{openfdem:openfdem-complete-ucs-thread-pool-generators-module}}\index{module@\spxentry{module}!openfdem.complete\_UCS\_thread\_pool\_generators@\spxentry{openfdem.complete\_UCS\_thread\_pool\_generators}}\index{openfdem.complete\_UCS\_thread\_pool\_generators@\spxentry{openfdem.complete\_UCS\_thread\_pool\_generators}!module@\spxentry{module}}\index{check\_loading\_direction() (in module openfdem.complete\_UCS\_thread\_pool\_generators)@\spxentry{check\_loading\_direction()}\spxextra{in module openfdem.complete\_UCS\_thread\_pool\_generators}}

\begin{fulllineitems}
\phantomsection\label{\detokenize{openfdem:openfdem.complete_UCS_thread_pool_generators.check_loading_direction}}
\pysigstartsignatures
\pysiglinewithargsret{\sphinxcode{\sphinxupquote{openfdem.complete\_UCS\_thread\_pool\_generators.}}\sphinxbfcode{\sphinxupquote{check\_loading\_direction}}}{\emph{\DUrole{n}{model}}, \emph{\DUrole{n}{f1}}, \emph{\DUrole{n}{f2}}}{}
\pysigstopsignatures
\end{fulllineitems}

\index{history\_strain\_func() (in module openfdem.complete\_UCS\_thread\_pool\_generators)@\spxentry{history\_strain\_func()}\spxextra{in module openfdem.complete\_UCS\_thread\_pool\_generators}}

\begin{fulllineitems}
\phantomsection\label{\detokenize{openfdem:openfdem.complete_UCS_thread_pool_generators.history_strain_func}}
\pysigstartsignatures
\pysiglinewithargsret{\sphinxcode{\sphinxupquote{openfdem.complete\_UCS\_thread\_pool\_generators.}}\sphinxbfcode{\sphinxupquote{history\_strain\_func}}}{\emph{\DUrole{n}{f\_name}}, \emph{\DUrole{n}{model}}, \emph{\DUrole{n}{cv}}, \emph{\DUrole{n}{ch}}, \emph{\DUrole{n}{axis}}}{}
\pysigstopsignatures
\sphinxAtStartPar
Calculate the axial stress from platens, axial strain from platens and SG as well as lateral strain from SG
\begin{quote}\begin{description}
\sphinxlineitem{Parameters}\begin{itemize}
\item {} 
\sphinxAtStartPar
\sphinxstyleliteralstrong{\sphinxupquote{f\_name}} (\sphinxstyleliteralemphasis{\sphinxupquote{str}}) \textendash{} name of vtu file being processed

\item {} 
\sphinxAtStartPar
\sphinxstyleliteralstrong{\sphinxupquote{model}} ({\hyperref[\detokenize{openfdem:openfdem.openfdem.Model}]{\sphinxcrossref{\sphinxstyleliteralemphasis{\sphinxupquote{openfdem.openfdem.Model}}}}}) \textendash{} FDEM Model Class

\item {} 
\sphinxAtStartPar
\sphinxstyleliteralstrong{\sphinxupquote{cv}} (\sphinxstyleliteralemphasis{\sphinxupquote{list}}\sphinxstyleliteralemphasis{\sphinxupquote{{[}}}\sphinxstyleliteralemphasis{\sphinxupquote{int}}\sphinxstyleliteralemphasis{\sphinxupquote{{]}}}) \textendash{} list of cells at the corner of the vertical strain gauge

\item {} 
\sphinxAtStartPar
\sphinxstyleliteralstrong{\sphinxupquote{ch}} (\sphinxstyleliteralemphasis{\sphinxupquote{list}}\sphinxstyleliteralemphasis{\sphinxupquote{{[}}}\sphinxstyleliteralemphasis{\sphinxupquote{int}}\sphinxstyleliteralemphasis{\sphinxupquote{{]}}}) \textendash{} list of cells at the corner of the horizontal strain gauge

\end{itemize}

\sphinxlineitem{Returns}
\sphinxAtStartPar
Stress, Platen Strain, SG Strain, SG Lateral Strain

\sphinxlineitem{Return type}
\sphinxAtStartPar
Generator{[}Tuple{[}list, list, list, list{]}, Any, None{]}

\end{description}\end{quote}

\end{fulllineitems}

\index{main() (in module openfdem.complete\_UCS\_thread\_pool\_generators)@\spxentry{main()}\spxextra{in module openfdem.complete\_UCS\_thread\_pool\_generators}}

\begin{fulllineitems}
\phantomsection\label{\detokenize{openfdem:openfdem.complete_UCS_thread_pool_generators.main}}
\pysigstartsignatures
\pysiglinewithargsret{\sphinxcode{\sphinxupquote{openfdem.complete\_UCS\_thread\_pool\_generators.}}\sphinxbfcode{\sphinxupquote{main}}}{\emph{\DUrole{n}{model}}, \emph{\DUrole{n}{platen\_id}}, \emph{\DUrole{n}{st\_status}}, \emph{\DUrole{n}{axis\_of\_loading}}, \emph{\DUrole{n}{gauge\_width}}, \emph{\DUrole{n}{gauge\_length}}, \emph{\DUrole{n}{c\_center}}, \emph{\DUrole{n}{user\_samp\_A}\DUrole{o}{=}\DUrole{default_value}{None}}, \emph{\DUrole{n}{user\_samp\_L}\DUrole{o}{=}\DUrole{default_value}{None}}, \emph{\DUrole{n}{progress\_bar}\DUrole{o}{=}\DUrole{default_value}{False}}}{}
\pysigstopsignatures
\sphinxAtStartPar
Main concurrent Thread Pool to calculate the full stress\sphinxhyphen{}strain
\begin{quote}\begin{description}
\sphinxlineitem{Parameters}\begin{itemize}
\item {} 
\sphinxAtStartPar
\sphinxstyleliteralstrong{\sphinxupquote{model}} ({\hyperref[\detokenize{openfdem:openfdem.openfdem.Model}]{\sphinxcrossref{\sphinxstyleliteralemphasis{\sphinxupquote{openfdem.openfdem.Model}}}}}) \textendash{} FDEM Model Class

\item {} 
\sphinxAtStartPar
\sphinxstyleliteralstrong{\sphinxupquote{platen\_id}} (\sphinxstyleliteralemphasis{\sphinxupquote{None}}\sphinxstyleliteralemphasis{\sphinxupquote{ or }}\sphinxstyleliteralemphasis{\sphinxupquote{int}}) \textendash{} Manual override of Platen ID

\item {} 
\sphinxAtStartPar
\sphinxstyleliteralstrong{\sphinxupquote{st\_status}} (\sphinxstyleliteralemphasis{\sphinxupquote{bool}}) \textendash{} Enable/Disable SG Calculations

\item {} 
\sphinxAtStartPar
\sphinxstyleliteralstrong{\sphinxupquote{axis\_of\_loading}} (\sphinxstyleliteralemphasis{\sphinxupquote{None}}\sphinxstyleliteralemphasis{\sphinxupquote{ or }}\sphinxstyleliteralemphasis{\sphinxupquote{int}}) \textendash{} Enable/Disable SG

\item {} 
\sphinxAtStartPar
\sphinxstyleliteralstrong{\sphinxupquote{gauge\_width}} (\sphinxstyleliteralemphasis{\sphinxupquote{float}}) \textendash{} SG width

\item {} 
\sphinxAtStartPar
\sphinxstyleliteralstrong{\sphinxupquote{gauge\_length}} (\sphinxstyleliteralemphasis{\sphinxupquote{float}}) \textendash{} SG length

\item {} 
\sphinxAtStartPar
\sphinxstyleliteralstrong{\sphinxupquote{c\_center}} (\sphinxstyleliteralemphasis{\sphinxupquote{None}}\sphinxstyleliteralemphasis{\sphinxupquote{ or }}\sphinxstyleliteralemphasis{\sphinxupquote{list}}\sphinxstyleliteralemphasis{\sphinxupquote{{[}}}\sphinxstyleliteralemphasis{\sphinxupquote{float}}\sphinxstyleliteralemphasis{\sphinxupquote{, }}\sphinxstyleliteralemphasis{\sphinxupquote{float}}\sphinxstyleliteralemphasis{\sphinxupquote{, }}\sphinxstyleliteralemphasis{\sphinxupquote{float}}\sphinxstyleliteralemphasis{\sphinxupquote{{]}}}) \textendash{} User\sphinxhyphen{}defined center of the SG

\item {} 
\sphinxAtStartPar
\sphinxstyleliteralstrong{\sphinxupquote{user\_samp\_A}} (\sphinxstyleliteralemphasis{\sphinxupquote{None}}\sphinxstyleliteralemphasis{\sphinxupquote{ or }}\sphinxstyleliteralemphasis{\sphinxupquote{float}}) \textendash{} Sample Area

\item {} 
\sphinxAtStartPar
\sphinxstyleliteralstrong{\sphinxupquote{user\_samp\_L}} (\sphinxstyleliteralemphasis{\sphinxupquote{None}}\sphinxstyleliteralemphasis{\sphinxupquote{ or }}\sphinxstyleliteralemphasis{\sphinxupquote{float}}) \textendash{} Sample Length

\item {} 
\sphinxAtStartPar
\sphinxstyleliteralstrong{\sphinxupquote{progress\_bar}} (\sphinxstyleliteralemphasis{\sphinxupquote{bool}}) \textendash{} Show/Hide progress bar

\end{itemize}

\sphinxlineitem{Returns}
\sphinxAtStartPar
full stress\sphinxhyphen{}strain information

\sphinxlineitem{Return type}
\sphinxAtStartPar
pd.DataFrame

\end{description}\end{quote}

\end{fulllineitems}

\index{set\_strain\_gauge() (in module openfdem.complete\_UCS\_thread\_pool\_generators)@\spxentry{set\_strain\_gauge()}\spxextra{in module openfdem.complete\_UCS\_thread\_pool\_generators}}

\begin{fulllineitems}
\phantomsection\label{\detokenize{openfdem:openfdem.complete_UCS_thread_pool_generators.set_strain_gauge}}
\pysigstartsignatures
\pysiglinewithargsret{\sphinxcode{\sphinxupquote{openfdem.complete\_UCS\_thread\_pool\_generators.}}\sphinxbfcode{\sphinxupquote{set\_strain\_gauge}}}{\emph{\DUrole{n}{model}}, \emph{\DUrole{n}{gauge\_length}\DUrole{o}{=}\DUrole{default_value}{None}}, \emph{\DUrole{n}{gauge\_width}\DUrole{o}{=}\DUrole{default_value}{None}}, \emph{\DUrole{n}{c\_center}\DUrole{o}{=}\DUrole{default_value}{None}}}{}
\pysigstopsignatures
\sphinxAtStartPar
Calculate local strain based on the dimensions of a virtual strain gauge placed at the center of teh model with
x/y dimensions. By default, set to 0.25 of the length/width.
\begin{quote}\begin{description}
\sphinxlineitem{Parameters}\begin{itemize}
\item {} 
\sphinxAtStartPar
\sphinxstyleliteralstrong{\sphinxupquote{model}} ({\hyperref[\detokenize{openfdem:openfdem.openfdem.Model}]{\sphinxcrossref{\sphinxstyleliteralemphasis{\sphinxupquote{openfdem.openfdem.Model}}}}}) \textendash{} FDEM Model Class

\item {} 
\sphinxAtStartPar
\sphinxstyleliteralstrong{\sphinxupquote{gauge\_length}} (\sphinxstyleliteralemphasis{\sphinxupquote{float}}) \textendash{} length of the virtual strain gauge

\item {} 
\sphinxAtStartPar
\sphinxstyleliteralstrong{\sphinxupquote{gauge\_width}} (\sphinxstyleliteralemphasis{\sphinxupquote{float}}) \textendash{} width of the virtual strain gauge

\item {} 
\sphinxAtStartPar
\sphinxstyleliteralstrong{\sphinxupquote{c\_center}} (\sphinxstyleliteralemphasis{\sphinxupquote{None}}\sphinxstyleliteralemphasis{\sphinxupquote{ or }}\sphinxstyleliteralemphasis{\sphinxupquote{list}}\sphinxstyleliteralemphasis{\sphinxupquote{{[}}}\sphinxstyleliteralemphasis{\sphinxupquote{float}}\sphinxstyleliteralemphasis{\sphinxupquote{, }}\sphinxstyleliteralemphasis{\sphinxupquote{float}}\sphinxstyleliteralemphasis{\sphinxupquote{, }}\sphinxstyleliteralemphasis{\sphinxupquote{float}}\sphinxstyleliteralemphasis{\sphinxupquote{{]}}}) \textendash{} User\sphinxhyphen{}defined center of the SG

\end{itemize}

\sphinxlineitem{Returns}
\sphinxAtStartPar
Cells that cover the horizontal and vertical gauges as well as the gauge width and length

\sphinxlineitem{Return type}
\sphinxAtStartPar
{[}list, list, float, float{]}

\end{description}\end{quote}

\end{fulllineitems}



\subsection{openfdem.extract\_cell\_thread\_pool\_generators module}
\label{\detokenize{openfdem:module-openfdem.extract_cell_thread_pool_generators}}\label{\detokenize{openfdem:openfdem-extract-cell-thread-pool-generators-module}}\index{module@\spxentry{module}!openfdem.extract\_cell\_thread\_pool\_generators@\spxentry{openfdem.extract\_cell\_thread\_pool\_generators}}\index{openfdem.extract\_cell\_thread\_pool\_generators@\spxentry{openfdem.extract\_cell\_thread\_pool\_generators}!module@\spxentry{module}}\index{history\_cellinfo\_func() (in module openfdem.extract\_cell\_thread\_pool\_generators)@\spxentry{history\_cellinfo\_func()}\spxextra{in module openfdem.extract\_cell\_thread\_pool\_generators}}

\begin{fulllineitems}
\phantomsection\label{\detokenize{openfdem:openfdem.extract_cell_thread_pool_generators.history_cellinfo_func}}
\pysigstartsignatures
\pysiglinewithargsret{\sphinxcode{\sphinxupquote{openfdem.extract\_cell\_thread\_pool\_generators.}}\sphinxbfcode{\sphinxupquote{history\_cellinfo\_func}}}{\emph{\DUrole{n}{f\_name}}, \emph{\DUrole{n}{model}}, \emph{\DUrole{n}{cell\_id}}, \emph{\DUrole{n}{array\_needed}}, \emph{\DUrole{n}{thres\_array}\DUrole{o}{=}\DUrole{default_value}{None}}}{}
\pysigstopsignatures
\sphinxAtStartPar
Generate a dictionary of the various array being interrogated for the said cell ID
\begin{quote}\begin{description}
\sphinxlineitem{Parameters}\begin{itemize}
\item {} 
\sphinxAtStartPar
\sphinxstyleliteralstrong{\sphinxupquote{f\_name}} (\sphinxstyleliteralemphasis{\sphinxupquote{str}}) \textendash{} name of vtu file being processed

\item {} 
\sphinxAtStartPar
\sphinxstyleliteralstrong{\sphinxupquote{model}} ({\hyperref[\detokenize{openfdem:openfdem.openfdem.Model}]{\sphinxcrossref{\sphinxstyleliteralemphasis{\sphinxupquote{openfdem.openfdem.Model}}}}}) \textendash{} FDEM Model Class

\item {} 
\sphinxAtStartPar
\sphinxstyleliteralstrong{\sphinxupquote{cell\_id}} (\sphinxstyleliteralemphasis{\sphinxupquote{int}}) \textendash{} ID of the cell from which the data needs to be extracted

\item {} 
\sphinxAtStartPar
\sphinxstyleliteralstrong{\sphinxupquote{array\_needed}} (\sphinxstyleliteralemphasis{\sphinxupquote{list}}\sphinxstyleliteralemphasis{\sphinxupquote{{[}}}\sphinxstyleliteralemphasis{\sphinxupquote{str}}\sphinxstyleliteralemphasis{\sphinxupquote{{]}}}) \textendash{} Name of the property to extract

\end{itemize}

\sphinxlineitem{Returns}
\sphinxAtStartPar
The value of the property from the cell being extracted

\sphinxlineitem{Return type}
\sphinxAtStartPar
Generator{[}Tuple(){]}

\end{description}\end{quote}

\end{fulllineitems}

\index{main() (in module openfdem.extract\_cell\_thread\_pool\_generators)@\spxentry{main()}\spxextra{in module openfdem.extract\_cell\_thread\_pool\_generators}}

\begin{fulllineitems}
\phantomsection\label{\detokenize{openfdem:openfdem.extract_cell_thread_pool_generators.main}}
\pysigstartsignatures
\pysiglinewithargsret{\sphinxcode{\sphinxupquote{openfdem.extract\_cell\_thread\_pool\_generators.}}\sphinxbfcode{\sphinxupquote{main}}}{\emph{\DUrole{n}{model}}, \emph{\DUrole{n}{cellid}}, \emph{\DUrole{n}{arrayname}}, \emph{\DUrole{n}{progress\_bar}\DUrole{o}{=}\DUrole{default_value}{False}}}{}
\pysigstopsignatures
\sphinxAtStartPar
Main concurrent Thread Pool to get value of the property from the cell being extracted
\begin{quote}\begin{description}
\sphinxlineitem{Parameters}\begin{itemize}
\item {} 
\sphinxAtStartPar
\sphinxstyleliteralstrong{\sphinxupquote{model}} ({\hyperref[\detokenize{openfdem:openfdem.openfdem.Model}]{\sphinxcrossref{\sphinxstyleliteralemphasis{\sphinxupquote{openfdem.openfdem.Model}}}}}) \textendash{} FDEM Model Class

\item {} 
\sphinxAtStartPar
\sphinxstyleliteralstrong{\sphinxupquote{cellid}} (\sphinxstyleliteralemphasis{\sphinxupquote{int}}) \textendash{} ID of the cell from which the data needs to be extracted

\item {} 
\sphinxAtStartPar
\sphinxstyleliteralstrong{\sphinxupquote{arrayname}} (\sphinxstyleliteralemphasis{\sphinxupquote{list}}\sphinxstyleliteralemphasis{\sphinxupquote{{[}}}\sphinxstyleliteralemphasis{\sphinxupquote{str}}\sphinxstyleliteralemphasis{\sphinxupquote{{]}}}) \textendash{} Name of the property to extract

\item {} 
\sphinxAtStartPar
\sphinxstyleliteralstrong{\sphinxupquote{progress\_bar}} \textendash{} Show/Hide progress bar

\end{itemize}

\sphinxlineitem{Returns}
\sphinxAtStartPar
DataFrame of the values of the property from the cell being extracted

\sphinxlineitem{Return type}
\sphinxAtStartPar
pandas.DataFrame

\end{description}\end{quote}

\end{fulllineitems}



\subsection{openfdem.formatting\_codes module}
\label{\detokenize{openfdem:module-openfdem.formatting_codes}}\label{\detokenize{openfdem:openfdem-formatting-codes-module}}\index{module@\spxentry{module}!openfdem.formatting\_codes@\spxentry{openfdem.formatting\_codes}}\index{openfdem.formatting\_codes@\spxentry{openfdem.formatting\_codes}!module@\spxentry{module}}\index{bold\_text() (in module openfdem.formatting\_codes)@\spxentry{bold\_text()}\spxextra{in module openfdem.formatting\_codes}}

\begin{fulllineitems}
\phantomsection\label{\detokenize{openfdem:openfdem.formatting_codes.bold_text}}
\pysigstartsignatures
\pysiglinewithargsret{\sphinxcode{\sphinxupquote{openfdem.formatting\_codes.}}\sphinxbfcode{\sphinxupquote{bold\_text}}}{\emph{\DUrole{n}{val}}}{}
\pysigstopsignatures
\sphinxAtStartPar
Returns text as bold
\begin{quote}\begin{description}
\sphinxlineitem{Parameters}
\sphinxAtStartPar
\sphinxstyleliteralstrong{\sphinxupquote{val}} (\sphinxstyleliteralemphasis{\sphinxupquote{str}}) \textendash{} Text

\sphinxlineitem{Returns}
\sphinxAtStartPar
Text as bold

\sphinxlineitem{Return type}
\sphinxAtStartPar
str

\end{description}\end{quote}

\end{fulllineitems}

\index{calc\_timer\_values() (in module openfdem.formatting\_codes)@\spxentry{calc\_timer\_values()}\spxextra{in module openfdem.formatting\_codes}}

\begin{fulllineitems}
\phantomsection\label{\detokenize{openfdem:openfdem.formatting_codes.calc_timer_values}}
\pysigstartsignatures
\pysiglinewithargsret{\sphinxcode{\sphinxupquote{openfdem.formatting\_codes.}}\sphinxbfcode{\sphinxupquote{calc\_timer\_values}}}{\emph{\DUrole{n}{end\_time}}}{}
\pysigstopsignatures
\sphinxAtStartPar
Function to calculate the time
\begin{quote}\begin{description}
\sphinxlineitem{Parameters}
\sphinxAtStartPar
\sphinxstyleliteralstrong{\sphinxupquote{end\_time}} (\sphinxstyleliteralemphasis{\sphinxupquote{float}}) \textendash{} Time (Difference in time in seconds)

\sphinxlineitem{Returns}
\sphinxAtStartPar
Time in minutes and seconds

\sphinxlineitem{Return type}
\sphinxAtStartPar
float

\end{description}\end{quote}

\end{fulllineitems}

\index{docstring\_creator() (in module openfdem.formatting\_codes)@\spxentry{docstring\_creator()}\spxextra{in module openfdem.formatting\_codes}}

\begin{fulllineitems}
\phantomsection\label{\detokenize{openfdem:openfdem.formatting_codes.docstring_creator}}
\pysigstartsignatures
\pysiglinewithargsret{\sphinxcode{\sphinxupquote{openfdem.formatting\_codes.}}\sphinxbfcode{\sphinxupquote{docstring\_creator}}}{\emph{\DUrole{n}{df}}}{}
\pysigstopsignatures
\sphinxAtStartPar
Write the example output for a docstring DataFrame
\begin{quote}\begin{description}
\sphinxlineitem{Parameters}
\sphinxAtStartPar
\sphinxstyleliteralstrong{\sphinxupquote{df}} (\sphinxstyleliteralemphasis{\sphinxupquote{pandas.DataFrame}}) \textendash{} DataFrame to be read

\sphinxlineitem{Returns}
\sphinxAtStartPar
prints the docstring and type for each element in the DataFrame

\sphinxlineitem{Return type}
\sphinxAtStartPar
str

\end{description}\end{quote}

\end{fulllineitems}

\index{green\_text() (in module openfdem.formatting\_codes)@\spxentry{green\_text()}\spxextra{in module openfdem.formatting\_codes}}

\begin{fulllineitems}
\phantomsection\label{\detokenize{openfdem:openfdem.formatting_codes.green_text}}
\pysigstartsignatures
\pysiglinewithargsret{\sphinxcode{\sphinxupquote{openfdem.formatting\_codes.}}\sphinxbfcode{\sphinxupquote{green\_text}}}{\emph{\DUrole{n}{val}}}{}
\pysigstopsignatures
\sphinxAtStartPar
Returns text as bold in green font color
\begin{quote}\begin{description}
\sphinxlineitem{Parameters}
\sphinxAtStartPar
\sphinxstyleliteralstrong{\sphinxupquote{val}} (\sphinxstyleliteralemphasis{\sphinxupquote{str}}) \textendash{} Text

\sphinxlineitem{Returns}
\sphinxAtStartPar
Text as bold in green font color

\sphinxlineitem{Return type}
\sphinxAtStartPar
str

\end{description}\end{quote}

\end{fulllineitems}

\index{print\_progress() (in module openfdem.formatting\_codes)@\spxentry{print\_progress()}\spxextra{in module openfdem.formatting\_codes}}

\begin{fulllineitems}
\phantomsection\label{\detokenize{openfdem:openfdem.formatting_codes.print_progress}}
\pysigstartsignatures
\pysiglinewithargsret{\sphinxcode{\sphinxupquote{openfdem.formatting\_codes.}}\sphinxbfcode{\sphinxupquote{print\_progress}}}{\emph{\DUrole{n}{iteration}}, \emph{\DUrole{n}{total}}, \emph{\DUrole{n}{prefix}\DUrole{o}{=}\DUrole{default_value}{\textquotesingle{}\textquotesingle{}}}, \emph{\DUrole{n}{suffix}\DUrole{o}{=}\DUrole{default_value}{\textquotesingle{}\textquotesingle{}}}, \emph{\DUrole{n}{decimals}\DUrole{o}{=}\DUrole{default_value}{1}}, \emph{\DUrole{n}{bar\_length}\DUrole{o}{=}\DUrole{default_value}{50}}}{}
\pysigstopsignatures
\sphinxAtStartPar
Call in a loop to create terminal progress bar
Adjusted bar length to 50, to display on small screen
\begin{quote}\begin{description}
\sphinxlineitem{Parameters}\begin{itemize}
\item {} 
\sphinxAtStartPar
\sphinxstyleliteralstrong{\sphinxupquote{iteration}} (\sphinxstyleliteralemphasis{\sphinxupquote{int}}) \textendash{} current iteration

\item {} 
\sphinxAtStartPar
\sphinxstyleliteralstrong{\sphinxupquote{total}} (\sphinxstyleliteralemphasis{\sphinxupquote{int}}) \textendash{} total iteration

\item {} 
\sphinxAtStartPar
\sphinxstyleliteralstrong{\sphinxupquote{prefix}} (\sphinxstyleliteralemphasis{\sphinxupquote{str}}) \textendash{} prefix string

\item {} 
\sphinxAtStartPar
\sphinxstyleliteralstrong{\sphinxupquote{suffix}} (\sphinxstyleliteralemphasis{\sphinxupquote{str}}) \textendash{} suffix string

\item {} 
\sphinxAtStartPar
\sphinxstyleliteralstrong{\sphinxupquote{decimals}} (\sphinxstyleliteralemphasis{\sphinxupquote{int}}) \textendash{} positive number of decimals in percent complete

\item {} 
\sphinxAtStartPar
\sphinxstyleliteralstrong{\sphinxupquote{bar\_length}} (\sphinxstyleliteralemphasis{\sphinxupquote{int}}) \textendash{} character length of bar

\end{itemize}

\sphinxlineitem{Returns}
\sphinxAtStartPar
system output showing progress

\sphinxlineitem{Return type}
\sphinxAtStartPar


\end{description}\end{quote}

\end{fulllineitems}

\index{red\_text() (in module openfdem.formatting\_codes)@\spxentry{red\_text()}\spxextra{in module openfdem.formatting\_codes}}

\begin{fulllineitems}
\phantomsection\label{\detokenize{openfdem:openfdem.formatting_codes.red_text}}
\pysigstartsignatures
\pysiglinewithargsret{\sphinxcode{\sphinxupquote{openfdem.formatting\_codes.}}\sphinxbfcode{\sphinxupquote{red\_text}}}{\emph{\DUrole{n}{val}}}{}
\pysigstopsignatures
\sphinxAtStartPar
Returns text as bold in red font color
\begin{quote}\begin{description}
\sphinxlineitem{Parameters}
\sphinxAtStartPar
\sphinxstyleliteralstrong{\sphinxupquote{val}} (\sphinxstyleliteralemphasis{\sphinxupquote{str}}) \textendash{} Text

\sphinxlineitem{Returns}
\sphinxAtStartPar
Text as bold in red font color

\sphinxlineitem{Return type}
\sphinxAtStartPar
str

\end{description}\end{quote}

\end{fulllineitems}



\subsection{openfdem.model\_reader module}
\label{\detokenize{openfdem:module-openfdem.model_reader}}\label{\detokenize{openfdem:openfdem-model-reader-module}}\index{module@\spxentry{module}!openfdem.model\_reader@\spxentry{openfdem.model\_reader}}\index{openfdem.model\_reader@\spxentry{openfdem.model\_reader}!module@\spxentry{module}}\index{mp\_read() (in module openfdem.model\_reader)@\spxentry{mp\_read()}\spxextra{in module openfdem.model\_reader}}

\begin{fulllineitems}
\phantomsection\label{\detokenize{openfdem:openfdem.model_reader.mp_read}}
\pysigstartsignatures
\pysiglinewithargsret{\sphinxcode{\sphinxupquote{openfdem.model\_reader.}}\sphinxbfcode{\sphinxupquote{mp\_read}}}{\emph{\DUrole{o}{*}\DUrole{n}{args}}, \emph{\DUrole{o}{**}\DUrole{n}{kwargs}}}{}
\pysigstopsignatures
\end{fulllineitems}

\index{multiprocess\_async() (in module openfdem.model\_reader)@\spxentry{multiprocess\_async()}\spxextra{in module openfdem.model\_reader}}

\begin{fulllineitems}
\phantomsection\label{\detokenize{openfdem:openfdem.model_reader.multiprocess_async}}
\pysigstartsignatures
\pysiglinewithargsret{\sphinxcode{\sphinxupquote{openfdem.model\_reader.}}\sphinxbfcode{\sphinxupquote{multiprocess\_async}}}{\emph{\DUrole{o}{*}\DUrole{n}{args}}, \emph{\DUrole{o}{**}\DUrole{n}{kwargs}}}{}
\pysigstopsignatures
\end{fulllineitems}

\index{multiprocess\_lib\_read() (in module openfdem.model\_reader)@\spxentry{multiprocess\_lib\_read()}\spxextra{in module openfdem.model\_reader}}

\begin{fulllineitems}
\phantomsection\label{\detokenize{openfdem:openfdem.model_reader.multiprocess_lib_read}}
\pysigstartsignatures
\pysiglinewithargsret{\sphinxcode{\sphinxupquote{openfdem.model\_reader.}}\sphinxbfcode{\sphinxupquote{multiprocess\_lib\_read}}}{\emph{\DUrole{o}{*}\DUrole{n}{args}}, \emph{\DUrole{o}{**}\DUrole{n}{kwargs}}}{}
\pysigstopsignatures
\end{fulllineitems}

\index{normal\_read() (in module openfdem.model\_reader)@\spxentry{normal\_read()}\spxextra{in module openfdem.model\_reader}}

\begin{fulllineitems}
\phantomsection\label{\detokenize{openfdem:openfdem.model_reader.normal_read}}
\pysigstartsignatures
\pysiglinewithargsret{\sphinxcode{\sphinxupquote{openfdem.model\_reader.}}\sphinxbfcode{\sphinxupquote{normal\_read}}}{\emph{\DUrole{o}{*}\DUrole{n}{args}}, \emph{\DUrole{o}{**}\DUrole{n}{kwargs}}}{}
\pysigstopsignatures
\end{fulllineitems}

\index{pv\_read() (in module openfdem.model\_reader)@\spxentry{pv\_read()}\spxextra{in module openfdem.model\_reader}}

\begin{fulllineitems}
\phantomsection\label{\detokenize{openfdem:openfdem.model_reader.pv_read}}
\pysigstartsignatures
\pysiglinewithargsret{\sphinxcode{\sphinxupquote{openfdem.model\_reader.}}\sphinxbfcode{\sphinxupquote{pv\_read}}}{\emph{\DUrole{o}{*}\DUrole{n}{args}}, \emph{\DUrole{o}{**}\DUrole{n}{kwargs}}}{}
\pysigstopsignatures
\end{fulllineitems}

\index{pv\_read\_queue() (in module openfdem.model\_reader)@\spxentry{pv\_read\_queue()}\spxextra{in module openfdem.model\_reader}}

\begin{fulllineitems}
\phantomsection\label{\detokenize{openfdem:openfdem.model_reader.pv_read_queue}}
\pysigstartsignatures
\pysiglinewithargsret{\sphinxcode{\sphinxupquote{openfdem.model\_reader.}}\sphinxbfcode{\sphinxupquote{pv\_read\_queue}}}{\emph{\DUrole{n}{list\_of\_files}}, \emph{\DUrole{n}{q}}}{}
\pysigstopsignatures
\end{fulllineitems}

\index{timed() (in module openfdem.model\_reader)@\spxentry{timed()}\spxextra{in module openfdem.model\_reader}}

\begin{fulllineitems}
\phantomsection\label{\detokenize{openfdem:openfdem.model_reader.timed}}
\pysigstartsignatures
\pysiglinewithargsret{\sphinxcode{\sphinxupquote{openfdem.model\_reader.}}\sphinxbfcode{\sphinxupquote{timed}}}{\emph{\DUrole{n}{func}}}{}
\pysigstopsignatures
\end{fulllineitems}



\subsection{Module contents}
\label{\detokenize{openfdem:module-openfdem}}\label{\detokenize{openfdem:module-contents}}\index{module@\spxentry{module}!openfdem@\spxentry{openfdem}}\index{openfdem@\spxentry{openfdem}!module@\spxentry{module}}

\chapter{Indices and tables}
\label{\detokenize{index:indices-and-tables}}\begin{itemize}
\item {} 
\sphinxAtStartPar
\DUrole{xref,std,std-ref}{genindex}

\item {} 
\sphinxAtStartPar
\DUrole{xref,std,std-ref}{modindex}

\item {} 
\sphinxAtStartPar
\DUrole{xref,std,std-ref}{search}

\end{itemize}


\renewcommand{\indexname}{Python Module Index}
\begin{sphinxtheindex}
\let\bigletter\sphinxstyleindexlettergroup
\bigletter{o}
\item\relax\sphinxstyleindexentry{openfdem}\sphinxstyleindexpageref{openfdem:\detokenize{module-openfdem}}
\item\relax\sphinxstyleindexentry{openfdem.aggregate\_storage}\sphinxstyleindexpageref{openfdem:\detokenize{module-openfdem.aggregate_storage}}
\item\relax\sphinxstyleindexentry{openfdem.complete\_BD\_thread\_pool\_generators}\sphinxstyleindexpageref{openfdem:\detokenize{module-openfdem.complete_BD_thread_pool_generators}}
\item\relax\sphinxstyleindexentry{openfdem.complete\_UCS\_thread\_pool\_generators}\sphinxstyleindexpageref{openfdem:\detokenize{module-openfdem.complete_UCS_thread_pool_generators}}
\item\relax\sphinxstyleindexentry{openfdem.extract\_cell\_thread\_pool\_generators}\sphinxstyleindexpageref{openfdem:\detokenize{module-openfdem.extract_cell_thread_pool_generators}}
\item\relax\sphinxstyleindexentry{openfdem.formatting\_codes}\sphinxstyleindexpageref{openfdem:\detokenize{module-openfdem.formatting_codes}}
\item\relax\sphinxstyleindexentry{openfdem.model\_reader}\sphinxstyleindexpageref{openfdem:\detokenize{module-openfdem.model_reader}}
\item\relax\sphinxstyleindexentry{openfdem.openfdem}\sphinxstyleindexpageref{openfdem:\detokenize{module-openfdem.openfdem}}
\end{sphinxtheindex}

\renewcommand{\indexname}{Index}
\printindex
\end{document}